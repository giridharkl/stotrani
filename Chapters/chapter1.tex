\section{\sanskrit काशी विश्वनाथ स्तोत्रम्}
\chandas
गङ्गातरङ्गरमणीयजटाकलापं\\
गौरीनिरन्तरविभूषितवामभागम् ।\\
नारायणप्रियमनङ्गमदापहारं\\
वाराणसीपुरपतिं भज विश्वनाथम् ॥१॥\\
\\
वाचामगोचरमनेकगुणस्वरूपं\\
वागीशविष्णुसुरसेवितपादपीठम् ।\\
वामेन विग्रहवरेण कलत्रवन्तं\\
वाराणसीपुरपतिं भज विश्वनाथम् ॥२॥\\
\\
भूताधिपं भुजगभूषणभूषिताङ्गं\\
व्याघ्राजिनाम्बरधरं जटिलं त्रिनेत्रम् ।\\
पाशाङ्कुशाभयवरप्रदशूलपाणिं\\
वाराणसीपुरपतिं भज विश्वनाथम् ॥३॥\\
\\
शीतांशुशोभितकिरीटविराजमानं\\
भालेक्षणानलविशोषितपञ्चबाणम् ।\\
नागाधिपारचितभासुरकर्णपुरं\\
वाराणसीपुरपतिं भज विश्वनाथम् ॥४॥\\
\\
पञ्चाननं दुरितमत्तमतङ्गजानां\\
नागान्तकं दनुजपुङ्गवपन्नगानाम् ।\\
दावानलं मरणशोकजराटवीनां\\
वाराणसीपुरपतिं भज विश्वनाथम् ॥५॥\\
\\
तेजोमयं सगुणनिर्गुणमद्वितीयम्\\
आनन्दकन्दमपराजितमप्रमेयम्\\
नागात्मकं सकलनिष्कलमात्मरूपं ।\\
वाराणसीपुरपतिं भज विश्वनाथम् ॥६॥\\
\\
रागादिदोषरहितं स्वजनानुरागं\\
वैराग्यशान्तिनिलयं गिरिजासहायम् ।\\
माधुर्यधैर्यसुभगं गरलाभिरामं\\
वाराणसीपुरपतिं भज विश्वनाथम् ॥७॥\\
\\
आशां विहाय परिहृत्य परस्य निन्दां\\
पापे रतिं च सुनिवार्य मनः समाधौ ।\\
आदाय हृत्कमलमध्यगतं परेशं\\
वाराणसीपुरपतिं भज विश्वनाथम् ॥८॥\\

\section{\sanskrit द्वादश ज्योतिर्लिङ्ग स्तोत्रम् }
\chandas
सौराष्ट्रदेशे विशदेऽतिरम्ये ज्योतिर्मयं चन्द्रकलावतंसम् |\\
भक्तिप्रदानाय कृपावतीर्णं तं सोमनाथं शरणं प्रपद्ये || १||\\
\\
श्रीशैलशृङ्गे विबुधातिसङ्गे तुलाद्रितुङ्गेऽपि मुदा वसन्तम् |\\
तमर्जुनं मल्लिकपूर्वमेकं नमामि संसारसमुद्रसेतुम् || २||\\
\\
अवन्तिकायां विहितावतारं मुक्तिप्रदानाय च सज्जनानाम् |\\
अकालमृत्योः परिरक्षणार्थं वन्दे महाकालमहासुरेशम् || ३||\\
\\
कावेरिकानर्मदयोः पवित्रे समागमे सज्जनतारणाय |\\
सदैवमान्धातृपुरे वसन्तमोङ्कारमीशं शिवमेकमीडे || ४||\\
\\
पूर्वोत्तरे प्रज्वलिकानिधाने सदा वसन्तं गिरिजासमेतम् |\\
सुरासुराराधितपादपद्मं श्रीवैद्यनाथं तमहं नमामि || ५||\\
\\
याम्ये सदङ्गे नगरेऽतिरम्ये विभूषिताङ्गं विविधैश्च भोगैः |\\
सद्भक्तिमुक्तिप्रदमीशमेकं श्रीनागनाथं शरणं प्रपद्ये || ६||\\
\\
महाद्रिपार्श्वे च तटे रमन्तं सम्पूज्यमानं सततं मुनीन्द्रैः |\\
सुरासुरैर्यक्ष महोरगाढ्यैः केदारमीशं शिवमेकमीडे || ७||\\
\\
सह्याद्रिशीर्षे विमले वसन्तं गोदावरितीरपवित्रदेशे |\\
यद्धर्शनात्पातकमाशु नाशं प्रयाति तं त्र्यम्बकमीशमीडे || ८||\\
\\
सुताम्रपर्णीजलराशियोगे निबध्य सेतुं विशिखैरसंख्यैः |\\
श्रीरामचन्द्रेण समर्पितं तं रामेश्वराख्यं नियतं नमामि || ९||\\
\\
यं डाकिनिशाकिनिकासमाजे निषेव्यमाणं पिशिताशनैश्च |\\
सदैव भीमादिपदप्रसिद्दं तं शङ्करं भक्तहितं नमामि || १०||\\
\\
सानन्दमानन्दवने वसन्तमानन्दकन्दं हतपापवृन्दम् |\\
वाराणसीनाथमनाथनाथं श्रीविश्वनाथं शरणं प्रपद्ये || ११||\\
\\
इलापुरे रम्यविशालकेऽस्मिन् समुल्लसन्तं च जगद्वरेण्यम् |\\
वन्दे महोदारतरस्वभावं घृष्णेश्वराख्यं शरणम् प्रपद्ये || १२||\\
\\
ज्योतिर्मयद्वादशलिङ्गकानां शिवात्मनां प्रोक्तमिदं क्रमेण |\\
स्तोत्रं पठित्वा मनुजोऽतिभक्त्या फलं तदालोक्य निजं भजेच्च ||\\
|| इति द्वादश ज्योतिर्लिङ्गस्तोत्रं संपूर्णम् ||\\

\section{\sanskrit निर्वाण अष्टकम् }
\chandas
मनो बुद्ध्यहंकारचित्तानि नाहम् न च श्रोत्र जिह्वे न च घ्राण नेत्रे \\
न च व्योम भूमिर् न तेजॊ न वायु: चिदानन्द रूप: शिवोऽहम् शिवॊऽहम् ॥\\ 
न च प्राण संज्ञो न वै पञ्चवायु: न वा सप्तधातुर् न वा पञ्चकोश: \\
न वाक्पाणिपादौ न चोपस्थपायू चिदानन्द रूप: शिवोऽहम् शिवॊऽहम् ॥ \\
न मे द्वेष रागौ न मे लोभ मोहौ मदो नैव मे नैव मात्सर्य भाव: \\
न धर्मो न चार्थो न कामो ना मोक्ष: चिदानन्द रूप: शिवोऽहम् शिवॊऽहम् ॥\\ 
न पुण्यं न पापं न सौख्यं न दु:खम् न मन्त्रो न तीर्थं न वेदा: न यज्ञा: \\
अहं भोजनं नैव भोज्यं न भोक्ता चिदानन्द रूप: शिवोऽहम् शिवॊऽहम् ॥\\ 
न मृत्युर् न शंका न मे जातिभेद: पिता नैव मे नैव माता न जन्म \\
न बन्धुर् न मित्रं गुरुर्नैव शिष्य: चिदानन्द रूप: शिवोऽहम् शिवॊऽहम् ॥\\ 
अहं निर्विकल्पॊ निराकार रूपॊ विभुत्वाच्च सर्वत्र सर्वेन्द्रियाणाम् \\
न चासंगतं नैव मुक्तिर् न मेय: चिदानन्द रूप: शिवोऽहम् शिवॊऽहम् ॥\\

\section{\sanskrit श्रीशिवाष्टकं }
\chandas
प्रभुं प्राणनाथं विभुं विश्वनाथं जगन्नाथनाथं सदानन्दभाजम् ।\\
भवद्भव्यभूतेश्वरं भूतनाथं शिवं शङ्करं शम्भुमीशानमीडे ॥ १॥\\
\\
गले रुण्डमालं तनौ सर्पजालं महाकालकालं गणेशाधिपालम् ।\\
जटाजूटगङ्गोत्तरङ्गैर्विशालं शिवं शङ्करं शम्भुमीशानमीडे ॥ २॥\\
\\
मुदामाकरं मण्डनं मण्डयन्तं महामण्डलं भस्मभूषाधरं तम् ।\\
अनादिह्यपारं महामोहहारं शिवं शङ्करं शम्भुमीशानमीडे ॥ ३॥\\
\\
वटाधोनिवासं महाट्टाट्टहासं महापापनाशं सदासुप्रकाशम् ।\\
गिरीशं गणेशं महेशं सुरेशं शिवं शङ्करं शम्भुमीशानमीडे ॥ ४॥\\
\\
गिरिन्द्रात्मजासंग्रहीतार्धदेहं गिरौ संस्थितं सर्वदा सन्नगेहम् ।\\
परब्रह्मब्रह्मादिभिर्वन्ध्यमानं शिवं शङ्करं शम्भुमीशानमीडे ॥ ५॥\\
\\
कपालं त्रिशूलं कराभ्यां दधानं पदाम्भोजनम्राय कामं ददानम् ।\\
बलीवर्दयानं सुराणां प्रधानं शिवं शङ्करं शम्भुमीशानमीडे ॥ ६॥\\
\\
शरच्चन्द्रगात्रं गुणानन्द पात्रं त्रिनेत्रं पवित्रं धनेशस्य मित्रम् ।\\
अपर्णाकलत्रं चरित्रं विचित्रं शिवं शङ्करं शम्भुमीशानमीडे ॥ ७॥\\
\\
हरं सर्पहारं चिता भूविहारं भवं वेदसारं सदा निर्विकारम् ।\\
श्मशाने वसन्तं मनोजं दहन्तं शिवं शङ्करं शम्भुमीशानमीडे ॥ ८॥\\
\\
स्तवं यः प्रभाते नरः शूलपाणे पठेत् सर्वदा भर्गभावानुरक्तः ।\\
स पुत्रं धनं धान्यमित्रं कलत्रं विचित्रं समासाद्य मोक्षं प्रयाति ॥ ९॥\\
॥ इति शिवाष्टकम् ॥\\

\section{\sanskrit दारिद्र्य दहन शिवस्तोत्रम् }
\sanskrit
विश्वेश्वराय नरकार्णव तारणाय कर्णामृताय शशिशेखरधारणाय ।\\ 
कर्पूरकान्तिधवलाय जटाधराय दारिद्र्य दुःखदहनाय नमः शिवाय ॥ १॥\\ 
\\
गौरीप्रियाय रजनीशकलाधराय कालान्तकाय भुजगाधिपकङ्कणाय । \\
गंगाधराय गजराजविमर्दनाय दारिद्र्य दुःखदहनाय नमः शिवाय ॥ २॥ \\
\\
भक्तिप्रियाय भवरोगभयापहाय उग्राय दुर्गभवसागरतारणाय । \\
ज्योतिर्मयाय गुणनामसुनृत्यकाय दारिद्र्य दुःखदहनाय नमः शिवाय ॥ ३॥\\ 
\\
चर्मम्बराय शवभस्मविलेपनाय भालेक्षणाय मणिकुण्डलमण्डिताय । \\
मंझीरपादयुगलाय जटाधराय दारिद्र्य दुःखदहनाय नमः शिवाय ॥ ४॥ \\
\\
पञ्चाननाय फणिराजविभूषणाय हेमांशुकाय भुवनत्रयमण्डिताय । \\
आनन्दभूमिवरदाय तमोमयाय दारिद्र्य दुःखदहनाय नमः शिवाय ॥ ५॥\\ 
\\
भानुप्रियाय भवसागरतारणाय कालान्तकाय कमलासनपूजिताय । \\
नेत्रत्रयाय शुभलक्षण लक्षिताय दारिद्र्य दुःखदहनाय नमः शिवाय ॥ ६॥ \\
\\
रामप्रियाय रघुनाथवरप्रदाय नागप्रियाय नरकार्णवतारणाय । \\
पुण्येषु पुण्यभरिताय सुरार्चिताय दारिद्र्य दुःखदहनाय नमः शिवाय ॥ ७॥\\ 
\\
मुक्तेश्वराय फलदाय गणेश्वराय गीतप्रियाय वृषभेश्वरवाहनाय । \\
मातङ्गचर्मवसनाय महेश्वराय दारिद्र्य दुःखदहनाय नमः शिवाय ॥ ८॥\\



