॥ श्रीहरिवायुस्तुतिः ॥ \\
॥ अथ श्रीनखस्तुतिः ॥  \\
\\
पान्त्वस्मान् पुरुहूतवैरि बलवन्मातङ्ग माद्यद्घटा\\
     कुम्भोच्चाद्रि विपाटनाधिकपटु प्रत्येक वज्रायिताः ।\\
श्रीमत्कण्ठीरवास्य प्रतत सुनखरा दारितारातिदूर\\
     प्रद्ध्वस्तध्वान्त शान्त प्रवितत मनसा भावितानाकिवृन्दैः ॥ १॥ भाविता भूरिभागैः\\
\\
लक्ष्मीकान्त समन्ततोऽपिकलयन् नैवेशितुस्ते समं\\
     पश्याम्युत्तम वस्तु दूरतरतोपास्तं रसोयोऽष्टमः ।\\
यद्रोशोत्कर दक्ष नेत्र कुटिल प्रान्तोत्थिताग्नि स्फुरत्\\
     खद्योतोपम विस्फुलिङ्गभसिता ब्रह्मेशशक्रोत्कराः ॥ २॥\\
\\
          इति श्रीमदानन्दतीर्थभगवत्पादाचार्यविरचिता\\
          श्रीनृसिंहनखस्तुतिः सम्पुर्णा ।\\
 ॥ अथ श्रीहरिवायुस्तुतिः ॥\\
\\
श्रीमद्विष्ण्वङ्घ्रि निष्ठा अतिगुणगुरुतम श्रीमदानन्दतीर्थ\\
     त्रैलोक्याचार्य पादोज्ज्वल जलजलसत् पांसवोऽस्मान्पुनन्तु ।\\
वाचांयत्रप्रणेत्रीत्रिभुवनमहिता शारदा शारदेन्दुः\\
     ज्योत्स्नाभद्रस्मित श्रीधवळितककुभाप्रेमभारम्बभार ॥ १॥\\
\\
उत्कण्ठाकुण्ठकोलाहलजवविदिताजस्रसेवानुवृद्ध\\
     प्राज्ञात्मज्ञान धूतान्धतमससुमनो मौलिरत्नावळीनाम् ।\\
भक्त्युद्रेकावगाढ प्रघटनसघटात्कार सङ्घृष्यमाण\\
     प्रान्तप्राग्र्याङ्घ्रि पीठोत्थित कनकरजः पिञ्जरारञ्जिताशाः ॥ २॥\\
\\
जन्माधिव्याध्युपाधिप्रतिहतिविरहप्रापकाणां गुणानाम्\\
     अग्र्याणां अर्पकाणां चिरमुदितचिदानन्द सन्दोहदानाम् ।\\
एतेषामेशदोष प्रमुषितमनसां द्वेषिणां दूषकाणाम्\\
     दैत्यानामार्थिमन्धे तमसि विदधतां संस्तवेनास्मि शक्तः ॥ ३॥\\
\\
अस्याविष्कर्तुकामं कलिमलकलुषेऽस्मिन्जनेज्ञानमार्गम्\\
     वन्द्यं चन्द्रेन्द्ररुद्र द्युमणिफणिवयोः नायकद्यैरिहाद्य ।\\
मध्वाख्यं मन्त्रसिद्धं किमुतकृतवतो मारुतस्यावतारम्\\
     पातारं पारमेष्ट्यं पदमपविपदः प्राप्तुरापन्न पुंसाम् ॥ ४॥\\
\\
उद्यद्विद्युत्प्रचण्डां निजरुचि निकरव्याप्त लोकावकाशो\\
     बिभ्रद्भीमो भुजेयोऽभ्युदित दिनकराभाङ्गदाढ्य प्रकाण्डे ।\\
वीर्योद्धार्यां गदाग्र्यामयमिह सुमतिंवायुदेवोविदध्यात्\\
     अध्यात्मज्ञाननेता यतिवरमहितो भूमिभूषामर्णिमे ॥ ५॥\\
\\
संसारोत्तापनित्योपशमद सदय स्नेहहासाम्बुपूर\\
     प्रोद्यद्विद्यावनद्य द्युतिमणिकिरण श्रेणिसम्पूरिताशः ।\\
श्रीवत्साङ्काधि वासोचित तरसरलश्रीमदानन्दतीर्थ\\
     क्षीराम्भोधिर्विभिन्द्याद्भवदनभिमतम्भूरिमेभूति हेतुः ॥ ६॥\\
\\
मूर्धन्येषोऽन्जलिर्मे दृढतरमिहते बध्यते बन्धपाश\\
     क्षेत्रेधात्रे सुखानां भजति भुवि भविष्यद्विधात्रे द्युभर्त्रे ।\\
अत्यन्तं सन्ततं त्वं प्रदिश पदयुगे हन्त सन्ताप भाजाम्\\
     अस्माकं भक्तिमेकां भगवत उतते माधवस्याथ वायोः ॥ ७॥\\
\\
साभ्रोष्णाभीशु शुभ्रप्रभमभयनभो भूरिभूभृद्विभूतिः\\
     भ्राजिष्णुर्भूरृभूणां भवनमपि विभोऽभेदिबभ्रेबभूवे ।\\
येनभ्रोविभ्रमस्ते भ्रमयतुसुभृशं बभ्रुवद्दुर्भृताशान्\\
     भ्रान्तिर्भेदाव भासस्त्वितिभयमभि भोर्भूक्ष्यतोमायिभिक्षून् ॥ ८॥\\
\\
येऽमुम्भावम्भजन्ते सुरमुखसुजनाराधितं ते तृतीयम्\\
     भासन्ते भासुरैस्ते सहचरचलितैश्चामरैश्चारुवेशाः ।\\
वैकुण्ठे कण्ठलग्न स्थिरशुचि विलसत्कान्ति तारुण्यलीला\\
     लावण्या पूर्णकान्ता कुचभरसुलभाश्लेषसम्मोदसान्द्राः ॥ ९॥\\
\\
आनन्दान्मन्दमन्दा ददति हि मरुतः कुन्दमन्दारनन्द्यावर्ता\\
     ऽमोदान् दधानां मृदुपद मुदितोद्गीतकैः सुन्दरीणाम् ।\\
वृन्दैरावन्द्य मुक्तेन्द्वहिमगु मदनाहीन्द्र देवेन्द्रसेव्ये\\
     मौकुन्दे मन्दरेऽस्मिन्नविरतमुदयन्मोदिनां देव देव ॥ १०॥\\
\\
उत्तप्तात्युत्कटत्विट् प्रकटकटकट ध्वानसङ्घट्टनोद्यद्\\
     विद्युद्व्यूढस्फुलिङ्ग प्रकर विकिरणोत्क्वाथिते बाधिताङ्गान् ।\\
उद्गाढम्पात्यमाना तमसि तत इतः किङ्करैः पङ्किलेते\\
     पङ्क्तिर्ग्राव्णां गरिम्णां ग्लपयति हि भवद्वेषिणो विद्वदाद्य ॥ ११॥\\
\\
अस्मिन्नस्मद्गुरूणां हरिचरण चिरध्यान सन्मङ्गलानाम्\\
     युष्माकं  पार्ष्वभूमिं धृतरणरणिकः स्वर्गिसेव्यांप्रपन्नः ।\\
यस्तूदास्ते स आस्तेऽधिभवमसुलभ क्लेश निर्मूकमस्त\\
     प्रायानन्दं कथं चिन्नवसति सततं पञ्चकष्टेऽतिकष्टे ॥ १२॥\\
\\
क्षुत् क्षामान् रूक्षरक्षो रदखरनखर क्षुण्णविक्षोभिताक्षान्\\
     आमग्नानान्धकूपे क्षुरमुखमुखरैः पक्षिभिर्विक्षताङ्गान् ।\\
पूयासृन्मूत्र विष्ठा क्रिमिकुलकलिलेतत्क्षणक्षिप्त शक्त्याद्यस्त्र\\
     व्रातार्दितान् स्त्वद्विष उपजिहते वज्रकल्पा जलूकाः ॥ १३॥\\
\\
मातर्मेमातरिश्वन् पितरतुलगुरो भ्रातरिष्टाप्तबन्धो\\
     स्वामिन्सर्वान्तरात्मन्नजरजरयितः जन्ममृत्यामयानाम् ।\\
गोविन्दे देहिभक्तिं भवतिच भगवन्नूर्जितां निर्निमित्ताम्\\
     निर्व्याजां निश्चलां सद्गुणगण बृहतीं शाश्वतीमाशुदेव ॥ १४॥\\
\\
विष्णोरत्त्युत्तमत्वादखिलगुणगणैस्तत्र भक्तिङ्गरिष्ठाम्\\
     संश्लिष्टे श्रीधराभ्याममुमथ परिवारात्मना सेवकेषु ।\\
यः सन्धत्ते विरिञ्चि श्वसन विहगपानन्त रुद्रेन्द्र पूर्वे\\
     ष्वाध्यायंस्तारतम्यं स्फुटमवति सदा वायुरस्मद्गुरुस्तम् ॥ १५॥\\
\\
तत्त्वज्ञान् मुक्तिभाजः सुखयिसि हि गुरो योग्यतातारतम्यात्\\
     आधत्से मिश्रबुद्धिं स्त्रिदिवनिरयभूगोचरान्नित्यबद्धान् ।\\
तामिस्रान्धादिकाख्ये  तमसिसुबहुलं दुःखयस्यन्यथाज्ञान्\\
     विष्णोराज्ञाभिरित्थं श‍ृति शतमितिहासादि चाकर्णयामः ॥ १६॥\\
\\
वन्देऽहं तं हनूमानिति महितमहापौरुषो बाहुशालि\\
     ख्यातस्तेऽग्र्योऽवतारः सहित इह बहुब्रह्मचर्यादि धर्मैः ।\\
सस्नेहानां सहस्वानहरहरहितं निर्दहन् देहभाजाम्\\
     अंहोमोहापहो यः स्पृहयति महतीं भक्तिमद्यापि रामे ॥ १७॥\\
\\
प्राक्पञ्चाशत्सहस्रैर्व्यवहितमहितं योजनैः पर्वतं त्वम्\\
     यावत्सञ्जीवनाद्यौषध निधिमधिकप्राणलङ्कामनैषिः ।\\
अद्राक्षीदुत्पतन्तं तत उत गिरिमुत्पाटयन्तं गृहीत्वा\\
     यान्तं खे राघवाङ्घ्रौ प्रणतमपि तदैकक्षणे त्वांहिलोकः ॥ १८॥\\
\\
क्षिप्तः पश्चात्सत्सलीलं शतमतुलमते योजनानां स\\
     उच्चस्तावद्विस्तार वंश्च्यापि उपललवैव व्यग्रबुद्ध्या त्वयातः ।\\
स्वस्वस्थानस्थिताति स्थिरशकल शिलाजाल संश्लेष नष्ट\\
     छेदाङ्कः प्रागिवाभूत् कपिवरवपुषस्ते नमः कौशलाय ॥ १९॥\\
\\
दृष्ट्वा दृष्टाधिपोरः स्फुटितकनक सद्वर्म घृष्टास्थिकूटम्\\
     निष्पिष्टं हाटकाद्रि प्रकट तट तटाकाति शङ्को जनोऽभूत् ।\\
येनाजौ रावणारिप्रियनटनपटुर्मुष्टिरिष्टं प्रदेष्टुम्\\
     किंनेष्टे मे स तेऽष्टापदकट कतटित्कोटि भामृष्ट काष्ठः ॥ २०॥\\
\\
देव्यादेश प्रणीति दृहिण हरवरावद्य रक्षो विघाता\\
     ऽद्यासेवोद्यद्दयार्द्रः सहभुजमकरोद्रामनामा मुकुन्दः ।\\
दुष्प्रापे पारमेष्ठ्ये करतलमतुलं मूर्धिविन्यस्य धन्यम्\\
     तन्वन्भूयः प्रभूत प्रणय विकसिताब्जेक्षणस्त्वेक्षमाणः ॥ २१॥\\
\\
जघ्नेनिघ्नेनविघ्नो  बहुलबलबकध्वंस नाद्येनशोचत्\\
     विप्रानुक्रोश पाशैरसु विधृति सुखस्यैकचक्राजनानाम् ।\\
तस्मैतेदेव कुर्मः कुरुकुलपतये कर्मणाचप्रणामान्\\
     किर्मीरं दुर्मतीनां प्रथमं अथ च यो नर्मणा निर्ममाथ ॥ २२॥\\
\\
निर्मृद्नन्नत्य यत्नं विजरवर जरासन्ध कायास्थिसन्धीन्\\
     युद्धे त्वं स्वध्वरे वापशुमिवदमयन् विष्णु पक्षद्विडीशम् ।\\
यावत्प्रत्यक्ष भूतं निखिलमखभुजं तर्पयामासिथासौ\\
     तावत्यायोजि तृप्त्याकिमुवद भघवन् राजसूयाश्वमेधे ॥ २३॥\\
\\
क्ष्वेलाक्षीणाट्टहासहं तवरणमरिहन्नुद्गदोद्दामबाहोः\\
     बह्वक्षौहिण्य नीकक्षपण सुनिपुणं यस्य सर्वोत्तमस्य ।\\
शुष्रूशार्थं चकर्थ स्वयमयमथ संवक्तुमानन्दतीर्थ\\
     श्रीमन्नामन्समर्थस्त्वमपि हि युवयोः पादपद्मं प्रपद्ये ॥ २४॥\\
\\
दृह्यन्तींहृदृहं मां दृतमनिल बलाद्रावयन्तीमविद्या\\
     निद्रांविद्राव्य सद्यो रचनपटुमथापाद्यविद्यासमुद्र ।\\
वाग्देवी सा सुविद्या द्रविणद विदिता द्रौपदी रुद्रपत्न्यात्\\
     उद्रिक्ताद्रागभद्रा द्रहयतु दयिता पूर्वभीमाज्ञयाते ॥ २५॥\\
\\
याभ्यां शुश्रूषुरासीः  कुरुकुल जनने क्षत्रविप्रोदिताभ्याम्\\
     ब्रह्मभ्यां बृंहिताभ्यां चितसुख वपुषा कृष्णनामास्पदाभ्याम् ।\\
निर्भेदाभ्यां विशेषाद्विवचन विशयाभ्यामुभाभ्याममूभ्याम्\\
     तुभ्यं च क्षेमदेभ्यः सरिसिजविलसल्लोचनेभ्यो नमोऽस्तु ॥ २६॥\\
\\
गच्छन् सौगन्धिकार्थं पथि स हनुमतः पुच्छमच्छस्य\\
     भीमः प्रोद्धर्तुं नाशकत्स त्वमुमुरुवपुषा भीषयामास चेति ।\\
पूर्णज्ञानौजसोस्ते गुरुतमवपुषोः श्रीमदानन्दतीर्थ\\
     क्रीडामात्रं तदेतत् प्रमदद सुधियां मोहक द्वेषभाजाम् ॥ २७॥\\
\\
बह्वीः कोटीरटीकः कुटलकटुमतीनुत्कटाटोप कोपान्\\
     द्राक्चत्वं सत्वरत्वाच्चरणद गदया पोथयामासिथारीन् ।\\
उन्मथ्या तत्थ्य मिथ्यात्व वचन वचनान् उत्पथस्थांस्तथाऽयान्\\
     प्रायच्छः स्वप्रियायै प्रियतम कुसुमं प्राण तस्मै नमस्ते ॥ २८॥\\
\\
देहादुत्क्रामितानामधिपति रसतामक्रमाद्वक्रबुद्धिः\\
     क्रुद्धः क्रोधैकवश्यः क्रिमिरिव मणिमान् दुष्कृती निष्क्रियार्थम् ।\\
चक्रे भूचक्रमेत्य क्रकचमिव सतां चेतसः कष्टशास्त्रं\\
     दुस्तर्कं चक्रपाणेर्गुणगण विरहं जीवतां चाधिकृत्य ॥ २९॥\\
\\
तद्दुत्प्रेक्षानुसारात्कतिपय कुनरैरादृतोऽन्यैर्विसृष्टो\\
     ब्रह्माहं निर्गुणोऽहं वितथमिदमिति ह्येषपाशण्डवादः ।\\
तद्युक्त्याभास जाल प्रसर विषतरूद्दाहदक्षप्रमाण\\
     ज्वालामालाधरोऽग्निः पवन विजयते तेऽवतारस्तृतीयः ॥ ३०॥\\
\\
आक्रोशन्तोनिराशा भयभर विवशस्वाशयाच्छिन्नदर्पा\\
     वाशन्तो देशनाशस्विति बत कुधियां नाशमाशादशाऽशु ।\\
धावन्तोऽश्लीलशीला वितथ शपथ शापा शिवाः शान्त शौर्याः\\
     त्वद्व्याख्या सिंहनादे सपदि ददृशिरे मायि गोमायवस्ते ॥ ३१॥\\
\\
त्रिष्वप्येवावतारेष्वरिभिरपघृणं हिंसितोनिर्विकारः\\
     सर्वज्ञः सर्वशक्तिः सकलगुणगणापूर्ण रूपप्रगल्भः ।\\
स्वच्छः स्वच्छन्द मृत्युः सुखयसि सुजनं देवकिं चित्रमत्र\\
     त्राता यस्य त्रिधामा जगदुतवशगं किङ्कराः शङ्कराद्याः ॥ ३२॥\\
\\
उद्यन्मन्दस्मित श्रीर्मृदु मधुमधुरालाप पीयूषधारा\\
     पूरासेकोपशान्ता सुखसुजन मनोलोचना पीयमानं ।\\
सन्द्रक्ष्येसुन्दरं सन्दुहदिह महदानन्दं आनन्दतीर्थ\\
     श्रीमद्वक्तेन्द्रु बिम्बं दुरतनुदुदितं नित्यदाहं कदानु ॥ ३३॥\\
\\
प्राचीनाचीर्ण पुण्योच्चय चतुरतराचारतश्चारुचित्तान्\\
     अत्युच्चां रोचयन्तीं श‍ृतिचित वचनांश्राव कांश्चोद्यचुञ्चून् ।\\
व्याख्यामुत्खात दुःखां चिरमुचित महाचार्य चिन्तारतांस्ते\\
     चित्रां सच्छास्त्रकर्ताश्चरण परिचरां छ्रावयास्मांश्चकिञ्चित् ॥ ३४॥\\
\\
पीठेरत्नोकपक्लृप्ते रुचिररुचिमणि ज्योतिषा सन्निषण्णम्\\
     ब्रह्माणं भाविनं त्वां ज्वलति निजपदे वैदिकाद्या हि विद्याः ।\\
सेवन्ते मूर्तिमत्यः सुचरितचरितं भाति गन्धर्व गीतं\\
     प्रत्येकं देवसंसत्स्वपि तव भघवन्नर्तितद्द्योवधूषु ॥ ३५॥\\
\\
सानुक्रोषैरजस्रं जनिमृति निरयाद्यूर्मिमालाविलेऽस्मिन्\\
     संसाराब्धौनिमग्नांशरणमशरणानिच्छतो वीक्ष्यजन्तून् ।\\
युष्माभिः प्र्राथितः सन् जलनिधिशयनः सत्यवत्यां महर्षेः\\
     व्यक्तश्चिन्मात्र मूर्तिनखलु भगवतः प्राकृतो जातु देहः ॥ ३६॥\\
\\
अस्तव्यस्तं समस्तश‍ृति गतमधमैः रत्नपूगं यथान्धैः\\
     अर्थं लोकोपकृत्यैः गुणगणनिलयः सूत्रयामास कृत्स्नम् ।\\
योऽसौ व्यासाभिधानस्तमहमहरहः भक्तितस्त्वत्प्रसादात्\\
     सद्यो विद्योपलब्ध्यै गुरुतममगुरुं देवदेवं नमामि ॥ ३७॥\\
\\
आज्ञामन्यैरधार्यां शिरसि परिसरद्रश्मि कोटीरकोटौ\\
     कृष्णस्याक्लिष्ट कर्मादधदनु सराणादर्थितो देवसङ्घैः ।\\
भूमावागत्य भूमन्नसुकरमकरोर्ब्रह्मसूत्रस्य भाष्यम्\\
     दुर्भाष्यं व्यास्यदस्योर्मणिमत उदितं वेदसद्युक्तिभिस्त्वम् ॥ ३८॥\\
\\
भूत्वाक्षेत्रे विशुद्धे द्विजगणनिलये रौप्यपीठाभिधाने\\
     तत्रापि ब्रह्मजातिस्त्रिभुवन  विशदे मध्यगेहाख्य गेहे ।\\
पारिव्राज्याधि राजः पुनरपि बदरीं प्राप्य कृष्णं च नत्वा\\
     कृत्वा भाष्याणि सम्यक् व्यतनुत च भवान् भरतार्थप्रकाशम् ॥ ३९॥\\
\\
वन्दे तं त्वां सुपूर्ण प्रमतिमनुदिना सेवितं देववृन्दैः\\
     वन्दे वन्दारुमीशे श्रिय उत नियतं श्रीमदानन्दतीर्थम् ।\\
वन्दे मन्दाकिनी सत्सरिदमल जलासेक साधिक्य सङ्गम्\\
     वन्देऽहं देव भक्त्या भव भय दहनं सज्जनान्मोदयन्तम् ॥ ४०॥\\
\\
सुब्रह्मण्याख्य सूरेः सुत इति सुभृशं केशवानन्दतीर्थ\\
     श्रीमत्पादाब्ज भक्तः स्तुतिमकृत हरेर्वायुदेवस्य चास्य ।\\
त्वत्पादार्चादरेण ग्रथित पदल सन्मालया त्वेतयाये\\
     संराध्यामूनमन्ति प्रततमतिगुणा मुक्तिमेते व्रजन्ति ॥ ४१॥\\
\\
          इति श्रीत्रिविक्रमपण्डिताचार्य विरचितं\\
          श्रीहरिवायुस्तुतिः सम्पूर्णम् ।\\
 ॥ अथ श्री नखस्तुतिः ॥\\
\\
पान्त्वस्मान् पुरुहूतवैरि बलवन्मातङ्ग माद्यद्घटा\\
     कुम्भोच्चाद्रि विपाटनाधिकपटु प्रत्येक वज्रायिताः ।\\
श्रीमत्कण्ठीरवास्य प्रतत सुनखरा दारितारातिदूर\\
     प्रद्ध्वस्तध्वान्त शान्त प्रवितत मनसा भावितानाकिवृन्दैः ॥ १॥\\
\\
लक्ष्मीकान्त समन्ततोऽपिकलयन् नैवेशितुस्ते समम्\\
     पश्याम्युत्तम वस्तु दूरतरतोपास्तं रसोयोऽष्टमः ।\\
यद्रोशोत्कर दक्ष नेत्र कुटिलः प्रान्तोत्थिताग्नि स्फुरत्\\
     खद्योतोपम विस्फुलिङ्गभसिता ब्रह्मेशशक्रोत्कराः ॥ २॥\\
\\
इति श्रीमदानन्दतीर्थभगवत्पादाचार्यविरचितं \\
श्रीनृसिंहनखस्तुतिः सम्पुर्णम् । \\
\section{\color{blue}\sanskrit भजगोविन्दं}
\sanskrit
भजगोविन्दं भजगोविन्दं गोविन्दं भज मूढमते ।\\
संप्राप्ते सन्निहिते काले नहि नहि रक्षति डुकृञ्करणे ॥ १ ॥\\
\\
मूढ जहीहि धनागमतृष्णां कुरु सद्बुद्धिं मनसि वितृष्णाम् ।\\
यल्लभसे निजकर्मोपात्तं वित्तं तेन विनोदय चित्तम् ॥ २ ॥\\
\\
नारीस्तनभर नाभीदेशं दृष्ट्वा मागामोहावेशम् ।\\
एतन्मांसावसादि विकारं मनसि विचिन्तय वारं वारम् ॥ ३ ॥\\
\\
नलिनीदलगत जलमतितरलं तद्वज्जीवितमतिशयचपलं ।\\
विद्धि व्याध्यभिमानग्रस्तं लोकं शोकहतं च समस्तम् ॥ ४ ॥\\
\\
यावद्वित्तोपार्जन सक्तः तावन्निज परिवारो रक्तः ।\\
पश्चाज्जीवति जर्जर देहे वार्तां कोऽपि न पृच्छति गेहे ॥ ५ ॥\\
\\
यावत्पवनो निवसति देहे तावत्पृच्छति कुशलं गेहे ।\\
गतवति वायौ देहापाये भार्या बिभ्यति तस्मिन्काये ॥ ६ ॥\\
\\
बालस्तावत्क्रीडासक्तः तरुणस्तावत्तरुणीसक्तः ।\\
वृद्धस्तावत्चिन्तासक्तः परे ब्रह्मणि कोऽपि न सक्तः ॥ ७ ॥\\
\\
काते कान्ता कस्ते पुत्रः संसारोऽयमतीव विचित्रः ।\\
कस्य त्वं कः कुत आयातः तत्त्वं चिन्तय तदिह भ्रातः ॥ ८ ॥\\
\\
सत्सङ्गत्वे निस्सङ्गत्वं निस्सङ्गत्वे निर्मोहत्वम् ।\\
निर्मोहत्वे निश्चलतत्त्वं निश्चलतत्त्वे जीवन्मुक्तिः ॥ ९ ॥\\
\\
वयसिगते कः कामविकारः शुष्के नीरे कः कासारः ।\\
क्षीणेवित्ते कः परिवारः ज्ञाते तत्त्वे कः संसारः ॥ १० ॥\\
\\
मा कुरु धन जन यौवन गर्वं हरति निमेषात्कालः सर्वम् ।\\
मायामयमिदमखिलं हित्वा ब्रह्मपदं त्वं प्रविश विदित्वा ॥ ११ ॥\\
\\
दिनयामिन्यौ सायं प्रातः शिशिरवसन्तौ पुनरायातः ।\\
कालः क्रीडति गच्छत्यायुः तदपि न मुञ्चत्याशावायुः ॥ १२ ॥\\
\\
द्वादशमञ्जरिकाभिरशेषः कथितो वैयाकरणस्यैषः ।\\
उपदेशोऽभूद्विद्यानिपुणैः श्रीमच्छङ्करभगवच्छरणैः ॥ १२अ ॥\\
\\
काते कान्ता धन गतचिन्ता वातुल किं तव नास्ति नियन्ता ।\\
त्रिजगति सज्जन सङ्गतिरेका भवति भवार्णवतरणे नौका ॥ १३ ॥\\
\\
जटिलो मुण्डी लुञ्छितकेशः काषायाम्बरबहुकृतवेषः ।\\
पश्यन्नपि चन पश्यति मूढः उदरनिमित्तं बहुकृतवेषः ॥ १४ ॥\\
\\
अङ्गं गलितं पलितं मुण्डं दशनविहीनं जतं तुण्डम् ।\\
वृद्धो याति गृहीत्वा दण्डं तदपि न मुञ्चत्याशापिण्डम् ॥ १५ ॥\\
\\
अग्रे वह्निः पृष्ठेभानुः रात्रौ चुबुकसमर्पितजानुः ।\\
करतलभिक्षस्तरुतलवासः तदपि न मुञ्चत्याशापाशः ॥ १६ ॥\\
\\
कुरुते गङ्गासागरगमनं व्रतपरिपालनमथवा दानम् ।\\
ज्ञानविहिनः सर्वमतेन मुक्तिं न भजति जन्मशतेन ॥ १७ ॥\\
\\
सुर मन्दिर तरु मूल निवासः शय्या भूतलमजिनं वासः ।\\
सर्व परिग्रह भोग त्यागः कस्य सुखं न करोति विरागः ॥ १८ ॥\\
\\
योगरतो वा भोगरतोवा सङ्गरतो वा सङ्गविहीनः ।\\
यस्य ब्रह्मणि रमते चित्तं नन्दति नन्दति नन्दत्येव ॥ १९ ॥\\
\\
भगवद् गीता किञ्चिदधीता गङ्गा जललव कणिकापीता ।\\
सकृदपि येन मुरारि समर्चा क्रियते तस्य यमेन न चर्चा ॥ २० ॥\\
\\
पुनरपि जननं पुनरपि मरणं पुनरपि जननी जठरे शयनम् ।\\
इह संसारे बहुदुस्तारे कृपयाऽपारे पाहि मुरारे ॥ २१ ॥\\
\\
रथ्या चर्पट विरचित कन्थः पुण्यापुण्य विवर्जित पन्थः ।\\
योगी योगनियोजित चित्तो रमते बालोन्मत्तवदेव ॥ २२ ॥\\
\\
कस्त्वं कोऽहं कुत आयातः का मे जननी को मे तातः ।\\
इति परिभावय सर्वमसारं विश्वं त्यक्त्वा स्वप्न विचारम् ॥ २३ ॥\\
\\
त्वयि मयि चान्यत्रैको विष्णुः व्यर्थं कुप्यसि मय्यसहिष्णुः ।\\
भव समचित्तः सर्वत्र त्वं वाञ्छस्यचिराद्यदि विष्णुत्वम् ॥ २४ ॥\\
\\
शत्रौ मित्रे पुत्रे बन्धौ मा कुरु यत्नं विग्रहसन्धौ ।\\
सर्वस्मिन्नपि पश्यात्मानं सर्वत्रोत्सृज भेदाज्ञानम् ॥ २५ ॥\\
\\
कामं क्रोधं लोभं मोहं त्यक्त्वाऽत्मानं भावय कोऽहम् ।\\
आत्मज्ञान विहीना मूढाः ते पच्यन्ते नरकनिगूढाः ॥ २६ ॥\\
\\
गेयं गीता नाम सहस्रं ध्येयं श्रीपति रूपमजस्रम् ।\\
नेयं सज्जन सङ्गे चित्तं देयं दीनजनाय च वित्तम् ॥ २७ ॥\\
\\
सुखतः क्रियते रामाभोगः पश्चाद्धन्त शरीरे रोगः ।\\
यद्यपि लोके मरणं शरणं तदपि न मुञ्चति पापाचरणम् ॥ २८ ॥\\
\\
अर्थमनर्थं भावय नित्यं नास्तिततः सुखलेशः सत्यम् ।\\
पुत्रादपि धन भाजां भीतिः सर्वत्रैषा विहिता रीतिः ॥ २९ ॥\\
\\
प्राणायामं प्रत्याहारं नित्यानित्य विवेकविचारम् ।\\
जाप्यसमेत समाधिविधानं कुर्ववधानं महदवधानम् ॥ ३० ॥\\
\\
गुरुचरणाम्बुज निर्भर भक्तः संसारादचिराद्भव मुक्तः ।\\
सेन्द्रियमानस नियमादेवं द्रक्ष्यसि निज हृदयस्थं देवम् ॥ ३१ ॥\\
\\
मूढः कश्चन वैयाकरणो डुकृञ्करणाध्ययन धुरिणः ।\\
श्रीमच्छङ्कर भगवच्छिष्यै बोधित आसिच्छोधितकरणः ॥ ३२ ॥\\
\\
भजगोविन्दं भजगोविन्दं गोविन्दं भजमूढमते ।\\
नामस्मरणादन्यमुपायं नहि पश्यामो भवतरणे ॥ ३३ ॥\\