\section{\color{blue}\sanskrit नारायणकवचं सार्थ}
\sanskrit
ॐ हरिर्विदध्यान् मम सर्व-रक्षां न्यस्ताङ्घ्रि-पद्मः पतगेन्द्र-पृष्ठे\\
दरारि-चर्मासि-गदेषु-चाप-पाशान् दधानोऽष्ट-गुणोऽष्ट-बाहुः ॥ १२॥\\
\\
ॐ - O Lord;  हरिः - the Supreme Personality of Godhead;  विदध्यात् - may He bestow;  मम  my;  सर्व-रक्षाम् - protection from all sides;  न्यस्त - placed;  अङ्घ्रि-पद्मः - whose lotus feet; पतगेन्द्र-पृष्ठे - on the back of GaruDa, the king of all birds; दर - conchshell;  अरि -   disc;  चर्म - shield;  असि - sword;  गदा - club;  इषु - arrows;  चाप - bow;  पाशान् - ropes;  दधानः - holding;  अष्ट - possessing eight;  गुणः - perfections;  अष्ट - eight;  बाहुः - arms.\\
\\
The Supreme Lord, who sits on the back of the bird GaruDa, touching him with His lotus feet, holds eight weapons : the conchshell, disc, shield, sword, club, arrows, bow and ropes. May that Supreme Personality of Godhead protect me at all times with His eight arms. He is all-powerful because He fully possesses the eight mystic powers [aNimA, laghimA, etc.].\\
\\
PURPORT - Thinking oneself one with the Supreme is called aha~NgrahopAsanA. Through aha~NgrahopAsanA one does not become God, but he thinks of himself as qualitatively one with the Supreme. Understanding that as a spirit soul he is equal in quality to the supreme soul the way the water of a river is of the same nature as the water of the sea, one should meditate upon the Supreme Lord, as described in this verse, and seek His protection. The living entities are always subordinate to the Supreme\footnote{To be considered as Godhead {\sanskrit  श्री महाविष्णु}}. Consequently their duty is to always seek the mercy of the Lord in order to be protected by Him in all circumstances.\\
\\
जलेषु मां रक्षतु मत्स्य-मूर्तिर्यादो-गणेभ्यो वरुणस्य पाशात्\\
स्थलेषु मायावटु-वामनोऽव्यात्त्रिविक्रमः खेऽवतु विश्वरूपः ॥ १३ ॥\\
\\
जलेषु - in the water;  माम् - me;  रक्षतु - protect;  मत्स्य-मूर्तिः  the Supreme Lord in the form of a great fish;  यादः-गणेभ्यः - from fierce aquatic animals;  वरुणस्य - of the demigod known as VaruNa;  पाशात् - from the arresting rope; स्थलेषु - on the land;  माया-वटु - the merciful form of the Lord as a dwarf;  वामनः - named VAmanadeva;  अव्यात् – may He protect;  त्रिविक्रमः - Trivikrama, whose three gigantic steps took the three worlds from Bali;  खे - in the sky;  अवतु - may the Lord protect;  विश्वरूपः - the gigantic universal form.\\
\\
May the Lord, who assumes the body of a great fish, protect me in the water from the fierce animals that are associates of the demigod VaruNa. By expanding His illusory energy, the Lord assumed the form of the dwarf VAmana. May VAmana protect me on the land. Since the gigantic form of the Lord, VishvarUpa, conquers the three worlds, may He protect me in the sky.\\
\\
PURPORT - This mantra seeks the protection of the Supreme Personality of Godhead in the water, land and sky in His incarnations as the fish, VAmanadeva and the VishvarUpa.\\
\\
दुर्गेष्वटव्याजि-मुखादिषु प्रभुः पायान् नृसिंहोऽसुर-यूथपारिः\\
विमुञ्चतो यस्य महाट्ट-हासं दिशो विनेदुर्यपतंश्च गर्भाः ॥ १४॥\\
\\
दुर्गेषु - in places where travel is very difficult; अटवि - in the dense forest;  आजि-मुख-आदिषु - on the war front and so on;  प्रभुः - the Supreme Lord;  पायात् - may He protect;  नृसिंहः  Lord NRisiMhadeva;  असुर-यूथप - of HiraNyakashipu, the leader of the demons;  अरिः - the enemy; विमुञ्चतः - releasing;  यस्य - of whom;  महा-अट्ट-हासम् - great and fearful laughing;  दिशः - all the directions;  विनेदुः - resounded through;  न्यपतन् - fell down;  च - and;  गर्भाः - the embryos of the wives of the demons.\\
\\
May Lord NRisiMhadeva, who appeared as the enemy of HiraNyakashipu, protect me in all directions. His loud laughing vibrated in all directions and caused the pregnant wives of the asuras to have miscarriages. May that Lord be kind enough to protect me in difficult places like the forest and battlefront.\\
\\ 
रक्षत्वसौ माध्वनि यज्ञ-कल्पः स्व-दंष्ट्रयोन्नीत-धरो वराहः\\
रामोऽद्रि-कूटेष्वथ विप्रवासे सलक्ष्मणोऽव्याद् भरताग्रजोऽस्मान् ॥ १५॥\\
\\
रक्षतु - may the Lord protect;  असौ - that;  मा - me; अध्वनि - on the street;  यज्ञ-कल्पः - who is ascertained by performance of ritualistic ceremonies;  स्व-दंष्ट्रया – by His own tusk;  उन्नीत - raising;  धरः - the planet earth; वराहः - Lord Boar;  रामः - Lord RAma;  अद्रि-कूटेषु - on the summits of the mountains;  अथ - then;  विप्रवासे – in foreign countries;  स-लक्ष्मणः -  with His brother LakShmaNa; अव्यात् - may He protect;  भरत-अग्रजः - the elder brother of MahArAja Bharata;  अस्मान् - us.\\
\\
The Supreme indestructible Lord is ascertained through the performance of ritualistic sacrifices and is therefore known as Yaj~neshvara. In His incarnation as Lord Boar, He raised the planet earth from the water at the bottom of the universe and kept it on His pointed tusks. May that Lord protect me from rogues on the street. May ParashurAma protect me on the tops of mountains, and may the elder brother of Bharata, Lord RAmacandra, along with His brother LakShmaNa, protect me in foreign countries.\\
\\
PURPORT - There are three RAmas. One RAma is ParashurAma (JAmadAgnya). Another RAma is Lord RAmacandra, and a third RAma is Lord BalarAma. In this verse the words rAmo.adri-kUTeShvatha indicate Lord ParashurAma. The brother of Bharata MahArAja and LakShmaNa is Lord RAmacandra.\\
\\ 
माम् उग्र-धर्माद् अखिलात् प्रमादान्नारायणः पातु नरश्च हासात्\\
दत्तस्त्वयोगाद् अथ योग-नाथः पायाद् गुणेशः कपिलः कर्म-बन्धात् ॥ १६॥\\
\\
माम् - me;  उग्र-धर्मात् - from unnecessary religious principles;  अखिलात् - from all kinds of activities; प्रमादात् - which are enacted in madness;  नारायणः – Lord NArAyaNa;  पातु - may He protect;  नरः च - and Nara; हासात् - from unnecessary pride;  दत्तः - DattAtreya;  तु - of course;  अयोगात् - from the path of false yoga;  अथ  indeed;  योग-नाथः - the master of all mystic powers;  पायात् – may He protect;  गुण-ईशः - the master of all spiritual qualities;  कपिलः - Lord Kapila;  कर्म-बन्धात् - from the bondage of fruitive activities.\\
\\
May Lord NArAyaNa protect me from unnecessarily following false religious systems and falling from my duties due to madness. May the Lord in His appearance as Nara protect me from unnecessary pride. May Lord DattAtreya, the master of all mystic power, protect me from falling while performing bhakti-yoga, and may Lord Kapila, the master of all good qualities, protect me from the material bondage of fruitive activities.\\
\\
सनत्-कुमारोऽवतु कामदेवाद्धयशीर्षा मां पथि देव-हेलनात्\\
देवर्षि-वर्यः पुरुषार्चनान्तरात्कूर्मो हरिर्मां निरयाद् अशेषात् ॥ १७॥\\
\\
सनत्-कुमारः - the great brahmacArI named Sanat-kumAra;  अवतु  may he protect;  काम-देवात् - from the hands of Cupid or lusty desire; हय-शीर्षा - Lord HayagrIva, the incarnation of the Lord whose head is like that of a horse;  माम् - me;  पथि - on the path;  देव-हेलनात् - from neglecting to offer respectful obeisances to brAhmaNas, VaiShNavas and the Supreme Lord;  देवर्षि-वर्यः - the best of the saintly sages, NArada;  पुरुष-अर्चन-अन्तरात् - from the offenses in worshiping the Deity;  कूर्मः - Lord KUrma, the tortoise;  हरिः - the Supreme Personality of Godhead;  माम् -  me;  निरयात् - from hell;  अशेषात् - unlimited.\\
\\
May Sanat-kumAra protect me from lusty desires. As I begin some auspicious activity, may Lord HayagrIva protect me from being an offender by neglecting to offer respectful obeisances to the Supreme Lord. May DevarShi NArada protect me from committing offenses in worshiping the Deity, and may Lord KUrma, the tortoise, protect me from falling to the unlimited hellish planets.\\
\\
PURPORT - Lusty desires are very strong in everyone, and they are the greatest impediment to the discharge of devotional service. Therefore those who are very much influenced by lusty desires are advised to take shelter of Sanat-kumAra, the great brahmacArI devotee. NArada Muni, who is the guide for arcana, is the author of the NArada-pa~ncharAtra, which prescribes the regulative principles for worshiping the Deity. Everyone engaged in Deity worship, whether at home or in the temple, should always seek the mercy of DevarShi NArada in order to avoid the thirty-two offenses while worshiping the Deity. These offenses in Deity worship are mentioned in The Nectar of Devotion.\\
\\
धन्वन्तरिर्भगवान् पात्वपथ्याद्द्वन्द्वाद् भयाद् ऋषभो निर्जितात्मा\\
यज्ञश्च लोकाद् अवताज् जनान्ताद्बलो गणात् क्रोध-वशाद् अहीन्द्रः ॥ १८॥\\
\\
धन्वन्तरिः - the incarnation Dhanvantari, the physician;  भगवान् - the Supreme Personality of Godhead;  पातु - may He protect me;  अपथ्यात् - from things injurious to the health, such as meat and intoxicants; द्वन्द्वात् - from duality;  भयात् - from fear;  ऋषभः – Lord RiShabhadeva;  निर्जित-आत्मा - who fully controlled his mind and self;  यज्ञः - Yaj~na;  च - and;  लोकात् - from the defamation of the populace;  अवतात् - may He protect;  जन-अन्तात्  from dangerous positions created by other people;  बलः - Lord BalarAma; गणात् - from the hordes of;  क्रोध-वशात् - the angry serpents; अहीन्द्रः - Lord BalarAma in the form of the serpent sheSha NAga.\\
\\
May the Supreme Personality of Godhead in His incarnation as Dhanvantari relieve me from undesirable eatables and protect me from physical illness. May Lord Shabhadeva, who conquered His inner and outer senses, protect me from fear produced by the duality of heat and cold. May Yaj~na protect me from defamation and harm from the populace, and may Lord BalarAma as sheSha protect me from envious serpents.\\
\\
PURPORT - To live within this material world, one must face many dangers, as described herein. For example, undesirable food poses a danger to health, and therefore one must give up such food. The Dhanvantari incarnation can protect us in this regard. Since Lord ViShNu is the Supersoul of all living entities, if He likes He can save us from adhibhautika disturbances, disturbances from other living entities. Lord BalarAma is the sheSha incarnation, and therefore He can save us from angry serpents or envious persons, who are always ready to attack.\\
\\
द्वैपायनो भगवान् अप्रबोधाद्बुद्धस् तु पाषण्ड-गण-प्रमादात्\\
कल्किः कलेः काल-मलात् प्रपातुधर्मावनायोरु-कृतावतारः ॥ १९॥\\
\\
द्वैपायनः - shrIla VyAsadeva, the giver of all Vedic knowledge;  भगवान् - the most powerful incarnation of the Supreme Personality of Godhead;  अप्रबोधात् - from ignorance of the shAstra;  बुद्धः तु - also Lord Buddha;  पाषण्ड-गण - of atheists creating disillusionment for innocent persons;  प्रमादात् - from the madness;  कल्किः - Lord Kalki, the incarnation of Keshava;  कलेः – of this Kali-yuga;  काल-मलात् - from the darkness of the age; प्रपातु - may He protect;  धर्म-अवनाय - for the protection of religious principles;  उरु - very great;  कृत-अवतारः - who took an incarnation.\\
\\
May the Personality of Godhead in His incarnation as VyAsadeva protect me from all kinds of ignorance resulting from the absence of Vedic knowledge. May Lord Buddhadeva protect me from activities opposed to Vedic principles and from laziness that causes one to madly forget the Vedic principles of knowledge and ritualistic action. May Kalkideva, the Supreme Personality of Godhead, who appeared as an incarnation to protect religious principles, protect me from the dirt of the age of Kali.\\
\\
PURPORT - This verse mentions various incarnations of the Supreme Personality of Godhead who appear for various purposes. shrIla VyAsadeva, MahAmuni, compiled the Vedic literature for the benefit of all human society. If one wants to be protected from the reactions of ignorance even in this age of Kali, one may consult the books left by shrIla VyAsadeva, UpaniShads, VedAnta-sUtra (Brahma-sUtra), MahAbhArata, shrImad-BhAgavatam MahA-PurANa (VyAsadeva.as commentary on the Brahma-sUtra) and the other seventeen PurANas. Only by the mercy of shrIla VyAsadeva do we have so many volumes of transcendental knowledge to save us from the clutches of ignorance.\\
As described by shrIla Jayadeva GosvAmI in his DashAvatAra-stotra, Lord Buddha apparently decried the Vedic knowledge:  \\
\\
निन्दसि यज्ञ-विधेर् अहह श्रुति-जातं\\
सदय-हृदय-दर्शित-पशु-घातम्\\
केशव धृत-बुद्ध-शरीर जय जगद्-ईश हरे\\
\\
The mission of Lord Buddha was to save people from the abominable activity of animal killing and to save the poor animals from being unnecessarily killed. When pAShaNDIs were cheating by killing animals on the plea of sacrificing them in Vedic yaj~nas, the Lord said, ᳚If the Vedic injunctions allow animal killing, I do not accept the Vedic principles.᳚ Thus he actually saved people who acted according to Vedic principles. One should therefore surrender to Lord Buddha so that he can help one avoid misusing the injunctions of the Vedas.
The Kalki avatAra is the fierce incarnation who vanquishes the class of the atheists born in this age of Kali. Now, in the beginning of Kali-yuga, many irreligious principles are in effect, and as Kali-yuga advances, many pseudo religious principles will certainly be introduced, and people will forget the real religious principles enunciated by Lord KRiShNa before the beginning of Kali-yuga, namely principles of surrender unto the lotus feet of the Lord. Unfortunately, because of Kali-yuga, foolish people do not surrender to the lotus feet of KRiShNa. Even most people who claim to belong to the Vedic system of religion are actually opposed to the Vedic principles. Every day they manufacture a new type of dharma on the plea that whatever one manufactures is also a path of liberation. Atheistic men generally say, yata mata tata patha. According to this view, there are hundreds and thousands of different opinions in human society, and each opinion is a valid religious principle. This philosophy of rascals has killed the religious principles mentioned in the Vedas, and such philosophies will become increasingly influential as Kali-yuga progresses. In the last stage of Kali-yuga, Kalkideva, the fierce incarnation of Keshava, will descend to kill all the atheists and will save only the devotees of the Lord.\\
\\ 
मां केशवो गदया प्रातर् अव्याद्गोविन्द आसङ्गवम् आत्त-वेणुः\\
नारायणः प्राह्ण उदात्त-शक्तिर्मध्यन्-दिने विष्णुररीन्द्र-पाणिः ॥ २०॥\\
\\
माम् - me;  केशवः - Lord Keshava;  गदया - by His club;  प्रातः - in the morning hours;  अव्यात् - may He protect;  गोविन्दः - Lord Govinda; आसङ्गवम् - during the second part of the day;  आत्त-वेणुः - holding His flute;  नारायणः - Lord NArAyaNa with four hands;  प्राह्णः - during the third part of the day;  उदात्त-शक्तिः -  controlling different types of potencies;  मध्यम्-दिने - during the fourth part of the day;  विष्णुः - Lord ViShNu; अरीन्द्र-पाणिः - bearing the disc in His hand to kill the enemies.\\
\\
May Lord Keshava protect me with His club in the first portion of the day, and may Govinda, who is always engaged in playing His flute, protect me in the second portion of the day. May Lord NArAyaNa, who is equipped with all potencies, protect me in the third part of the day, and may Lord ViShNu, who carries a disc to kill His enemies, protect me in the fourth part of the day.\\
\\
PURPORT - According to Vedic astronomical calculations, day and night are each divided into thirty ghaTikAs (twenty-four minutes), instead of twelve hours. Generally, each day and each night is divided into six parts consisting of five ghaTikAs. In each of these six portions of the day and night, the Lord may be addressed for protection according to different names. Lord Keshava, the proprietor of the holy place of MathurA, is the Lord of the first portion of the day, and Govinda, the Lord of VRindAvana. is the master of the second portion.\\
\\ 
देवोऽपराह्णे मधु-होग्रधन्वा सायं त्रि-धामावतु माधवो माम्\\
दोषे हृषीकेश उतार्ध-रात्रे निशीथ एकोऽवतु पद्मनाभः ॥ २१॥\\
\\
देवः - the Lord;  अपराह्णे - in the fifth part of the day;  मधु-हा - named MadhusUdana;  उग्र-धन्वा - bearing the very fearful bow known as shAr~Nga;  सायम् - the sixth part of the day;  त्रि-धामा - manifesting as the three deities BrahmA, ViShNu and Maheshvara;  अवतु - may He protect;  माधवः - named MAdhava;  माम् - me;  दोषे - during the first portion of the night;  हृषीकेशः - Lord HRiShIkesha;  उत - also; अर्ध-रात्रे - during the second part of the night;  निशीथे - during the third part of the night;  एकः - alone;  अवतु -  may He protect;  पद्मनाभः - Lord PadmanAbha.\\
\\
May Lord MadhusUdana, who carries a bow very fearful for the demons, protect me during the fifth part of the day. In the evening, may Lord MAdhava, appearing as BrahmA, ViShNu and Maheshvara, protect me, and in the beginning of night may Lord HRiShIkesha protect me. At the dead of night [in the second and third parts of night] may Lord PadmanAbha alone protect me.\\
\\
श्रीवत्स-धामापर-रात्र ईशः प्रत्यूष ईशोऽसि-धरो जनार्दनः\\
दामोदरोऽव्यादनुसन्ध्यं प्रभाते विश्वेश्वरो भगवान् काल-मूर्तिः ॥ २२॥\\
\\
श्रीवत्स-धामा - the Lord, on whose chest the mark of shrIvatsa is resting;  अपर-रात्रे - in the fourth part of the night;  ईशः - the Supreme Lord;  प्रत्यूषे - in the end of the night;  ईशः - the Supreme Lord;  असि-धरः -  carrying a sword in the hand;  जनार्दनः - Lord JanArdana;  दामोदरः - Lord DAmodara;  अव्यात् - may He protect;  अनुसन्ध्यम् - during each junction or twilight;  प्रभाते - in the early morning (the sixth part of the night);  विश्व-ईश्वरः - the Lord of the whole universe; भगवान् - the Supreme Personality of Godhead;  काल-मूर्तिः - the personification of time.\\
\\
May the Supreme Personality of Godhead, who bears the shrIvatsa on His chest, protect me after midnight until the sky becomes pinkish. May Lord JanArdana, who carries a sword in His hand, protect me at the end of night [during the last four ghaTikAs of night]. May Lord DAmodara protect me in the early morning, and may Lord Vishveshvara protect me during the junctions of day and night.\\
\\
चक्रं युगान्तानल-तिग्म-नेमि भ्रमत् समन्ताद् भगवत्-प्रयुक्तम्\\
दन्दग्धि दन्दग्ध्य् अरि-सैन्यम् आशु कक्षं यथा वात-सखो हुताशः ॥ २३॥\\
\\
चक्रम् – the disc of the Lord;  युग-अन्त - at the end of the millennium;  अनल -  like the fire of devastation;  तिग्म-नेमि - with a sharp rim; भ्रमत् - wandering;  समन्तात् - on all sides;  भगवत्-प्रयुक्तम् - being engaged by the Lord;  दन्दग्धि दन्दग्धि - please burn completely, please burn completely;  अरि-सैन्यम् - the army of our enemies;  आशु - immediately;  कक्षम् - dry grass;  यथा - like;  वात-सखः - the friend of the wind;  हुताशः - blazing fire.\\
\\
Set into motion by the Supreme Personality of Godhead and wandering in all the four directions, the disc of the Supreme Lord has sharp edges as destructive as the fire of devastation at the end of the millennium. As a blazing fire burns dry grass to ashes with the assistance of the breeze, may that Sudarshana cakra burn our enemies to ashes.\\
\\ 
गदेऽशनि-स्पर्शन-विस्फुलिङ्गे निष्पिण्ढि निष्पिण्ढ्य् अजित-प्रियासि\\
कुष्माण्ड-वैनायक-यक्ष-रक्षो-भूत-ग्रहांश्चूर्णय चूर्णयारीन् ॥ २४॥\\
\\
गदे - O club in the hands of the Supreme Personality of Godhead;  अशनि - like thunderbolts; स्पर्शन -  whose touch;  विस्फुलिङ्गे - giving off sparks of fire;  निष्पिण्ढि निष्पिण्ढि  pound to pieces, pound to pieces;  अजित-प्रिया - very dear to the Supreme Personality of Godhead; असि - you are;  कुष्माण्ड - imps named KuShmANDas;  वैनायक -  ghosts named VainAyakas;  यक्ष - ghosts named YakShas; रक्षः - ghosts named RAkShasas;  भूत - ghosts named BhUtas;  ग्रहान् - and evil demons named Grahas;  चूर्णय - pulverize; चूर्णय -  pulverize;  अरीन् - my enemies.\\
\\
O club in the hand of the Supreme Personality of Godhead, you produce sparks of fire as powerful as thunderbolts, and you are extremely dear to the Lord. I am also His servant. Therefore kindly help me pound to pieces the evil living beings known as KuShmANDas, VainAyakas, YakShas, RAkShasas, BhUtas and Grahas. Please pulverize them.\\
\\
त्वं\\
यातुधान-प्रमथ-प्रेत-मातृ-पिशाच-विप्रग्रह-घोर-दृष्टीन्\\
दरेन्द्र विद्रावय कृष्ण-पूरितो भीम-स्वनोऽरेर्हृदयानि कम्पयन् ॥ २५॥\\
\\
त्वम् - you;  यातुधान - RAkShasas;  प्रमथ - Pramathas;  प्रेत - Pretas;  मातृ - MAtAs;  पिशाच - PishAcas;  विप्र-ग्रह -  brAhmaNa ghosts; घोर-दृष्टीन् - who have very fearful eyes;  दरेन्द्र – O PA~ncajanya, the conchshell in the hands of the Lord;  विद्रावय - drive away;  कृष्ण-पूरितः - being filled with air from the mouth of KRiShNa;  भीम-स्वनः - sounding extremely fearful;  अरेः - of the enemy;  हृदयानि - the cores of the hearts; कम्पयन् - causing to tremble.  O best of conchshells, O PA~ncajanya in the hands of the Lord, you are always filled with the breath of Lord KRiShNa. Therefore you create a fearful sound vibration that causes trembling in the hearts of enemies like the RAkShasas, pramatha ghosts, Pretas, MAtAs, PishAcas and brAhmaNa ghosts with fearful eyes.\\
\\ 
त्वं तिग्म-धारासि-वरारि-सैन्यम् ईश-प्रयुक्तो मम चिन्धि चिन्धि\\
चक्षूंषि चर्मञ् चत-चन्द्र चादय द्विषाम्\\
अघोनां हर पाप-चक्षुषाम् ॥ २६॥\\
\\
त्वम् - you;  तिग्म-धार-असि-वर - O best of swords possessing very sharp blades;  अरि-सैन्यम् - the soldiers of the enemy;  ईश-प्रयुक्तः - being engaged by the Supreme Personality of Godhead;  मम - my;  चिन्धि चिन्धि -  chop to pieces, chop to pieces;  चक्षूंषि - the eyes; चर्मन् - O shield;  शत-चन्द्र - possessing brilliant circles like a hundred moons;  चादय - please cover;  द्विषाम् – of those who are envious of me;  अघोनाम् - who are completely sinful;  हर - please take away;  पाप-चक्षुषाम् - of those whose eyes are very sinful.  O king of sharp-edged swords, you are engaged by the Supreme Personality of Godhead. Please cut the soldiers of my enemies to pieces. Please cut them to pieces! O shield marked with a hundred brilliant moonlike circles, please cover the eyes of the sinful enemies. Pluck out their sinful eyes.\\
\\
यन् नो भयं ग्रहेभ्योऽभूत्केतुभ्यो नृभ्य एव च\\
सरीसृपेभ्यो दंष्ट्रिभ्यो भूतेभ्योऽंहोभ्य एव च ॥ २७॥\\
\\
सर्वाण्य् एतानि भगवन्-नाम-रूपानुकीर्तनात्\\
प्रयान्तु सङ्क्षयं सद्यो ये नः श्रेयः-प्रतीपकाः ॥ २८॥\\
\\
यत् - which;  नः - our;  भयम् - fear;  ग्रहेभ्यः – from the Graha demons;  अभूत् - was;  केतुभ्यः - from meteors, or falling stars;  नृभ्यः - from envious human beings;  एव च - also;  सरीसृपेभ्यः - from snakes or scorpions;  दंष्ट्रिभ्यः - from animals with fierce teeth like tigers, wolves and boars; भूतेभ्यः  - from ghosts or the material elements (earth, water, fire, etc.);  अंहोभ्यः - from sinful activities;  एव च - as well as; सर्वाणि एतानि - all these;  भगवत्-नाम-रूप-अनुकीर्तनात् - by glorifying the transcendental form, name, attributes and paraphernalia of the Supreme Personality of Godhead;  प्रयान्तु - let them go; सङ्क्षयम् - to complete destruction;  सद्यः - immediately; ये - which;  नः - our;  श्रेयः-प्रतीपकाः - hindrances to well-being.\\
\\
May the glorification of the transcendental name, form, qualities and paraphernalia of the Supreme Personality of Godhead protect us from the influence of bad planets, meteors, envious human beings, serpents, scorpions, and animals like tigers and wolves. May it protect us from ghosts and the material elements like earth, water, fire and air, and may it also protect us from lightning and our past sins. We are always afraid of these hindrances to our auspicious life. Therefore, may they all be completely destroyed by the chanting of the Hare KRiShNa mahA-mantra.\\
\\
गरुडो भगवान् स्तोत्र-स्तोभश्चन्दोमयः प्रभुः\\
रक्षत्वशेष-कृच्च्रेभ्यो विष्वक्सेनः स्व-नामभिः ॥ २९॥\\
\\
गरुडः - His Holiness GaruDa, the carrier of Lord ViShNu;  भगवान् - as powerful as the Supreme Personality of Godhead;  स्तोत्र-स्तोभः - who is glorified by selected verses and songs;  चन्दः-मयः - the personified Vedas;  प्रभुः -  the lord;  रक्षतु - may He protect;  अशेष-कृच्च्रेभ्यः - from unlimited miseries;  विष्वक्सेनः - Lord ViShvaksena; स्व-नामभिः - by His holy names.\\
\\
Lord GaruDa, the carrier of Lord ViShNu, is the most worshipable lord, for he is as powerful as the Supreme Lord Himself. He is the personified Vedas and is worshiped by selected verses. May he protect us from all dangerous conditions, and may Lord ViShvaksena, the Personality of Godhead, also protect us from all dangers by His holy names.\\
\\
सर्वापद्भ्यो हरेर्नाम-रूप-यानायुधानि नः\\
बुद्धीन्द्रिय-मनः-प्राणान्पान्तु पार्षद-भूषणाः ॥ ३०॥\\
\\
सर्व-आपद्भ्यः - from all kinds of danger;  हरेः - of the Supreme Personality of Godhead;  नाम - the holy name; रूप - the transcendental form;  यान - the carriers;  आयुधानि - and all the weapons;  नः - our;  बुद्धि -  intelligence; इन्द्रिय - senses;  मनः - mind;  प्राणान् - life air; पान्तु - may they protect and maintain;  पार्षद-भूषणाः - the decorations who are personal associates.\\
\\
May the Supreme Personality of Godhead.as holy names, His transcendental forms, His carriers and all the weapons decorating Him as personal associates protect our intelligence, senses, mind and life air from all dangers.\\
\\
PURPORT - There are various associates of the transcendental Personality of Godhead, and His weapons and carrier are among them. In the spiritual world, nothing is material. The sword, bow, club, disc and everything decorating the personal body of the Lord are spiritual living force. Therefore the Lord is called advaya j~nAna, indicating that there is no difference between Him and His names, forms, qualities, weapons and so on. Anything pertaining to Him is in the same category of spiritual existence. They are all engaged in the service of the Lord in varieties of spiritual forms.\\
\\
यथा हि भगवान् एव वस्तुतः सद् असच्च यत्\\
सत्येनानेन नः सर्वे यान्तु नाशम् उपद्रवाः ॥ ३१॥\\
\\
यथा - just as;  हि - indeed;  भगवान् - the Supreme Personality of Godhead;  एव - undoubtedly;  वस्तुतः - at the ultimate issue;  सत् - manifested;  असत् - unmanifested;  च - and;  यत् - whatever;  सत्येन - by the truth;  अनेन -  this;  नः - our;  सर्वे - all;  यान्तु - let them go; नाशम् - to annihilation;  उपद्रवाः - disturbances.\\
\\
The subtle and gross cosmic manifestation is material, but nevertheless it is nondifferent from the Supreme Personality of Godhead because He is ultimately the cause of all causes. Cause and effect are factually one because the cause is present in the effect. Therefore the Absolute Truth, the Supreme Personality of Godhead, can destroy all our dangers by any of His potent parts.\\
\\
यथैकात्म्यानुभावानां विकल्प-रहितः स्वयम्\\
भूषणायुध-लिङ्गाख्या धत्ते शक्तीः स्व-मायया ॥ ३२॥\\
\\
तेनैव सत्य-मानेन सर्व-ज्ञो भगवान् हरिः\\
पातु सर्वैः स्वरूपैर्नः सदा सर्वत्र सर्व-गः ॥ ३३॥\\
\\
यथा - just as;  ऐकात्म्य - in terms of oneness manifested in varieties;  अनुभावानाम् - of those thinking;  विकल्प-रहितः -  the absence of difference;  स्वयम् - Himself;  भूषण - decorations;  आयुध - weapons;  लिङ्ग-आख्याः – characteristics and different names;  धत्ते - possesses;  शक्तीः – potencies like wealth, influence, power, knowledge, beauty and renunciation; स्व-मायया - by expanding His spiritual energy;  तेन एव - by that;  सत्य-मानेन - true understanding;  सर्व-ज्ञः - omniscient;  भगवान् - the Supreme Personality of Godhead; हरिः - who can take away all the illusion of the living entities;  पातु - may He protect;  सर्वैः - with all;  स्व-रूपैः - His forms;  नः - us;  सदा - always;  सर्वत्र - everywhere;  सर्व-गः - who is all-pervasive.\\
\\
The Supreme Personality of Godhead, the living entities, the material energy, the spiritual energy and the entire creation are all individual substances. In the ultimate analysis, however, together they constitute the supreme one, the Personality of Godhead. Therefore those who are advanced in spiritual knowledge see unity in diversity. For such advanced persons, the Lord.as bodily decorations, His name, His fame, His attributes and forms and the weapons in His hand are manifestations of the strength of His potency. According to their elevated spiritual understanding, the omniscient Lord, who manifests various forms, is present everywhere. May He always protect us everywhere from all calamities.\\
\\
PURPORT - A person highly elevated in spiritual knowledge knows that nothing exists but the Supreme Personality of Godhead. This is also confirmed in Bhagavad-gItA (9.4) where Lord KRiShNa says, mayA tatam idaM sarvam, indicating that everything we see is an expansion of His energy. This is confirmed in the ViShNu PurANa (1.22.52):\\

\section{\sanskrit द्वादश स्तुति}
\subsection{\sanskrit प्रथमो अध्याय}
\sanskrit
वन्दे वन्द्यं सदानन्दं वासुदेवं निरञ्जनम् ।\\
इंदिरापतिमाद्यादिवरदेशवरप्रदम् ॥ 01 ॥\\
\\
नमामि निखिलाधीशकिरीटाघृष्ट पीठवत् ।\\
हृत्तमःशमनेऽर्क्काभं श्रीपतेः पादपङ्कजम् ॥ 02 ॥\\
\\
जाम्बूनदाम्बराधारं नितम्बं चिन्त्यमीशितुः ।\\
स्वर्ण्णमञ्जीरसंवीतमारूढं जगदम्भया ॥ 03 ॥\\
\\
उदरं चिन्त्यमीशस्य तनुत्वेऽप्यखिलम्भरम्।\\
वलित्रयाङ्कितं नित्यमुपगूढं श्रियैकया ॥ 04 ॥\\
\\
 स्मरणीयमुरो विष्णोरिन्दिरावासमुत्तमैः ।\\
अनन्तमन्तवदिव भुजयोरन्तरं गतम् ॥ 05 ॥\\
\\
 शङ्खचक्रगदापद्मधराश्चिन्त्या हरेर्भुजाः ।\\
पीनवृत्ता जगद्रक्षाकेवलोद्योगिनोऽनिशम् ॥ 06 ॥\\
\\
सन्ततं चिन्तयेत् कण्ठं भास्वत् कौस्तुभभासकम् ।\\
वैकुण्ठस्याखिला वेदा उद्गीर्यन्तेऽनिशं यतः ॥ 07 ॥\\
\\
स्मरेत यामिनीनाथसहस्रामितकान्तिमत् ।\\
भवतापापनोदीड्यं श्रीपतेर्मुखपङ्कजम् ॥ 08 ॥\\
\\
पूर्ण्णानन्य सुखोद्भासि मन्दस्मितमधीशितुः ।\\
गोविन्दस्य सदा चिन्त्यं नित्यानन्दपदप्रदम् ॥ 09 ॥\\
\\
 स्मरामि भवसन्ताप हानिदामृतसागरम् ।\\
पूर्ण्णानन्दस्य रामस्य सानुरागावलोकनम्  ॥ 10 ॥\\
\\
द्ध्यायेदजस्रमीशस्य पद्मजादिप्रतीक्षितम् ।\\
भ्रूभङ्गं पारमेष्ठ्यादि पददायि विमुक्तिदम्  ॥ 11 ॥\\
\\
सन्ततं चिन्तयेनं तमन्तकाले विशेषतः ।\\
नैवोदापुर्ग्गृणन्तोऽन्तं यद् गुणानामजादयः ॥ 12 ॥\\
\\
\subsection{\sanskrit द्वितियो अध्याय}
\sanskrit
स्वजनोदधिसंवृद्धिपूर्ण्णचन्द्रो गुणार्ण्णवः\\
अमन्दानन्दसान्द्रो नः सदाऽव्यादिन्दिरापतिः ॥ 01 ॥\\
\\
रमाचकोरीविधवे दुष्टसर्प्पोदवह्नये ।\\
सत् पान्थजनगेहाय नमो नारायणाय ते ॥ 02 ॥\\
\\
चिदचिद् भेदमखिलं विधायाऽधाय भुञ्जते ।\\
अव्याकृतगृहस्थाय रमाप्रणयिने नमः ॥ 03 ॥\\
\\
अमन्दगुणसारोऽपि मन्दहासेन वीक्षितः ।\\
नित्यमिन्दिरायाऽऽनन्दसान्द्रो यो नौमि तं हरिम् ॥ 04 ॥\\
\\
वशी वशो न कस्यापि योऽजितो विजिताखिलः ।\\
सर्वकर्त्ता न क्रियते तं नमामि रमापतिम् ॥ 05 ॥\\
\\
अगुणाय गुणोद्रेकस्वरूपायाऽदिकारिणे ।\\
विदारितारिसङ्घाय वासुदेवाय ते नमः ॥ 06 ॥\\
\\
आदिदेवाय देवानां पतये सादितारये ।\\
अनाद्यज्ञानपाराय नमः पारावराश्रय ॥ 07 ॥\\
\\
अजाय जनयित्रेऽस्य विजिताखिलदानव ।\\
अजादिपूज्यपादाय नमस्ते गरुडध्वज ॥ 08 ॥\\
\\
रमारमण एवैको रणजिच्छरणं सताम् ।\\
कारणं कारणस्यापि तरुणादित्यसप्रभः ॥ 09 ॥\\
\\
 इन्दिरामन्दसान्द्राग्र्यकटाक्षप्रेक्षितात्मने ।\\
अस्मदिष्टैककार्याय पूर्ण्णाय हरये नमः ॥ 10 ॥\\
\\
 एवंविधः परो विष्णुरव्याच्छ्रीपुरुषोत्तमः ।\\
तमहं सर्वदा वन्दे श्रीनिकेतं परं हरिम् ॥ 11 ॥\\
\\
\subsection{\sanskrit त्रुतियो अध्याय}
\sanskrit
शङ्खचक्रगदापद्मशार्ङ्गखड्गधरं सदा ।\\
नमामि पितरं नित्य वासुदेवं जगत् पतिम् ॥ 01 ॥\\
\\
सच्चिदानन्दरूपं तमनामयमनन्तरम् ।\\
एकमेकान्तमभयं विष्णुं विश्वहृदि स्थितम् ॥ 02 ॥\\
\\
नानास्वभावमत्यन्तस्वभावेन विचारितम् ।\\
अविशेषमनाद्यन्तं प्रणमामि सनातनम् ॥ 03 ॥\\
\\
कुन्देन्दुसन्निभं वन्दे सुधारसनिभं विभुम् ।\\
ज्ञानमुद्रापुस्तकारिशङ्खाक्षायुधधारिणम् ॥ 04 ॥\\
\\
आनन्दमजरं नित्यमात्मेशममितद्युतिम् ।\\
वन्दे विश्वस्य पितरं विष्णुं विश्वेश्वरं सदा ॥ 05 ॥\\
\\
अमन्दानन्दसन्दोहसन्तोषितजगत्रयम् ।\\
नारायणमणीयांसं वन्दे देवं सदातनम् ॥ 06 ॥\\
\\
सूर्यमण्डलमद्ध्यस्थं वराभयकरोद्यतम् ।\\
सूर्यामितद्युतिं वन्दे नारायणमनामयम् ॥ 07 ॥\\
\\
विश्वं विश्वाकरं वन्दे विश्वस्य प्रपितामहम् ।\\
नानारूपमजं नित्यं विश्वात्मानं प्रजापतिम् ॥ 08 ॥\\
\\
कालाकालविचारादिप्रकालितसदातनम् ।\\
पुरुषं प्रकृतिस्थं च वन्दे विष्णुमजोत्तमम् ॥ 09 ॥\\
\\
ब्रह्मणः पितरं वन्दे शङ्करस्य पितामहम् ।\\
श्रियः पतिमजं नित्यमिन्द्रादिप्रपितामहम् ॥ 10 ॥\\
\\
प्रतिप्रति स्थितं विष्णुं नित्यामृतमनामयम् ।\\
शङ्कचक्रधरं वन्दे वराभयकरोद्यतम् ॥ 11 ॥\\
\\
प्रतापप्रविशेषेण दूरीकृतविदूषणम् ।\\
विदारितारिमत्यन्तप्रसन्नं प्रणमाम्यहम् ॥ 12 ॥\\
\\
अप्रमेयमजं नित्यं विशालममृतं परम् ।\\
कृतनित्यालयं लोके नमस्यामि जगत् पतिम् ॥ 13 ॥\\
\\
सुनित्यसुखमक्षय्यं शुद्धं शान्तं निरञ्जनम् ।\\
लोकालोकविचाराढ्यं नमस्यामि श्रियःपतिम् ॥ 14 ॥\\
\\
अमन्दानन्दसन्दोहसान्द्रमिन्द्रानुजं परम् ।\\
नित्यावदातमेकान्तं प्रमाणातीतमक्षयम् ।\\
लोकालोकपतिं दिव्यं नमस्यामि रमापतिम् ॥ 15 ॥\\
\\
\subsection{\sanskrit चतुर्थो अध्याय}
\sanskrit
कुरु भुङ्क्ष्व च कर्म निजं नियतं हरिपादविनम्रधिया सततं ।\\
हरिरेव परो हरिरेव गुरुर्हरिरेव जगत् पितृमातृगतिः ॥ 01 ॥\\
\\
 न ततोऽस्त्यपरं जगदीड्यतमं परमात् परतः पुरुषोत्तमतः ।\\
तदलं बहुलोकविचिन्तनया प्रवणं कुरु मानसमीशपदे ॥ 02 ॥\\
\\
 यततोऽपि हरेः पदसंस्मरणे सकलं ह्यघमाशु लयं व्रजति ।\\
स्मरतस्तु विमुक्तिपदं परमं स्फुटमेष्यति तत् किमपाक्रियते ॥ 03 ॥\\
\\
 शृणुतामलसत्यवचः परमं शपथेरितमुच्छ्रितबाहुयुगम् ।\\
न हरेः परमो न हरेः सदृशः परमः स तु सर्वचिदात्मगणात्॥04॥\\
\\
यदि नाम परो न भवेत हरिः कथमस्य वशे जगदेतदभूत् ।\\
यदि नाम न तस्य वशे सकलं कथमेव तु नित्यसुखं न भवेत्॥ 05 ॥\\
\\
न च कर्म्मविमामलकालगुणप्रभृतीशमचित्तनु तद्धि यतः ।\\
चिदचित्तनु सर्वमसौ तु हरिर्यमयेदिति वैदिकमस्ति वचः ॥ 06 ॥\\
\\
व्यवहारभिदाऽपि गुरोर्ज्जगतां न तु चित्तगता स हि चोद्यपरम् ।\\
बहवः पुरुषाः पुरुषप्रवरो हरिरित्यवदत् स्वयमेव हरिः ॥ 07 ॥\\
\\
चतुराननपूर्वविमुक्तगणा हरिमेत्य तु पूर्ववदेव सदा ।\\
नियतोच्चविनीचतयैव निजां स्थितिमापुरिति स्म परं वचनम्॥ 08 ॥\\
\\
स्मरणे हि परेशितुरस्य विभोर्म्मलिनानि मनांसि कुतः करणम् ।\\
विमलं हि पदं परमं स्मरतं तरुणार्क्कसवर्ण्णमजस्य हरेः ॥ 09 ॥\\
\\
विमलैः श्रुतिशाणनिशाततमैः सुमनोऽसिभिराशु निहत्य दृढम् ।\\
बलिनं निजवैरिणमात्मतमोभिदमीशमनन्तमुपेत हरिम् ॥ 10 ॥\\
\\\
स हि विश्वसृजो विभुशम्भुपुरन्दरपूर्वमुखानपरानपरान् ।\\
सृजतीड्यतमोऽवति हन्ति निजं पदमापयति प्रणतान् स्वधिया ॥ 11 ॥\\
\\
परमोऽपि रमेशितुरस्य समो न हि कश्चिदभून्न भविष्यति च ।\\
क्वचिदद्यतनोऽपि न पूर्ण्णसदाऽगणितेड्यगुणानुभवैकतनोः ॥ 12 ॥\\
\\
 निजपूर्ण्णसुखामितबोधतनुः परशक्तिरनन्तगुणः परमः ।\\
अजरामरणः सकलार्त्तिहरः कमलापतिरीड्यतमोऽवतु नः ॥ 13 ॥\\
\\
 यदसुप्तिगतो वरिनः सुखवान् सुखरूपिणमाहुरतो निगमाः ।\\
स्वमतिप्रभवं जगदस्य यतः परबोधतनुं च ततः खपतिम् ॥ 14 ॥\\
\\
बहुचित्रजगद् बहुधाकरणात् परशक्तिरनन्तगुणः परमः ।\\
सुखरूपममुष्य पदं परमं स्मरतस्तु भविष्यति तत् सततम्॥ 15 ॥\\
\\
\subsection{\sanskrit पञ्चमो अध्याय}
\sanskrit
 ॐ ॥ पान्त्वस्मान् पुरुहूतवैरिबलवन्मातङ्गमाद्यद् घटा-
कुम्भोच्चाद्रिविपाटनाधिकपटुप्रत्येकवज्रायिताः ।\\
श्रीमत् कण्ठीरवास्यप्रततसुनखरा दारितारातिदूर-
प्रध्वस्तध्वान्तशान्तप्रविततमनसा भाविता भूरिभागैः ॥ * ॥\\
\[लक्ष्मीकान्त समन्ततोऽपि कलयन् नैवेशितुस्ते समं
पश्याम्युत्तमवस्तु दूरतरतोऽपास्तं रसो योऽष्टमः।\\
यद्रोषोत्करदक्षनेत्रकुटिलप्रान्तोत्थिताग्निस्फुरत्-
खद्योतोपमविष्फुलिङ्गभसिता ब्रह्मेशशक्रोत्कराः ॥ * ॥ \]\\
\\
\subsection{\sanskrit शश्टमो अध्याय}
\sanskrit
 वन्दिताशेषवन्द्योरुवृन्दारकं चन्दनाचर्चितोदारपीनांसकम् ।\\
इन्दिराचञ्चलापाङ्गनीराजितं मन्दरोद्धारिवृत्तोद्भुजाभोगिनम् ॥\\
प्रीणयामो वासुदेवं देवतामण्डलाखण्डमण्डानम् प्रीणयामो वासुदेवम् ॥ 1 ॥\\
\\
 सृष्टिसंहारलीलाविलासाततं पुष्टषाड्गुण्यसद्विग्रहोल्लासिनम् ।\\
दुष्टनिःशेषसंहारकर्म्मोद्यतं हृष्टपुष्टातिशिष्टप्रजासंश्रयम् ॥\\
प्रीणयामो वासुदेवं देवतामण्डलाखण्डमण्डानम् प्रीणयामो वासुदेवम्  ॥2॥\\
\\
उन्नतप्रार्थ्थिताशेषसंसाधकं सन्नतालौकिकानन्ददश्रीपदम् ।\\
भिन्नकर्म्माशयाप्राणिसम्प्रेरकम् तन्न किं नेति विद्वत्सु मीमांसितम् ।\\
प्रीणयामो वासुदेवं देवतामण्डलाखण्डमण्डानम् प्रीणयामो वासुदेवम् ॥3॥\\
\\
विप्रमुख्यैः सदा वेदवादोन्मुखैः सुप्रतापैः क्षितीशेश्वरैश्चार्च्चितम् ।\\
अप्रतर्क्क्योरुसंविद् गुणं निर्म्मलं सत्प्रकाशाजरानन्दरूपं परम् ॥\\
प्रीणयामो वासुदेवं देवतामण्डलाखण्डमण्डानम् प्रीणयामो वासुदेवम् ॥ 4 ॥\\
\\
 अत्ययो यस्य केनापि न क्वापि हि प्रत्ययो यद् गुणेषूत्तमानां परः ।\\
सत्यसङ्कल्प एको वरेण्यो वशी मत्यनूनैः सदा वेदवादोदितः ॥\\
प्रीणयामो वासुदेवं देवतामण्डलाखण्डमण्डानम् प्रीणयामो वासुदेवम् ॥ 5 ॥\\
\\
पश्यतां दुःखसन्ताननिर्मूलनं दृष्यतां दृश्यतामित्यजेशार्च्चिम् ।\\
नश्यतां दूरगं सर्वदाऽप्यात्मकं वश्यतां स्वेच्छया सज्जनेष्वागतम् ॥\\
प्रीणयामो वासुदेवं देवतामण्डलाखण्डमण्डानम्  प्रीणयामो वासुदेवम् ॥ 6 ॥\\
\\
अग्रजं यः ससर्ज्जाजमग्र्याकृतिं विग्रहो यस्य सर्वे गुणा एव हि ।\\
उग्र अद्योऽपि यस्याऽत्मजाग्र्यात्मजः सद्गृहीतः सदा यः परं दैवतम् ।\\
प्रीणयामो वासुदेवं देवतामण्डलाखण्डमण्डानम्  प्रीणयामो वासुदेवम् ॥ 7 ॥\\
\\
अच्युतो यो गुणैर्न्नित्यमेवाखिलैः प्रच्युतोऽशेषदोषैः सदा पूर्त्तितः ।\\
उच्यते सर्ववेदोरुवादैरजः स्वर्च्यते ब्रह्मरुद्रेन्द्र पूर्वैः सदा ॥\\
प्रीणयामो वासुदेवं देवतामण्डलाखण्डमण्डानम्  प्रीणयामो वासुदेवम् ॥ 8 ॥\\
\\
धार्यते येन विश्वं सदाऽजादिकं वार्यतेऽशेषदुःखं निजाद्ध्यायिनाम् ।\\
पार्यते सर्वमन्यैर्न्न यत् पार्यते कार्यते चाखिलं सर्वभूतैः सदा।\\
प्रीणयामो वासुदेवं देवतामण्डलाखण्डमण्डानम्  प्रीणयामो वासुदेवम् ॥ 9 ॥\\
\\
सर्वपापानि यत् संस्मृतेः सङ्क्षयं सर्वदा यान्ति भक्त्या विशुद्धात्मनाम्।\\
शर्वगुर्वादिनिर्बाणसंस्थानदः कुर्वते कर्म्म यत् प्रीतये सज्जनाः ॥\\
प्रीणयामो वासुदेवं देवतामण्डलाखण्डमण्डानम्  प्रीणयामो वासुदेवम् ॥10 ॥\\ 
\\
अक्षयं कर्म्म यस्मिन् परे स्वर्प्पितं प्रक्षयं यान्ति दुःखानि यन्नामतः ।\\
अक्षरो योऽजरः  सर्वदैवामृतः कुक्षिगं यस्य विश्वं सदाऽजादिकम् ॥\\
प्रीणयामो वासुदेवं देवतामण्डलाखण्डमण्डानम्  प्रीणयामो वासुदेवम्॥ 11 ॥\\
\\
नन्दितीर्थ्थोरुसन्नामिनो नन्दिना सन्दधानाः सदानन्ददेवे मतिम् ।\\
मन्दहासारुणापाङ्गदत्तोन्नतिं नन्दिताशेषदेवादिवृन्दं सदा ॥\\
प्रीणयामो वासुदेवं देवतामण्डलाखण्डमण्डनम्  प्रीणयामो वासुदेवम् ॥ 12 ॥\\
\\
\subsection{\sanskrit अश्टमो अध्याय}
\sanskrit
 देवकिनन्दन नन्दकुमारा वृन्दवनाञ्जन गोकुलचन्द्र ॥ 01 ॥\\
 कन्दफलाशन सुन्दररूपा नन्दितगोकुल वन्दितपाद ॥ 02 ॥\\
 इन्द्रसुतावक नन्दकहस्ता चन्दनचर्च्चित सुन्दरिनाथ ॥ 03 ॥\\
 इन्दिवरोदरदळनयनाऽऽमन्दरधारा गोविन्द वन्दे ॥ 04 ॥\\
 मत्स्यकरूप लयोदविहारिन् वेदविनेत्र चतुर्मुखवन्द्य ॥ 05 ॥\\
 कूर्मस्वरूपक मन्दरधारिन् लोकविधारक देववरेण्य ॥ 06 ॥\\
 सूकररूपक दानवशत्रो भूमिविधारक यङ्ञवराङ्ग ॥ 07 ॥\\
 देव नृसिंह हिरण्यकशत्रो सर्वभयान्तक दैवतबन्धो ॥ 08 ॥\\
 वामनवामन माणववेषा दैत्यकुलान्तक कारणभूत ॥ 09 ॥\\
 राम भृगूद्वह सूर्ज्जितदीप्ते क्षत्रकुलान्तक शम्भुवरेण्य ॥ 10 ॥\\
 राघवराघव राक्षसशत्रो मारुतिवल्लभ जानकिकान्त ॥ 11 ॥\\
 देवकिनन्दन सुन्दररूपा रुग्मिणिवल्लभ पाण्डवबन्धो ॥ 12 ॥\\
 दैत्यविमोहक नित्यसुखादे देवसुबोधक बुद्धस्वरूप ॥ 13 ॥\\
 दुष्टकुलान्तक कल्किस्वरूपा धर्म्मविवर्द्धन मूलयुगादे ॥ 14 ॥\\
 आनन्दतीर्त्थकृता हरिगाथा पापहरा शुभनित्यसुखार्त्था ॥ 15 ॥\\
\\
\subsection{\sanskrit नवमो अध्याय}
\sanskrit
 अतिमत तमॊगिरि समितिविभेदन पितमहविभूतिद गुणगणनिलय ।\\
शुभतमकथशय परम सदोदित जगदॆककारण रम रमरमणा॥ 01 ॥\\
\\
 विधिभवमुखसुरसततसुवन्दितरममनॊवल्लभ भव मम शरणम् ।\\
शुभतमकथशय परम सदोदित जगदॆककारण रम रमरमणा॥ 02 ॥\\
\\
अपरिमितगुणगणमयशरीर हॆ विगतगुणेतर भव मम शरणम् ।
शुभतमकथशय परम सदोदित जगदॆककारण रम रमरमणा॥ 03 ॥\\
\\
अगणितसुखनिधिविमलसुदेह हॆ विगतसुखेतर भव मम शरणम् ।\\
शुभतमकथशय परम सदोदित जगदॆककारण रम रमरमणा॥ 04 ॥\\
\\
 प्रचलितलयजलविहरण शाश्वत सुखमय मीन हॆ भव मम शरणम्।\\
शुभतमकथशय परम सदोदित जगदॆककारण रम रमरमणा॥ 05 ॥\\
\\
सुरदितिजसुबलविलुळितमन्दरधर परकूर्म्म हॆ भव मम शरणम् ।\\
शुभतमकथशय परम सदोदित जगदॆककारण रम रमरमणा ॥ 06 ॥\\
\\
सगिरिवरधरतळवह सुसूकर परमविबोध हॆ भव मम शरणम् ।\\
शुभतमकथशय परम सदोदित जगदॆककारण रम रमरमणा॥ 07 ॥\\
\\
अतिबलदितिसुतहृदयविभेदन जय नृहरेऽमल भव मम शरणम्।\\
शुभतमकथशय परम सदोदित जगदॆककारण रम रमरमणा॥ 08 ॥\\
\\
बलिमुखदितिसुतविजयविनाशन जगदवनाजित भव मम शरणम् ।\\
शुभतमकथशय परम सदोदित जगदॆककारण रम रमरमणा॥ 09 ॥\\
\\
अविजितकुनृपतिसमितिविखण्डन रमवर वीरप भव मम शरणम् ।\\
शुभतमकथशय परम सदोदित जगदॆककारण रम रमरमणा॥ 10 ॥\\
\\
खरतरनिशिचरदहन परामृत रघुवर मानद भव मम शरणम् ।\\
शुभतमकथशय परम सदोदित जगदॆककारण रम रमरमणा॥ 11 ॥\\
\\
सुलळिततनुवर वरद महाबल यदुवर पार्त्थप भव मम शरणम्।\\
शुभतमकथशय परम सदोदित जगदॆककारण रम रमरमणा॥ 12 ॥\\
\\
दितिसुतविमॊहन विमलविबोधन परगुणबुद्ध हॆ भव मम शरणम्।\\
शुभतमकथशय परम सदोदित जगदॆककारण रम रमरमणा॥ 13 ॥\\
\\
कलिमलहुतवह सुभग महोत्सव शरणद कल्किश भव मम शरणम् ।\\
शुभतमकथशय परम सदोदित जगदॆककारण रम रमरमणा॥ 14 ॥\\
\\
इति तव नुतिवरसततरतेर्भव शरणमुरुसुखतीर्र्थमुनेर्भगवन्।\\
शुभतमकथशय परम सदोदित जगदॆककारण रम रमरमणा॥ 15 ॥\\
\\
\subsection{\sanskrit दशमो अध्याय}
\sanskrit
 केशवकेशव शासक वन्दे पाशधरार्च्चित शूर परेश ।\\
नारायणामरतारण वन्दे कारणकारण पूर्ण्ण वरेण्य ॥ 01 ॥\\
\\
माधवमाधव शोधक वन्दे बाधक बोधक शुद्धसमाधे ।\\
गोविन्दगोविन्द पुरन्दर वन्दे स्कन्दसनन्दनवन्दितपाद ॥ 02 ॥\\
\\
विष्णु सृजिष्णु ग्रसिष्णु वि वन्दे कृष्ण सदुष्णवधिष्ण स्वधृष्णो ।\\
मधुसूदन दानवसादन वन्दे दैवतमोदन वेदितपाद ॥ 03 ॥\\
\\
त्रिविक्रम निष्क्रम विक्रम वन्दे सुक्रम सङ्क्रमहुङ्कृतवक्त्र ।\\
वामनवामन भामन वन्दे सामन सीमन शामन सानो ॥ 04 ॥\\
\\
श्रीधरश्रीधर शन्धर वन्दे भूर्द्धर वार्द्धर कन्धरधारिन् ।\\
हृषीकेश सुकेश परेश वि वन्दे शरणेश कलेश बलेश सुखेश ॥ 05 ॥\\
\\
पद्मनाभ शुभोद्भव वन्दे सम्भृतलोकभराभर भूरे ।\\
दामोदर दूरतरान्तर वन्दे दारितपारक पार परस्मात् ॥ 06 ॥\\
\\
आनन्दसुतीर्त्थमुनीन्द्र कृता हरिगीतिरियं परमादरतः ।\\
परलोकविलोकनसूर्यनिभा हरिभक्तिविवर्द्धनशौण्डतमा ॥ 07 ॥\\
\\
\subsection{\sanskrit एकदशो अध्याय}
\sanskrit
अवन श्रीपतिरप्रतिरधिकेशादिभवादे ।\\
करुणापूर्ण्ण वरप्रद चरितं ङ्ञापय मे ते ॥ 01 ॥\\
\\
सुरवन्द्याधिप सद्वर भरिताशेषगुणालम् ।\\
करुणापूर्ण्ण वरप्रद चरितं ङ्ञापय मे ते ॥ 02 ॥\\
\\
सकलध्वान्तविनाशन परमानन्दसुधाहो ।\\
करुणापूर्ण्ण वरप्रद चरितं ङ्ञापय मे ते ॥ 03 ॥\\
\\
त्रिजगत् पोत सदार्च्चितचरणाशापतिधातो ।\\
करुणापूर्ण्ण वरप्रद चरितं ङ्ञापय मे ते ॥ 04 ॥\\
\\
त्रिगुणातीत विधारक परितो देहि सुभक्तिम् ।\\
करुणापूर्ण्ण वरप्रद चरितं ङ्ञापय मे ते ॥ 05 ॥\\
\\
मरणप्राणद पालक जगदीशाव सुभक्तिम् ।\\
करुणापूर्ण्ण वरप्रद चरितं ङ्ञापय मे ते ॥ 06 ॥\\
\\
शरणं कारण भावन भव मे तात सदाऽलम् ।\\
करुणापूर्ण्ण वरप्रद चरितं ङ्ञापय मे ते ॥ 07 ॥\\
\\
तरुणादित्यसवर्ण्णक चरणाब्जामलकीर्त्ते ।\\
करुणापूर्ण्ण वरप्रद चरितं ङ्ञापय मे ते ॥ 08 ॥\\
\\
सलिलप्रोत्थसरागकमणिवर्ण्णोच्चनखादे ।\\
करुणापूर्ण्ण वरप्रद चरितं ङ्ञापय मे ते ॥ 09 ॥\\
\\
कजतूणीनिभपावनवरजङ्घामितशक्ते ।\\
करुणापूर्ण्ण वरप्रद चरितं ङ्ञापय मे ते ॥ 10 ॥\\
\\
असनोत्फुल्लसुपुष्पकसमवर्ण्णावरणांन्ते ।\\
करुणापूर्ण्ण वरप्रद चरितं ङ्ञापय मे ते ॥ 11 ॥\\
\\
शतमोदोद्भवसुन्दरवरपद्मोत्थितनाभे ।\\
करुणापूर्ण्ण वरप्रद चरितं ङ्ञापय मे ते ॥ 12 ॥\\
\\
इभहस्तप्रभशोभनपरमोरुस्थरमाळे ।\\
करुणापूर्ण्ण वरप्रद चरितं ङ्ञापय मे ते ॥ 13 ॥\\
\\
जगदागूहकपल्लवसमकुक्षे शरणादे ।\\
करुणापूर्ण्ण वरप्रद चरितं ङ्ञापय मे ते ॥ 14 ॥\\
\\
जगदम्बामलसुन्दरिगृहवक्षोवरयोगिन् ।\\
करुणापूर्ण्ण वरप्रद चरितं ङ्ञापय मे ते ॥ 15 ॥\\
\\
दितिजान्तप्रदचक्रधरगदायुग्वरबाहो ।\\
करुणापूर्ण्ण वरप्रद चरितं ङ्ञापय मे ते ॥ 16 ॥\\
\\
परमज्ञानमहानिधिवदनश्रीरमणेन्दो ।\\
करुणापूर्ण्ण वरप्रद चरितं ङ्ञापय मे ते ॥ 17 ॥\\
\\
निखिलाघौघविनाशकपरसौख्यप्रददृष्टे ।\\
करुणापूर्ण्ण वरप्रद चरितं ङ्ञापय मे ते ॥ 18 ॥\\
\\
परमानन्दसुतीर्त्थसुमुनिराजो हरिगाथाम्।\\
कृतवान् नित्यसुपूर्ण्णकपरमानन्दपदैषिन् ॥ 19 ॥\\
\\
\subsection{\sanskrit द्वादशो अध्याय}
\sanskrit
उदीर्ण्णमजरं दिव्यममृतस्यन्द्यधीशितुः ।\\
आनन्दस्य पदं वन्दे ब्रह्मेन्द्राद्यभिवन्दितम् ॥ 01 ॥\\
\\
सर्ववेदपदोद्गीतमिन्दिराधारमुत्तमम् ।\\
आनन्दस्य पदं वन्दे ब्रह्मेन्द्राद्यभिवन्दितम् ॥ 02 ॥\\
\\
सर्वदेवादिदेवस्य विदारितमहत्तमः ।\\
आनन्दस्य पदं वन्दे ब्रह्मेन्द्राद्यभिवन्दितम् ॥ 03 ॥\\
\\
उदारमादरान्नित्यमनिन्द्यं सुन्दरीपतेः ।\\
आनन्दस्य पदं वन्दे ब्रह्मेन्द्राद्यभिवन्दितम् ॥ 04 ॥\\
\\
इन्दीवरोदरनिभं सुपूर्ण्णं वादिमोहनम् ।\\
आनन्दस्य पदं वन्दे ब्रह्मेन्द्राद्यभिवन्दितम् ॥ 05 ॥\\
\\
दातृ सर्वामरैश्वर्यविमुक्त्यादेरहो वरम् ।\\
आनन्दस्य पदं वन्दे ब्रह्मेन्द्राद्यभिवन्दितम् ॥ 06 ॥\\
\\
दूराद् दूरतरं यत्तु तदेवान्तिकमन्तिकात् ।\\
आनन्दस्य पदं वन्दे ब्रह्मेन्द्राद्यभिवन्दितम् ॥ 07 ॥\\
\\
पूर्ण्णसर्वगुणैकार्ण्णमनाद्यन्तं सुरेशितुः ।\\
आनन्दस्य पदं वन्दे ब्रह्मेन्द्राद्यभिवन्दितम् ॥ 08 ॥\\
\\
आनन्दतीर्थमुनिना हरेरानन्दरूपिणः\\
कृतं स्तोत्रमिदं पुण्यं पठन्नानन्दमाप्नुयात् ॥ 09 ॥\\
 -13-
आनन्द मुकुन्द अरविन्दनयन । आनन्दतीर्त्थपरानन्दवरद ॥ 01 ॥
सुन्दरिमन्दिर गोविन्द वन्दे । आनन्दतीर्त्थपरानन्दवरद ॥ 02 ॥
चन्द्रकमन्दिरनन्दक वन्दे । आनन्दतीर्त्थपरानन्दवरद ॥ 03 ॥
चन्द्रसुरेन्द्र सुवन्दित वन्दे । आनन्दतीर्त्थपरानन्दवरद ॥ 04 ॥
वृन्दारवृन्दसुवन्दित वन्दे । आनन्दतीर्त्थपरानन्दवरद ॥ 05 ॥
इन्दिरस्यन्दनस्यन्दक वन्दे । आनन्दतीर्त्थपरानन्दवरद ॥ 06 ॥
इन्दिरानन्दक सुन्दर वन्दे । आनन्दतीर्त्थपरानन्दवरद ॥ 07 ॥
मन्दारस्यन्दितमन्दिर वन्दे । आनन्दतीर्त्थपरानन्दवरद ॥ 08 ॥
आनन्दचन्द्रिकास्यन्दक वन्दे । आनन्दतीर्त्थपरानन्दवरद ॥ 09 ॥
-14-
रामोऽखिलानन्दतनुः स एव भीमोऽतिपापेषु दुरन्तवीर्यः ।
कामः स एवाजितकामिनीषु सोमः स एवाऽत्मपदाश्रितेषु ॥01॥
 -15-
अम्बरगङ्गाचुम्बितपादः पदतलविदलितगुरुतरशकटः ।
कालियनागक्ष्वेळनिहन्ता सरसिजनवदळविकसितनयनः ॥01॥
 कालघनालीकर्बुरकायः शरशतशकलितरिपुशतनिवहः ।
सन्ततमस्मान् पातु मुरारिः सततगसमजवखगपतिनिरतः ॥02॥
 ॥ मुख्यप्राण वशे सर्वं स विष्णोर्वशगः सदा ॥
