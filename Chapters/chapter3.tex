प्रोष्ठीशविग्रह सुनिष्ठीवनोद्धत विशिष्टाम्बुचारिजलधे  ।\\
कोष्ठान्तराहितविचेष्टागमौघ परमेष्ठीडित त्वमवमाम्  ।\\
प्रेष्ठार्कसूनुमनुचेष्ठार्थमात्मविदतीष्टो युगान्तसमये  ।\\
स्थेष्ठात्मश‍ृङ्गधृतकाष्ठाम्बुवाहन वराष्टापदप्रभतनो  ॥ १॥\\
\\
खण्डीभवद्बहुळडिण्डीरजृम्भण सुचण्डी कृतो दधि महा  ।\\
काण्डाति चित्र गति शौण्डाद्य हैमरद भाण्डा प्रमेय चरित  ।\\
चण्डाश्वकण्ठमद शुण्डाल दुर्हृदय गण्डा भिखण्डाकर दो-\\
श्चण्डा मरेशहय तुण्डाकृते दृशमखण्डामलं प्रदिश मे  ॥ २॥\\
\\
कूर्माकृते त्ववतु नर्मात्म पृष्ठधृत भर्मात्म मन्दर गिरे  ।\\
धर्मावलम्बन सुधर्मासदाकलितशर्मा सुधावितरणात्  ।\\
दुर्मान राहुमुख दुर्मायि दानवसुमर्माभिभेदन पटो  ।\\
धर्मार्क कान्ति वर वर्मा भवान् भुवन निर्माण धूत विकृतिः  ॥ ३॥\\
\\
धन्वन्तरेऽङ्गरुचि धन्वन्तरेऽरितरु धन्वन्स्तरीभवसुधा-
भान्वन्तरावसथ मन्वन्तराधिकृत तन्वन्तरौषधनिधे  ।\\
दन्वन्तरङ्गशुगुदन्वन्तमाजिषु वितन्वन्ममाब्धि तनया  ।\\
सून्वन्तकात्महृदतन्वरावयव तन्वन्तरार्तिजलधौ  ॥ ४॥\\
\\
या क्षीरवार्धिमथनाक्षीणदर्पदितिजाक्षोभितामरगणा-
पेक्षाप्तयेऽजनि वलक्षांषुबिम्बजिदतीक्ष्णालकावृतमुखी  ।\\
सूक्ष्मावलग्नवसनाक्षेपकृत्कुच कटाक्षाक्षमीकृतमनो-\\
दीक्षासुराहृतसुधाक्षाणिनोऽवतुसु रूक्षेक्षणाद्धरितनुः  ॥ ५॥\\
\\
शिक्षादियुङ्निगम दीक्षासुलक्षण परिक्षाक्षमाविधिसती  ।\\
दाक्षायणी क्षमति साक्षाद्रमापिनय दाक्षेपवीक्षणविधौ  ।\\
प्रेक्षाक्षिलोभकरलाक्षार सोक्षित पदाक्षेपलक्षितधरा  ।\\
साऽक्षारितात्मतनु भूक्षारकारिनिटिलाक्षाक्षमानवतु नः  ॥ ६॥\\
\\
नीलाम्बुदाभशुभ शीलाद्रिदेहधर खेलाघृतोधधिधुनी-\\
शैलादियुक्त निखिलेला कटाद्यसुर तूलाटवीदहन ते  ।\\
कोलाकृते जलधि कालाचलावयव नीलाब्जदंष्ट्र धरणी-\\
लीलास्पदोरुतर मूलाशियोगिवर जालाभिवन्दित नमः  ॥ ७॥\\
\\
दम्भोलितीक्ष्णनख सम्भेदितेन्द्ररिपु कुम्भीन्द्र पाहि कृपया  ।\\
स्तम्भार्भ कासहनडिम्भाय दत्तवर गम्भीरनाद नृहरे  ।\\
अंभोधिजानुसरणांभोजभूपवनकुम्भीनसेशखगराट्  ।\\
कुम्भीन्द्रकृत्तिधर जम्भारिषण्मुखमुखांभोरुहाभिनुत माम्  ॥ ८॥\\
\\
पिङ्गाक्ष विक्रम तुरङ्गादि सैन्य चतुरङ्गा वलिप्त दनुजा-\\
साङ्गाध्वरस्थ बलि साङ्गावपात हृषिताङ्गा मरालिनुत ते  ।\\
श‍ृङ्गारपादनख तुङ्गाग्रभिन्न कन काङ्गाण्डपत्तितटिनी-\\
तुङ्गाति मङ्गल तरङ्गाभिभूत भज काङ्गाघ वामन नमः  ॥ ९॥\\
\\
ध्यानार्ह वामनतनोनाथ पाहि यजमाना सुरेशवसुधा-\\
दानाय याचनिक लीनार्थवाग्वशितनानासदस्यदनुज  ।\\
मीनाङ्कनिर्मलनिशानाथकोटिलसमानात्म मौञ्जिगुण कौ-\\
पीनाच्छसूत्रपदयानातपत्रकरकानम्यदण्डवरभृत्  ॥ १०॥\\
\\
धैर्याम्बुधे परशुचर्याधिकृत्तखलवर्यावनीश्वर महा-\\
शौर्याभिभूत कृतवीर्यात्मजातभुजवीर्यावलेपनिकर  ।\\
भार्यापराधकुपितार्याज्ञयागलितनार्यात्मसूगलतरो  ।\\
कार्यापराधमविचार्यार्यमौघजयिवीर्यामिता मयि दया  ॥ ११॥\\
\\
श्रीरामलक्ष्मणशुकाराम भूरवतुगौरामलामितमहो-\\
हारामरस्तुत यशोरामकान्तिसुत नोरामनोरथहर  ।\\
स्वारामवर्यरिपु वीरामयार्धिकर चीरामलावृतकटे ।\\
स्वाराम दर्शनजमारामयागतसुघोरामनोरमलब्धकलह  ॥ १२॥\\
\\
श्रीकेशवप्रदिशनाकेश जातकपिलोकेश भग्नरविभू-\\
तोकेतरार्तिहरणाकेवलार्तसुखधीकेकिकालजलद  ।\\
साकेतनाथवरपाकेरमुख्यसुत कोकेन भक्तिमतुलाम्  ।\\
राकेन्दु बिम्बमुख काकेक्षणापह हृशीकेश तेऽङ्घ्रिकमले  ॥ १३॥\\
\\
रामे नृणां हृदभिरामेनराशिकुलभीमे मनोऽद्यरमताम्  ।\\
गोमेदिनीजयितपोऽमेयगाधिसुतकामेनिविष्ट मनसि  ।\\
श्यामे सदा त्वयि जितामेयतापसजरामे गताधिकसमे  ।\\
भीमेशचापदलनामेयशौर्यजितवामेक्षणे विजयिनि  ॥ १४॥\\
\\
कान्तारगेहखलकान्तारटद्वदन कान्तालकान्तकशरम्  ।\\
कान्तारयाम्बुजनिकान्तान्ववायविधुकान्ताश्मभादिपहरे  ।\\
कान्तालिलोलदलकान्ताभिशोभितिलकान्ताभवन्तमनुसा  ।\\
कान्तानुयानजित  कान्तारदुर्गकटकान्ता रमात्ववतु माम्  ॥ १५॥\\
\\
दान्तं दशाननसुतान्तं धरामधिवसन्तं प्रचण्डतपसा  ।\\
क्लान्तं समेत्य विपिनान्तं त्ववाप यमनन्तं तपस्विपटलम्   ।\\
यान्तं भवारतिभयान्तं ममाशु भगवन्तं भरेण भजतात्  ।\\
स्वान्तं सवारिदनुजान्तं धराधरनिशान्तं स तापसवरम्  ॥ १६॥\\
\\
शम्पाभचापलवकंपास्तशत्रुबलसम्पादितामितयशाः  ।\\
शं पादतामरससम्पातिनोऽलमनुकम्पारसेन दिश मे  ।\\
सम्पातिपक्षिसहजं पापिरावणहतं पावनं यदकृथाः  ।\\
त्वं पापकूपपतितं पाहि मां तदपि पम्पासरस्तटचर  ॥ १७॥\\
\\
लोलाक्ष्यपेक्षितसुलीलाकुरङ्गवधखेलाकुतूहलगते  ।\\
स्वालापभूमिजनिबालापहार्यनुजपालाद्य भो जयजय  ।\\
बालाग्निदग्धपुरशालानिलात्मजनिफालात्तपत्तलरजो  ।\\
नीलाङ्गदादिकपिमालाकृतालिपथमूलाभ्यतीतजलधे  ॥ १८॥\\
\\
तूणीरकार्मुक कृपाणीकिणाङ्कभुजपाणीरविप्रतिमभाः  ।\\
क्षोणिधरालिनिभघोणीमुखादिघनवेणीसुरक्षणकरः  ।\\
शोणिभवन्नयन कोणीजिताम्बुनिधिपाणीरितार्हणमणि-\\
श्रेणीवृताङ्घ्रिरिह वाणीशसूनुवरवाणीस्तुतो विजयते  ॥ १९॥\\
\\
हुङ्कारपूर्वमथ टङ्कारनादमतिपङ्कावधार्यचलिता  ।\\
लङ्काशिलोच्चयविशङ्कापतद्भिदुर शङ्काऽऽस यस्य धनुषः  ।\\
लङ्काधिपोऽमनुत यं कालरात्रिमिव शङ्काशताकुलधिया  ।\\
तं कालदण्डशतसङ्काशकार्मुकशराङ्कान्वितं भज हरिम्  ॥ २०॥\\
\\
धीमानमेयतनुधामार्तमङ्गळदनामा रमाकमलभू-\\
कामारिपन्नगपकामाहिवैरिगुरुसोमादिवन्द्यमहिमा  ।\\
स्थेमादिनापगतसीमावतात्सखलसामाजरावणरिपू  ।\\
रामाभिदो हरिरभौमाकृतिः प्रतनसामादिवेदविषयः  ॥ २१॥\\
\\
दोषात्मभूवशतुराषाडतिक्रमजदोषात्मभर्तृवचसा  ।\\
पाषाणभूतमुनियोषावरात्मतनुवेषादिदायिचरणः  ।\\
नैषादयोषिदशुभेषाकृदण्डजनिदोषाचरादिशुभदो  ।\\
दोषाग्रजन्ममृतिशोषापहोऽवतु सुदोषाङ्घ्रिजातहननात्  ॥ २२॥\\
\\
वृन्दावनस्थपशुवृन्दावनं विनुतवृन्दारकैकशरणम्  ।\\
नन्दात्मजं निहतनिन्दाकृदासुरजनं दामबद्धजठरम्  ।\\
वन्दामहे वयममन्दावदातरुचिमान्दाक्षकारिवदनम्  ।\\
कुन्दालिदन्तमुत कन्दासितप्रभतनुं दावराक्षसहरम्  ॥ २३॥\\
\\
गोपालकोत्सवकृतापारभक्ष्यरससूपान्नलोपकुपिता  ।\\
शापालयापितलयापाम्बुदालिसलिलापायधारितगिरे  ।\\
स्वापाङ्गदर्शनज तापाङ्गरागयुतगोपाङ्गनांशुकहृति-\\
व्यापारशौण्ड विविधापायतस्त्वमव गोपारिजातहरण  ॥ २४॥\\
\\
कंसादिकासदवतंसावनीपतिविहिंसाकृतात्मजनुषम्  ।\\
संसारभूतमिह संसारबद्धमनसं सारचित्सुखतनुम्  ।\\
संसाधयन्तमनिशं सात्त्विकव्रजमहं सादरं बत भजे  ।\\
हंसादितापसरिरंसास्पदं परमहंसादिवन्द्यचरणम्  ॥ २५॥\\
\\
राजीवनेत्र विदुराजीव मामवतु राजीवकेतनवशम्  ।\\
वाजीभपत्तिनृपराजीरथान्वितजराजीवगर्वशमन  ।\\
वाजीशवाह सितवाजीश दैत्यतनुवाजीशभेदकरदोः  ।\\
जाजीकदम्बनवराजीवमुख्यसुमराजीसुवासितशिरः  ॥ २६॥\\
\\
कालीहृदावसथकालीयकुण्डलिपकालीस्थपादनखरा  ।\\
व्यालीनवांशुकरवालीगणारुणितकालीरुचे जय जय  ।\\
केलीलवापहृतकालीशदत्तवरनालीकदृप्तदितिभू-\\
चूलीकगोपमहिलालीतनूघुसृणधूलीकणाङ्कहृदय  ॥ २७॥\\
\\
कृष्णादिपाण्डुसुतकृष्णामनःप्रचुरतृष्णासुतृप्तिक रवाक्  ।\\
कृष्णाङ्कपालिरत कृष्णाभिधाघहर कृष्णादिषण्महिळ भोः  ।\\
पुष्णातु मामजित निष्णातवार्धिमुदनुष्णांशुमण्डल हरे  ।\\
जिष्णो गिरीन्द्रधर विष्णो वृषावरज धृष्णो भवान्करुणया  ॥ २८॥\\
\\
रामाशिरोमणिधरामासमेत बलरामानुजाभिध रतिम्  ।\\
व्योमासुरान्तकर ते मारतात दिश मे माधवाङ्घ्रिकमले  ।\\
कामार्तभौमपुररामावलीप्रणयवामाक्षिपीततनुभा  ।\\
भीमाहिनाथमुखवैमानिकाभिनुत भीमाभिवन्द्यचरण  ॥ २९॥\\
\\
सक्ष्वेळभक्ष्यभयदाक्षिश्रवोगणजलाक्षेपपाशयमनम्  ।\\
लाक्षागृहज्वलनरक्षोहिडिम्बबकभैक्षान्नपूर्वविपदः  ।\\
अक्षानुबन्धभवरूक्षाक्षरश्रवणसाक्षान्महिष्यवमती  ।\\
कक्षानुयानमधमक्ष्मापसेवनमभीक्ष्णापहासमसताम्  ॥ ३०॥\\
\\
चक्षाण एव निजपक्षाग्रभूदशशताक्षात्मजादिसुहृदा-\\
माक्षेपकारिकुनृपाक्षौहिणीशतबलाक्षोभदीक्षितमनाः  ।\\
तार्क्ष्यासिचापशरतीक्ष्णारिपूर्वनिजलक्ष्माणि चाप्यगणयन्  ।\\
वृक्षालयध्वजरिरक्षाकरो जयति लक्ष्मीपतिर्यदुपतिः  ॥ ३१॥\\
\\
बुद्धावतार कविबद्धानुकम्प कुरु बद्धाञ्जलौ मयि दयाम्  ।\\
शौद्धोदनिप्रमुखसैद्धान्तिकासुगमबौद्धागमप्रणयन  ।\\
क्रुद्धाहितासुहृतिसिद्धासिखेटधर शुद्धाश्वयान कमला  ।\\
शुद्धान्त मां रुचिपिनद्धाखिलाङ्ग निजमद्धाव कल्क्यभिध भोः  ॥ ३२॥\\
\\
सारङ्गकृत्तिधरसारङ्गवारिधर सारङ्गराजवरदा-\\
सारं गदारितरसारं गतात्ममदसारं गतौषधबलम्  ।\\
सारङ्गवत्कुसुमसारं गतं च तव सारङ्गमाङ्घ्रियुगलम्  ।\\
सारङ्गवर्णमपसारं गताब्जमदसारं गदिंस्त्वमव माम्  ॥ ३३॥\\
\\
ग्रीवास्यवाहतनुदेवाण्डजादिदशभावाभिरामचरितम्  ।\\
भावातिभव्यशुभधीवादिराजयतिभूवाग्विलासनिलयम्  ।\\
श्रीवागधीशमुखदेवाभिनम्यहरिसेवार्चनेषु पठता-\\
मावास एव भवितावाग्भवेतरसुरावासलोकनिकरे  ॥ ३४॥\\
\\
इति श्रीमद्वादिराजपूज्यचरण विरचितं\\
श्रीदशावतारस्तुतिः सम्पूर्णम्\\
 भारतीरमणमुख्यप्राणान्तर्गत श्रीकृष्णार्पणमस्तु\\