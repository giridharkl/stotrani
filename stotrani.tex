\documentclass[twoside,a5paper]{report}
\usepackage[margin=1in]{geometry}
\usepackage{xcolor}
\usepackage{tocloft}
\usepackage{fancyhdr}
  \fancyhf{}
  \fancyfoot[RO]{\vfootline\hskip\linepagesep\thepage}
  \fancyfoot[LE]{\thepage\hskip\linepagesep\vfootline}
  \fancyhead[RO]{Sanskrit}
  \fancyhead[LE]{Stotrani}
  \renewcommand\headrulewidth{1pt}
  \fancypagestyle{plain}{%
    \fancyhf{}
    \fancyfoot[RO]{\vfootline\hskip\linepagesep\thepage}
    \fancyfoot[LE]{\thepage\hskip\linepagesep\vfootline}
    \renewcommand\headrulewidth{0pt}
  }
  \newskip\linepagesep \linepagesep 5pt\relax
  \renewcommand\footrulewidth{0.5pt}
  \def\vfootline{%
    \begingroup\color{blue}\rule[-990pt]{20pt}{1000pt}\endgroup}
\usepackage{lipsum}
% This assumes your files are encoded as UTF8
\usepackage[utf8]{inputenc}
% Devanagari Related Packages
\usepackage{fontspec}
\usepackage{setspace}
\usepackage[english]{babel}
\newfontfamily\sanskrit[Script=Devanagari]{Siddhanta}
\newfontfamily\chandas[Script=Devanagari]{Siddhanta}


\newcommand*\NewPage{\newpage\null\thispagestyle{empty}\newpage}

\pagestyle{fancy}
\title{\Huge \sanskrit संस्कृत स्तोत्रानि}
\author{\small \sanskrit सङ्ग्रहकारः - गिरिधर कन्दाचार्}
\renewcommand*\contentsname{\sanskrit {विशय सूचि}}


\begin{document}
\renewcommand{\chaptername}{\Huge \color{blue}\sanskrit अध्याय}
\maketitle
\NewPage
\pagenumbering{roman}
\tableofcontents
\pagenumbering{arabic}
\setstretch{1.25} %line spacing
\chapter{\color{orange}\sanskrit श्री शिव स्तोत्रानि }
\section{\sanskrit काशी विश्वनाथ स्तोत्रम्}
\begin{quotation}\chandas
गङ्गातरङ्गरमणीयजटाकलापं\\
गौरीनिरन्तरविभूषितवामभागम् ।\\
नारायणप्रियमनङ्गमदापहारं\\
वाराणसीपुरपतिं भज विश्वनाथम् ॥१॥\\
\\
वाचामगोचरमनेकगुणस्वरूपं\\
वागीशविष्णुसुरसेवितपादपीठम् ।\\
वामेन विग्रहवरेण कलत्रवन्तं\\
वाराणसीपुरपतिं भज विश्वनाथम् ॥२॥\\
\\
भूताधिपं भुजगभूषणभूषिताङ्गं\\
व्याघ्राजिनाम्बरधरं जटिलं त्रिनेत्रम् ।\\
पाशाङ्कुशाभयवरप्रदशूलपाणिं\\
वाराणसीपुरपतिं भज विश्वनाथम् ॥३॥\\
\\
शीतांशुशोभितकिरीटविराजमानं\\
भालेक्षणानलविशोषितपञ्चबाणम् ।\\
नागाधिपारचितभासुरकर्णपुरं\\
वाराणसीपुरपतिं भज विश्वनाथम् ॥४॥\\
\\
पञ्चाननं दुरितमत्तमतङ्गजानां\\
नागान्तकं दनुजपुङ्गवपन्नगानाम् ।\\
दावानलं मरणशोकजराटवीनां\\
वाराणसीपुरपतिं भज विश्वनाथम् ॥५॥\\
\\
तेजोमयं सगुणनिर्गुणमद्वितीयम्\\
आनन्दकन्दमपराजितमप्रमेयम्\\
नागात्मकं सकलनिष्कलमात्मरूपं ।\\
वाराणसीपुरपतिं भज विश्वनाथम् ॥६॥\\
\\
रागादिदोषरहितं स्वजनानुरागं\\
वैराग्यशान्तिनिलयं गिरिजासहायम् ।\\
माधुर्यधैर्यसुभगं गरलाभिरामं\\
वाराणसीपुरपतिं भज विश्वनाथम् ॥७॥\\
\\
आशां विहाय परिहृत्य परस्य निन्दां\\
पापे रतिं च सुनिवार्य मनः समाधौ ।\\
आदाय हृत्कमलमध्यगतं परेशं\\
वाराणसीपुरपतिं भज विश्वनाथम् ॥८॥\\
\end{quotation}

\section{\sanskrit द्वादश ज्योतिर्लिङ्ग स्तोत्रम् }
\begin{quotation}\chandas
सौराष्ट्रदेशे विशदेऽतिरम्ये ज्योतिर्मयं चन्द्रकलावतंसम् |\\
भक्तिप्रदानाय कृपावतीर्णं तं सोमनाथं शरणं प्रपद्ये || १||\\
\\
श्रीशैलशृङ्गे विबुधातिसङ्गे तुलाद्रितुङ्गेऽपि मुदा वसन्तम् |\\
तमर्जुनं मल्लिकपूर्वमेकं नमामि संसारसमुद्रसेतुम् || २||\\
\\
अवन्तिकायां विहितावतारं मुक्तिप्रदानाय च सज्जनानाम् |\\
अकालमृत्योः परिरक्षणार्थं वन्दे महाकालमहासुरेशम् || ३||\\
\\
कावेरिकानर्मदयोः पवित्रे समागमे सज्जनतारणाय |\\
सदैवमान्धातृपुरे वसन्तमोङ्कारमीशं शिवमेकमीडे || ४||\\
\\
पूर्वोत्तरे प्रज्वलिकानिधाने सदा वसन्तं गिरिजासमेतम् |\\
सुरासुराराधितपादपद्मं श्रीवैद्यनाथं तमहं नमामि || ५||\\
\\
याम्ये सदङ्गे नगरेऽतिरम्ये विभूषिताङ्गं विविधैश्च भोगैः |\\
सद्भक्तिमुक्तिप्रदमीशमेकं श्रीनागनाथं शरणं प्रपद्ये || ६||\\
\\
महाद्रिपार्श्वे च तटे रमन्तं सम्पूज्यमानं सततं मुनीन्द्रैः |\\
सुरासुरैर्यक्ष महोरगाढ्यैः केदारमीशं शिवमेकमीडे || ७||\\
\\
सह्याद्रिशीर्षे विमले वसन्तं गोदावरितीरपवित्रदेशे |\\
यद्धर्शनात्पातकमाशु नाशं प्रयाति तं त्र्यम्बकमीशमीडे || ८||\\
\\
सुताम्रपर्णीजलराशियोगे निबध्य सेतुं विशिखैरसंख्यैः |\\
श्रीरामचन्द्रेण समर्पितं तं रामेश्वराख्यं नियतं नमामि || ९||\\
\\
यं डाकिनिशाकिनिकासमाजे निषेव्यमाणं पिशिताशनैश्च |\\
सदैव भीमादिपदप्रसिद्दं तं शङ्करं भक्तहितं नमामि || १०||\\
\\
सानन्दमानन्दवने वसन्तमानन्दकन्दं हतपापवृन्दम् |\\
वाराणसीनाथमनाथनाथं श्रीविश्वनाथं शरणं प्रपद्ये || ११||\\
\\
इलापुरे रम्यविशालकेऽस्मिन् समुल्लसन्तं च जगद्वरेण्यम् |\\
वन्दे महोदारतरस्वभावं घृष्णेश्वराख्यं शरणम् प्रपद्ये || १२||\\
\\
ज्योतिर्मयद्वादशलिङ्गकानां शिवात्मनां प्रोक्तमिदं क्रमेण |\\
स्तोत्रं पठित्वा मनुजोऽतिभक्त्या फलं तदालोक्य निजं भजेच्च ||\\
|| इति द्वादश ज्योतिर्लिङ्गस्तोत्रं संपूर्णम् ||\\
\end{quotation}

\section{\sanskrit निर्वाण अष्टकम् }
\begin{quotation}\chandas
मनो बुद्ध्यहंकारचित्तानि नाहम् न च श्रोत्र जिह्वे न च घ्राण नेत्रे \\
न च व्योम भूमिर् न तेजॊ न वायु: चिदानन्द रूप: शिवोऽहम् शिवॊऽहम् ॥\\ 
न च प्राण संज्ञो न वै पञ्चवायु: न वा सप्तधातुर् न वा पञ्चकोश: \\
न वाक्पाणिपादौ न चोपस्थपायू चिदानन्द रूप: शिवोऽहम् शिवॊऽहम् ॥ \\
न मे द्वेष रागौ न मे लोभ मोहौ मदो नैव मे नैव मात्सर्य भाव: \\
न धर्मो न चार्थो न कामो ना मोक्ष: चिदानन्द रूप: शिवोऽहम् शिवॊऽहम् ॥\\ 
न पुण्यं न पापं न सौख्यं न दु:खम् न मन्त्रो न तीर्थं न वेदा: न यज्ञा: \\
अहं भोजनं नैव भोज्यं न भोक्ता चिदानन्द रूप: शिवोऽहम् शिवॊऽहम् ॥\\ 
न मृत्युर् न शंका न मे जातिभेद: पिता नैव मे नैव माता न जन्म \\
न बन्धुर् न मित्रं गुरुर्नैव शिष्य: चिदानन्द रूप: शिवोऽहम् शिवॊऽहम् ॥\\ 
अहं निर्विकल्पॊ निराकार रूपॊ विभुत्वाच्च सर्वत्र सर्वेन्द्रियाणाम् \\
न चासंगतं नैव मुक्तिर् न मेय: चिदानन्द रूप: शिवोऽहम् शिवॊऽहम् ॥\\
\end{quotation}

\section{\sanskrit श्रीशिवाष्टकं }
\begin{quotation}\chandas
प्रभुं प्राणनाथं विभुं विश्वनाथं जगन्नाथनाथं सदानन्दभाजम् ।\\
भवद्भव्यभूतेश्वरं भूतनाथं शिवं शङ्करं शम्भुमीशानमीडे ॥ १॥\\
\\
गले रुण्डमालं तनौ सर्पजालं महाकालकालं गणेशाधिपालम् ।\\
जटाजूटगङ्गोत्तरङ्गैर्विशालं शिवं शङ्करं शम्भुमीशानमीडे ॥ २॥\\
\\
मुदामाकरं मण्डनं मण्डयन्तं महामण्डलं भस्मभूषाधरं तम् ।\\
अनादिह्यपारं महामोहहारं शिवं शङ्करं शम्भुमीशानमीडे ॥ ३॥\\
\\
वटाधोनिवासं महाट्टाट्टहासं महापापनाशं सदासुप्रकाशम् ।\\
गिरीशं गणेशं महेशं सुरेशं शिवं शङ्करं शम्भुमीशानमीडे ॥ ४॥\\
\\
गिरिन्द्रात्मजासंग्रहीतार्धदेहं गिरौ संस्थितं सर्वदा सन्नगेहम् ।\\
परब्रह्मब्रह्मादिभिर्वन्ध्यमानं शिवं शङ्करं शम्भुमीशानमीडे ॥ ५॥\\
\\
कपालं त्रिशूलं कराभ्यां दधानं पदाम्भोजनम्राय कामं ददानम् ।\\
बलीवर्दयानं सुराणां प्रधानं शिवं शङ्करं शम्भुमीशानमीडे ॥ ६॥\\
\\
शरच्चन्द्रगात्रं गुणानन्द पात्रं त्रिनेत्रं पवित्रं धनेशस्य मित्रम् ।\\
अपर्णाकलत्रं चरित्रं विचित्रं शिवं शङ्करं शम्भुमीशानमीडे ॥ ७॥\\
\\
हरं सर्पहारं चिता भूविहारं भवं वेदसारं सदा निर्विकारम् ।\\
श्मशाने वसन्तं मनोजं दहन्तं शिवं शङ्करं शम्भुमीशानमीडे ॥ ८॥\\
\\
स्तवं यः प्रभाते नरः शूलपाणे पठेत् सर्वदा भर्गभावानुरक्तः ।\\
स पुत्रं धनं धान्यमित्रं कलत्रं विचित्रं समासाद्य मोक्षं प्रयाति ॥ ९॥\\
॥ इति शिवाष्टकम् ॥\\
\end{quotation}

\section{\sanskrit दारिद्र्य दहन शिवस्तोत्रम् }
\begin{quotation}\sanskrit
विश्वेश्वराय नरकार्णव तारणाय कर्णामृताय शशिशेखरधारणाय ।\\ 
कर्पूरकान्तिधवलाय जटाधराय दारिद्र्य दुःखदहनाय नमः शिवाय ॥ १॥\\ 
\\
गौरीप्रियाय रजनीशकलाधराय कालान्तकाय भुजगाधिपकङ्कणाय । \\
गंगाधराय गजराजविमर्दनाय दारिद्र्य दुःखदहनाय नमः शिवाय ॥ २॥ \\
\\
भक्तिप्रियाय भवरोगभयापहाय उग्राय दुर्गभवसागरतारणाय । \\
ज्योतिर्मयाय गुणनामसुनृत्यकाय दारिद्र्य दुःखदहनाय नमः शिवाय ॥ ३॥\\ 
\\
चर्मम्बराय शवभस्मविलेपनाय भालेक्षणाय मणिकुण्डलमण्डिताय । \\
मंझीरपादयुगलाय जटाधराय दारिद्र्य दुःखदहनाय नमः शिवाय ॥ ४॥ \\
\\
पञ्चाननाय फणिराजविभूषणाय हेमांशुकाय भुवनत्रयमण्डिताय । \\
आनन्दभूमिवरदाय तमोमयाय दारिद्र्य दुःखदहनाय नमः शिवाय ॥ ५॥\\ 
\\
भानुप्रियाय भवसागरतारणाय कालान्तकाय कमलासनपूजिताय । \\
नेत्रत्रयाय शुभलक्षण लक्षिताय दारिद्र्य दुःखदहनाय नमः शिवाय ॥ ६॥ \\
\\
रामप्रियाय रघुनाथवरप्रदाय नागप्रियाय नरकार्णवतारणाय । \\
पुण्येषु पुण्यभरिताय सुरार्चिताय दारिद्र्य दुःखदहनाय नमः शिवाय ॥ ७॥\\ 
\\
मुक्तेश्वराय फलदाय गणेश्वराय गीतप्रियाय वृषभेश्वरवाहनाय । \\
मातङ्गचर्मवसनाय महेश्वराय दारिद्र्य दुःखदहनाय नमः शिवाय ॥ ८॥\\
\end{quotation}





\chapter{\color{orange}\sanskrit श्री विश्णु स्तोत्रानि }
\section{\color{blue}\sanskrit नारायणकवचं सार्थ}
\sanskrit
ॐ हरिर्विदध्यान् मम सर्व-रक्षां न्यस्ताङ्घ्रि-पद्मः पतगेन्द्र-पृष्ठे\\
दरारि-चर्मासि-गदेषु-चाप-पाशान् दधानोऽष्ट-गुणोऽष्ट-बाहुः ॥ १२॥\\
\\
ॐ - O Lord;  हरिः - the Supreme Personality of Godhead;  विदध्यात् - may He bestow;  मम  my;  सर्व-रक्षाम् - protection from all sides;  न्यस्त - placed;  अङ्घ्रि-पद्मः - whose lotus feet; पतगेन्द्र-पृष्ठे - on the back of GaruDa, the king of all birds; दर - conchshell;  अरि -   disc;  चर्म - shield;  असि - sword;  गदा - club;  इषु - arrows;  चाप - bow;  पाशान् - ropes;  दधानः - holding;  अष्ट - possessing eight;  गुणः - perfections;  अष्ट - eight;  बाहुः - arms.\\
\\
The Supreme Lord, who sits on the back of the bird GaruDa, touching him with His lotus feet, holds eight weapons : the conchshell, disc, shield, sword, club, arrows, bow and ropes. May that Supreme Personality of Godhead protect me at all times with His eight arms. He is all-powerful because He fully possesses the eight mystic powers [aNimA, laghimA, etc.].\\
\\
PURPORT - Thinking oneself one with the Supreme is called aha~NgrahopAsanA. Through aha~NgrahopAsanA one does not become God, but he thinks of himself as qualitatively one with the Supreme. Understanding that as a spirit soul he is equal in quality to the supreme soul the way the water of a river is of the same nature as the water of the sea, one should meditate upon the Supreme Lord, as described in this verse, and seek His protection. The living entities are always subordinate to the Supreme\footnote{To be considered as Godhead {\sanskrit  श्री महाविष्णु}}. Consequently their duty is to always seek the mercy of the Lord in order to be protected by Him in all circumstances.\\
\\
जलेषु मां रक्षतु मत्स्य-मूर्तिर्यादो-गणेभ्यो वरुणस्य पाशात्\\
स्थलेषु मायावटु-वामनोऽव्यात्त्रिविक्रमः खेऽवतु विश्वरूपः ॥ १३ ॥\\
\\
जलेषु - in the water;  माम् - me;  रक्षतु - protect;  मत्स्य-मूर्तिः  the Supreme Lord in the form of a great fish;  यादः-गणेभ्यः - from fierce aquatic animals;  वरुणस्य - of the demigod known as VaruNa;  पाशात् - from the arresting rope; स्थलेषु - on the land;  माया-वटु - the merciful form of the Lord as a dwarf;  वामनः - named VAmanadeva;  अव्यात् – may He protect;  त्रिविक्रमः - Trivikrama, whose three gigantic steps took the three worlds from Bali;  खे - in the sky;  अवतु - may the Lord protect;  विश्वरूपः - the gigantic universal form.\\
\\
May the Lord, who assumes the body of a great fish, protect me in the water from the fierce animals that are associates of the demigod VaruNa. By expanding His illusory energy, the Lord assumed the form of the dwarf VAmana. May VAmana protect me on the land. Since the gigantic form of the Lord, VishvarUpa, conquers the three worlds, may He protect me in the sky.\\
\\
PURPORT - This mantra seeks the protection of the Supreme Personality of Godhead in the water, land and sky in His incarnations as the fish, VAmanadeva and the VishvarUpa.\\
\\
दुर्गेष्वटव्याजि-मुखादिषु प्रभुः पायान् नृसिंहोऽसुर-यूथपारिः\\
विमुञ्चतो यस्य महाट्ट-हासं दिशो विनेदुर्यपतंश्च गर्भाः ॥ १४॥\\
\\
दुर्गेषु - in places where travel is very difficult; अटवि - in the dense forest;  आजि-मुख-आदिषु - on the war front and so on;  प्रभुः - the Supreme Lord;  पायात् - may He protect;  नृसिंहः  Lord NRisiMhadeva;  असुर-यूथप - of HiraNyakashipu, the leader of the demons;  अरिः - the enemy; विमुञ्चतः - releasing;  यस्य - of whom;  महा-अट्ट-हासम् - great and fearful laughing;  दिशः - all the directions;  विनेदुः - resounded through;  न्यपतन् - fell down;  च - and;  गर्भाः - the embryos of the wives of the demons.\\
\\
May Lord NRisiMhadeva, who appeared as the enemy of HiraNyakashipu, protect me in all directions. His loud laughing vibrated in all directions and caused the pregnant wives of the asuras to have miscarriages. May that Lord be kind enough to protect me in difficult places like the forest and battlefront.\\
\\ 
रक्षत्वसौ माध्वनि यज्ञ-कल्पः स्व-दंष्ट्रयोन्नीत-धरो वराहः\\
रामोऽद्रि-कूटेष्वथ विप्रवासे सलक्ष्मणोऽव्याद् भरताग्रजोऽस्मान् ॥ १५॥\\
\\
रक्षतु - may the Lord protect;  असौ - that;  मा - me; अध्वनि - on the street;  यज्ञ-कल्पः - who is ascertained by performance of ritualistic ceremonies;  स्व-दंष्ट्रया – by His own tusk;  उन्नीत - raising;  धरः - the planet earth; वराहः - Lord Boar;  रामः - Lord RAma;  अद्रि-कूटेषु - on the summits of the mountains;  अथ - then;  विप्रवासे – in foreign countries;  स-लक्ष्मणः -  with His brother LakShmaNa; अव्यात् - may He protect;  भरत-अग्रजः - the elder brother of MahArAja Bharata;  अस्मान् - us.\\
\\
The Supreme indestructible Lord is ascertained through the performance of ritualistic sacrifices and is therefore known as Yaj~neshvara. In His incarnation as Lord Boar, He raised the planet earth from the water at the bottom of the universe and kept it on His pointed tusks. May that Lord protect me from rogues on the street. May ParashurAma protect me on the tops of mountains, and may the elder brother of Bharata, Lord RAmacandra, along with His brother LakShmaNa, protect me in foreign countries.\\
\\
PURPORT - There are three RAmas. One RAma is ParashurAma (JAmadAgnya). Another RAma is Lord RAmacandra, and a third RAma is Lord BalarAma. In this verse the words rAmo.adri-kUTeShvatha indicate Lord ParashurAma. The brother of Bharata MahArAja and LakShmaNa is Lord RAmacandra.\\
\\ 
माम् उग्र-धर्माद् अखिलात् प्रमादान्नारायणः पातु नरश्च हासात्\\
दत्तस्त्वयोगाद् अथ योग-नाथः पायाद् गुणेशः कपिलः कर्म-बन्धात् ॥ १६॥\\
\\
माम् - me;  उग्र-धर्मात् - from unnecessary religious principles;  अखिलात् - from all kinds of activities; प्रमादात् - which are enacted in madness;  नारायणः – Lord NArAyaNa;  पातु - may He protect;  नरः च - and Nara; हासात् - from unnecessary pride;  दत्तः - DattAtreya;  तु - of course;  अयोगात् - from the path of false yoga;  अथ  indeed;  योग-नाथः - the master of all mystic powers;  पायात् – may He protect;  गुण-ईशः - the master of all spiritual qualities;  कपिलः - Lord Kapila;  कर्म-बन्धात् - from the bondage of fruitive activities.\\
\\
May Lord NArAyaNa protect me from unnecessarily following false religious systems and falling from my duties due to madness. May the Lord in His appearance as Nara protect me from unnecessary pride. May Lord DattAtreya, the master of all mystic power, protect me from falling while performing bhakti-yoga, and may Lord Kapila, the master of all good qualities, protect me from the material bondage of fruitive activities.\\
\\
सनत्-कुमारोऽवतु कामदेवाद्धयशीर्षा मां पथि देव-हेलनात्\\
देवर्षि-वर्यः पुरुषार्चनान्तरात्कूर्मो हरिर्मां निरयाद् अशेषात् ॥ १७॥\\
\\
सनत्-कुमारः - the great brahmacArI named Sanat-kumAra;  अवतु  may he protect;  काम-देवात् - from the hands of Cupid or lusty desire; हय-शीर्षा - Lord HayagrIva, the incarnation of the Lord whose head is like that of a horse;  माम् - me;  पथि - on the path;  देव-हेलनात् - from neglecting to offer respectful obeisances to brAhmaNas, VaiShNavas and the Supreme Lord;  देवर्षि-वर्यः - the best of the saintly sages, NArada;  पुरुष-अर्चन-अन्तरात् - from the offenses in worshiping the Deity;  कूर्मः - Lord KUrma, the tortoise;  हरिः - the Supreme Personality of Godhead;  माम् -  me;  निरयात् - from hell;  अशेषात् - unlimited.\\
\\
May Sanat-kumAra protect me from lusty desires. As I begin some auspicious activity, may Lord HayagrIva protect me from being an offender by neglecting to offer respectful obeisances to the Supreme Lord. May DevarShi NArada protect me from committing offenses in worshiping the Deity, and may Lord KUrma, the tortoise, protect me from falling to the unlimited hellish planets.\\
\\
PURPORT - Lusty desires are very strong in everyone, and they are the greatest impediment to the discharge of devotional service. Therefore those who are very much influenced by lusty desires are advised to take shelter of Sanat-kumAra, the great brahmacArI devotee. NArada Muni, who is the guide for arcana, is the author of the NArada-pa~ncharAtra, which prescribes the regulative principles for worshiping the Deity. Everyone engaged in Deity worship, whether at home or in the temple, should always seek the mercy of DevarShi NArada in order to avoid the thirty-two offenses while worshiping the Deity. These offenses in Deity worship are mentioned in The Nectar of Devotion.\\
\\
धन्वन्तरिर्भगवान् पात्वपथ्याद्द्वन्द्वाद् भयाद् ऋषभो निर्जितात्मा\\
यज्ञश्च लोकाद् अवताज् जनान्ताद्बलो गणात् क्रोध-वशाद् अहीन्द्रः ॥ १८॥\\
\\
धन्वन्तरिः - the incarnation Dhanvantari, the physician;  भगवान् - the Supreme Personality of Godhead;  पातु - may He protect me;  अपथ्यात् - from things injurious to the health, such as meat and intoxicants; द्वन्द्वात् - from duality;  भयात् - from fear;  ऋषभः – Lord RiShabhadeva;  निर्जित-आत्मा - who fully controlled his mind and self;  यज्ञः - Yaj~na;  च - and;  लोकात् - from the defamation of the populace;  अवतात् - may He protect;  जन-अन्तात्  from dangerous positions created by other people;  बलः - Lord BalarAma; गणात् - from the hordes of;  क्रोध-वशात् - the angry serpents; अहीन्द्रः - Lord BalarAma in the form of the serpent sheSha NAga.\\
\\
May the Supreme Personality of Godhead in His incarnation as Dhanvantari relieve me from undesirable eatables and protect me from physical illness. May Lord Shabhadeva, who conquered His inner and outer senses, protect me from fear produced by the duality of heat and cold. May Yaj~na protect me from defamation and harm from the populace, and may Lord BalarAma as sheSha protect me from envious serpents.\\
\\
PURPORT - To live within this material world, one must face many dangers, as described herein. For example, undesirable food poses a danger to health, and therefore one must give up such food. The Dhanvantari incarnation can protect us in this regard. Since Lord ViShNu is the Supersoul of all living entities, if He likes He can save us from adhibhautika disturbances, disturbances from other living entities. Lord BalarAma is the sheSha incarnation, and therefore He can save us from angry serpents or envious persons, who are always ready to attack.\\
\\
द्वैपायनो भगवान् अप्रबोधाद्बुद्धस् तु पाषण्ड-गण-प्रमादात्\\
कल्किः कलेः काल-मलात् प्रपातुधर्मावनायोरु-कृतावतारः ॥ १९॥\\
\\
द्वैपायनः - shrIla VyAsadeva, the giver of all Vedic knowledge;  भगवान् - the most powerful incarnation of the Supreme Personality of Godhead;  अप्रबोधात् - from ignorance of the shAstra;  बुद्धः तु - also Lord Buddha;  पाषण्ड-गण - of atheists creating disillusionment for innocent persons;  प्रमादात् - from the madness;  कल्किः - Lord Kalki, the incarnation of Keshava;  कलेः – of this Kali-yuga;  काल-मलात् - from the darkness of the age; प्रपातु - may He protect;  धर्म-अवनाय - for the protection of religious principles;  उरु - very great;  कृत-अवतारः - who took an incarnation.\\
\\
May the Personality of Godhead in His incarnation as VyAsadeva protect me from all kinds of ignorance resulting from the absence of Vedic knowledge. May Lord Buddhadeva protect me from activities opposed to Vedic principles and from laziness that causes one to madly forget the Vedic principles of knowledge and ritualistic action. May Kalkideva, the Supreme Personality of Godhead, who appeared as an incarnation to protect religious principles, protect me from the dirt of the age of Kali.\\
\\
PURPORT - This verse mentions various incarnations of the Supreme Personality of Godhead who appear for various purposes. shrIla VyAsadeva, MahAmuni, compiled the Vedic literature for the benefit of all human society. If one wants to be protected from the reactions of ignorance even in this age of Kali, one may consult the books left by shrIla VyAsadeva, UpaniShads, VedAnta-sUtra (Brahma-sUtra), MahAbhArata, shrImad-BhAgavatam MahA-PurANa (VyAsadeva.as commentary on the Brahma-sUtra) and the other seventeen PurANas. Only by the mercy of shrIla VyAsadeva do we have so many volumes of transcendental knowledge to save us from the clutches of ignorance.\\
As described by shrIla Jayadeva GosvAmI in his DashAvatAra-stotra, Lord Buddha apparently decried the Vedic knowledge:  \\
\\
निन्दसि यज्ञ-विधेर् अहह श्रुति-जातं\\
सदय-हृदय-दर्शित-पशु-घातम्\\
केशव धृत-बुद्ध-शरीर जय जगद्-ईश हरे\\
\\
The mission of Lord Buddha was to save people from the abominable activity of animal killing and to save the poor animals from being unnecessarily killed. When pAShaNDIs were cheating by killing animals on the plea of sacrificing them in Vedic yaj~nas, the Lord said, ᳚If the Vedic injunctions allow animal killing, I do not accept the Vedic principles.᳚ Thus he actually saved people who acted according to Vedic principles. One should therefore surrender to Lord Buddha so that he can help one avoid misusing the injunctions of the Vedas.
The Kalki avatAra is the fierce incarnation who vanquishes the class of the atheists born in this age of Kali. Now, in the beginning of Kali-yuga, many irreligious principles are in effect, and as Kali-yuga advances, many pseudo religious principles will certainly be introduced, and people will forget the real religious principles enunciated by Lord KRiShNa before the beginning of Kali-yuga, namely principles of surrender unto the lotus feet of the Lord. Unfortunately, because of Kali-yuga, foolish people do not surrender to the lotus feet of KRiShNa. Even most people who claim to belong to the Vedic system of religion are actually opposed to the Vedic principles. Every day they manufacture a new type of dharma on the plea that whatever one manufactures is also a path of liberation. Atheistic men generally say, yata mata tata patha. According to this view, there are hundreds and thousands of different opinions in human society, and each opinion is a valid religious principle. This philosophy of rascals has killed the religious principles mentioned in the Vedas, and such philosophies will become increasingly influential as Kali-yuga progresses. In the last stage of Kali-yuga, Kalkideva, the fierce incarnation of Keshava, will descend to kill all the atheists and will save only the devotees of the Lord.\\
\\ 
मां केशवो गदया प्रातर् अव्याद्गोविन्द आसङ्गवम् आत्त-वेणुः\\
नारायणः प्राह्ण उदात्त-शक्तिर्मध्यन्-दिने विष्णुररीन्द्र-पाणिः ॥ २०॥\\
\\
माम् - me;  केशवः - Lord Keshava;  गदया - by His club;  प्रातः - in the morning hours;  अव्यात् - may He protect;  गोविन्दः - Lord Govinda; आसङ्गवम् - during the second part of the day;  आत्त-वेणुः - holding His flute;  नारायणः - Lord NArAyaNa with four hands;  प्राह्णः - during the third part of the day;  उदात्त-शक्तिः -  controlling different types of potencies;  मध्यम्-दिने - during the fourth part of the day;  विष्णुः - Lord ViShNu; अरीन्द्र-पाणिः - bearing the disc in His hand to kill the enemies.\\
\\
May Lord Keshava protect me with His club in the first portion of the day, and may Govinda, who is always engaged in playing His flute, protect me in the second portion of the day. May Lord NArAyaNa, who is equipped with all potencies, protect me in the third part of the day, and may Lord ViShNu, who carries a disc to kill His enemies, protect me in the fourth part of the day.\\
\\
PURPORT - According to Vedic astronomical calculations, day and night are each divided into thirty ghaTikAs (twenty-four minutes), instead of twelve hours. Generally, each day and each night is divided into six parts consisting of five ghaTikAs. In each of these six portions of the day and night, the Lord may be addressed for protection according to different names. Lord Keshava, the proprietor of the holy place of MathurA, is the Lord of the first portion of the day, and Govinda, the Lord of VRindAvana. is the master of the second portion.\\
\\ 
देवोऽपराह्णे मधु-होग्रधन्वा सायं त्रि-धामावतु माधवो माम्\\
दोषे हृषीकेश उतार्ध-रात्रे निशीथ एकोऽवतु पद्मनाभः ॥ २१॥\\
\\
देवः - the Lord;  अपराह्णे - in the fifth part of the day;  मधु-हा - named MadhusUdana;  उग्र-धन्वा - bearing the very fearful bow known as shAr~Nga;  सायम् - the sixth part of the day;  त्रि-धामा - manifesting as the three deities BrahmA, ViShNu and Maheshvara;  अवतु - may He protect;  माधवः - named MAdhava;  माम् - me;  दोषे - during the first portion of the night;  हृषीकेशः - Lord HRiShIkesha;  उत - also; अर्ध-रात्रे - during the second part of the night;  निशीथे - during the third part of the night;  एकः - alone;  अवतु -  may He protect;  पद्मनाभः - Lord PadmanAbha.\\
\\
May Lord MadhusUdana, who carries a bow very fearful for the demons, protect me during the fifth part of the day. In the evening, may Lord MAdhava, appearing as BrahmA, ViShNu and Maheshvara, protect me, and in the beginning of night may Lord HRiShIkesha protect me. At the dead of night [in the second and third parts of night] may Lord PadmanAbha alone protect me.\\
\\
श्रीवत्स-धामापर-रात्र ईशः प्रत्यूष ईशोऽसि-धरो जनार्दनः\\
दामोदरोऽव्यादनुसन्ध्यं प्रभाते विश्वेश्वरो भगवान् काल-मूर्तिः ॥ २२॥\\
\\
श्रीवत्स-धामा - the Lord, on whose chest the mark of shrIvatsa is resting;  अपर-रात्रे - in the fourth part of the night;  ईशः - the Supreme Lord;  प्रत्यूषे - in the end of the night;  ईशः - the Supreme Lord;  असि-धरः -  carrying a sword in the hand;  जनार्दनः - Lord JanArdana;  दामोदरः - Lord DAmodara;  अव्यात् - may He protect;  अनुसन्ध्यम् - during each junction or twilight;  प्रभाते - in the early morning (the sixth part of the night);  विश्व-ईश्वरः - the Lord of the whole universe; भगवान् - the Supreme Personality of Godhead;  काल-मूर्तिः - the personification of time.\\
\\
May the Supreme Personality of Godhead, who bears the shrIvatsa on His chest, protect me after midnight until the sky becomes pinkish. May Lord JanArdana, who carries a sword in His hand, protect me at the end of night [during the last four ghaTikAs of night]. May Lord DAmodara protect me in the early morning, and may Lord Vishveshvara protect me during the junctions of day and night.\\
\\
चक्रं युगान्तानल-तिग्म-नेमि भ्रमत् समन्ताद् भगवत्-प्रयुक्तम्\\
दन्दग्धि दन्दग्ध्य् अरि-सैन्यम् आशु कक्षं यथा वात-सखो हुताशः ॥ २३॥\\
\\
चक्रम् – the disc of the Lord;  युग-अन्त - at the end of the millennium;  अनल -  like the fire of devastation;  तिग्म-नेमि - with a sharp rim; भ्रमत् - wandering;  समन्तात् - on all sides;  भगवत्-प्रयुक्तम् - being engaged by the Lord;  दन्दग्धि दन्दग्धि - please burn completely, please burn completely;  अरि-सैन्यम् - the army of our enemies;  आशु - immediately;  कक्षम् - dry grass;  यथा - like;  वात-सखः - the friend of the wind;  हुताशः - blazing fire.\\
\\
Set into motion by the Supreme Personality of Godhead and wandering in all the four directions, the disc of the Supreme Lord has sharp edges as destructive as the fire of devastation at the end of the millennium. As a blazing fire burns dry grass to ashes with the assistance of the breeze, may that Sudarshana cakra burn our enemies to ashes.\\
\\ 
गदेऽशनि-स्पर्शन-विस्फुलिङ्गे निष्पिण्ढि निष्पिण्ढ्य् अजित-प्रियासि\\
कुष्माण्ड-वैनायक-यक्ष-रक्षो-भूत-ग्रहांश्चूर्णय चूर्णयारीन् ॥ २४॥\\
\\
गदे - O club in the hands of the Supreme Personality of Godhead;  अशनि - like thunderbolts; स्पर्शन -  whose touch;  विस्फुलिङ्गे - giving off sparks of fire;  निष्पिण्ढि निष्पिण्ढि  pound to pieces, pound to pieces;  अजित-प्रिया - very dear to the Supreme Personality of Godhead; असि - you are;  कुष्माण्ड - imps named KuShmANDas;  वैनायक -  ghosts named VainAyakas;  यक्ष - ghosts named YakShas; रक्षः - ghosts named RAkShasas;  भूत - ghosts named BhUtas;  ग्रहान् - and evil demons named Grahas;  चूर्णय - pulverize; चूर्णय -  pulverize;  अरीन् - my enemies.\\
\\
O club in the hand of the Supreme Personality of Godhead, you produce sparks of fire as powerful as thunderbolts, and you are extremely dear to the Lord. I am also His servant. Therefore kindly help me pound to pieces the evil living beings known as KuShmANDas, VainAyakas, YakShas, RAkShasas, BhUtas and Grahas. Please pulverize them.\\
\\
त्वं\\
यातुधान-प्रमथ-प्रेत-मातृ-पिशाच-विप्रग्रह-घोर-दृष्टीन्\\
दरेन्द्र विद्रावय कृष्ण-पूरितो भीम-स्वनोऽरेर्हृदयानि कम्पयन् ॥ २५॥\\
\\
त्वम् - you;  यातुधान - RAkShasas;  प्रमथ - Pramathas;  प्रेत - Pretas;  मातृ - MAtAs;  पिशाच - PishAcas;  विप्र-ग्रह -  brAhmaNa ghosts; घोर-दृष्टीन् - who have very fearful eyes;  दरेन्द्र – O PA~ncajanya, the conchshell in the hands of the Lord;  विद्रावय - drive away;  कृष्ण-पूरितः - being filled with air from the mouth of KRiShNa;  भीम-स्वनः - sounding extremely fearful;  अरेः - of the enemy;  हृदयानि - the cores of the hearts; कम्पयन् - causing to tremble.  O best of conchshells, O PA~ncajanya in the hands of the Lord, you are always filled with the breath of Lord KRiShNa. Therefore you create a fearful sound vibration that causes trembling in the hearts of enemies like the RAkShasas, pramatha ghosts, Pretas, MAtAs, PishAcas and brAhmaNa ghosts with fearful eyes.\\
\\ 
त्वं तिग्म-धारासि-वरारि-सैन्यम् ईश-प्रयुक्तो मम चिन्धि चिन्धि\\
चक्षूंषि चर्मञ् चत-चन्द्र चादय द्विषाम्\\
अघोनां हर पाप-चक्षुषाम् ॥ २६॥\\
\\
त्वम् - you;  तिग्म-धार-असि-वर - O best of swords possessing very sharp blades;  अरि-सैन्यम् - the soldiers of the enemy;  ईश-प्रयुक्तः - being engaged by the Supreme Personality of Godhead;  मम - my;  चिन्धि चिन्धि -  chop to pieces, chop to pieces;  चक्षूंषि - the eyes; चर्मन् - O shield;  शत-चन्द्र - possessing brilliant circles like a hundred moons;  चादय - please cover;  द्विषाम् – of those who are envious of me;  अघोनाम् - who are completely sinful;  हर - please take away;  पाप-चक्षुषाम् - of those whose eyes are very sinful.  O king of sharp-edged swords, you are engaged by the Supreme Personality of Godhead. Please cut the soldiers of my enemies to pieces. Please cut them to pieces! O shield marked with a hundred brilliant moonlike circles, please cover the eyes of the sinful enemies. Pluck out their sinful eyes.\\
\\
यन् नो भयं ग्रहेभ्योऽभूत्केतुभ्यो नृभ्य एव च\\
सरीसृपेभ्यो दंष्ट्रिभ्यो भूतेभ्योऽंहोभ्य एव च ॥ २७॥\\
\\
सर्वाण्य् एतानि भगवन्-नाम-रूपानुकीर्तनात्\\
प्रयान्तु सङ्क्षयं सद्यो ये नः श्रेयः-प्रतीपकाः ॥ २८॥\\
\\
यत् - which;  नः - our;  भयम् - fear;  ग्रहेभ्यः – from the Graha demons;  अभूत् - was;  केतुभ्यः - from meteors, or falling stars;  नृभ्यः - from envious human beings;  एव च - also;  सरीसृपेभ्यः - from snakes or scorpions;  दंष्ट्रिभ्यः - from animals with fierce teeth like tigers, wolves and boars; भूतेभ्यः  - from ghosts or the material elements (earth, water, fire, etc.);  अंहोभ्यः - from sinful activities;  एव च - as well as; सर्वाणि एतानि - all these;  भगवत्-नाम-रूप-अनुकीर्तनात् - by glorifying the transcendental form, name, attributes and paraphernalia of the Supreme Personality of Godhead;  प्रयान्तु - let them go; सङ्क्षयम् - to complete destruction;  सद्यः - immediately; ये - which;  नः - our;  श्रेयः-प्रतीपकाः - hindrances to well-being.\\
\\
May the glorification of the transcendental name, form, qualities and paraphernalia of the Supreme Personality of Godhead protect us from the influence of bad planets, meteors, envious human beings, serpents, scorpions, and animals like tigers and wolves. May it protect us from ghosts and the material elements like earth, water, fire and air, and may it also protect us from lightning and our past sins. We are always afraid of these hindrances to our auspicious life. Therefore, may they all be completely destroyed by the chanting of the Hare KRiShNa mahA-mantra.\\
\\
गरुडो भगवान् स्तोत्र-स्तोभश्चन्दोमयः प्रभुः\\
रक्षत्वशेष-कृच्च्रेभ्यो विष्वक्सेनः स्व-नामभिः ॥ २९॥\\
\\
गरुडः - His Holiness GaruDa, the carrier of Lord ViShNu;  भगवान् - as powerful as the Supreme Personality of Godhead;  स्तोत्र-स्तोभः - who is glorified by selected verses and songs;  चन्दः-मयः - the personified Vedas;  प्रभुः -  the lord;  रक्षतु - may He protect;  अशेष-कृच्च्रेभ्यः - from unlimited miseries;  विष्वक्सेनः - Lord ViShvaksena; स्व-नामभिः - by His holy names.\\
\\
Lord GaruDa, the carrier of Lord ViShNu, is the most worshipable lord, for he is as powerful as the Supreme Lord Himself. He is the personified Vedas and is worshiped by selected verses. May he protect us from all dangerous conditions, and may Lord ViShvaksena, the Personality of Godhead, also protect us from all dangers by His holy names.\\
\\
सर्वापद्भ्यो हरेर्नाम-रूप-यानायुधानि नः\\
बुद्धीन्द्रिय-मनः-प्राणान्पान्तु पार्षद-भूषणाः ॥ ३०॥\\
\\
सर्व-आपद्भ्यः - from all kinds of danger;  हरेः - of the Supreme Personality of Godhead;  नाम - the holy name; रूप - the transcendental form;  यान - the carriers;  आयुधानि - and all the weapons;  नः - our;  बुद्धि -  intelligence; इन्द्रिय - senses;  मनः - mind;  प्राणान् - life air; पान्तु - may they protect and maintain;  पार्षद-भूषणाः - the decorations who are personal associates.\\
\\
May the Supreme Personality of Godhead.as holy names, His transcendental forms, His carriers and all the weapons decorating Him as personal associates protect our intelligence, senses, mind and life air from all dangers.\\
\\
PURPORT - There are various associates of the transcendental Personality of Godhead, and His weapons and carrier are among them. In the spiritual world, nothing is material. The sword, bow, club, disc and everything decorating the personal body of the Lord are spiritual living force. Therefore the Lord is called advaya j~nAna, indicating that there is no difference between Him and His names, forms, qualities, weapons and so on. Anything pertaining to Him is in the same category of spiritual existence. They are all engaged in the service of the Lord in varieties of spiritual forms.\\
\\
यथा हि भगवान् एव वस्तुतः सद् असच्च यत्\\
सत्येनानेन नः सर्वे यान्तु नाशम् उपद्रवाः ॥ ३१॥\\
\\
यथा - just as;  हि - indeed;  भगवान् - the Supreme Personality of Godhead;  एव - undoubtedly;  वस्तुतः - at the ultimate issue;  सत् - manifested;  असत् - unmanifested;  च - and;  यत् - whatever;  सत्येन - by the truth;  अनेन -  this;  नः - our;  सर्वे - all;  यान्तु - let them go; नाशम् - to annihilation;  उपद्रवाः - disturbances.\\
\\
The subtle and gross cosmic manifestation is material, but nevertheless it is nondifferent from the Supreme Personality of Godhead because He is ultimately the cause of all causes. Cause and effect are factually one because the cause is present in the effect. Therefore the Absolute Truth, the Supreme Personality of Godhead, can destroy all our dangers by any of His potent parts.\\
\\
यथैकात्म्यानुभावानां विकल्प-रहितः स्वयम्\\
भूषणायुध-लिङ्गाख्या धत्ते शक्तीः स्व-मायया ॥ ३२॥\\
\\
तेनैव सत्य-मानेन सर्व-ज्ञो भगवान् हरिः\\
पातु सर्वैः स्वरूपैर्नः सदा सर्वत्र सर्व-गः ॥ ३३॥\\
\\
यथा - just as;  ऐकात्म्य - in terms of oneness manifested in varieties;  अनुभावानाम् - of those thinking;  विकल्प-रहितः -  the absence of difference;  स्वयम् - Himself;  भूषण - decorations;  आयुध - weapons;  लिङ्ग-आख्याः – characteristics and different names;  धत्ते - possesses;  शक्तीः – potencies like wealth, influence, power, knowledge, beauty and renunciation; स्व-मायया - by expanding His spiritual energy;  तेन एव - by that;  सत्य-मानेन - true understanding;  सर्व-ज्ञः - omniscient;  भगवान् - the Supreme Personality of Godhead; हरिः - who can take away all the illusion of the living entities;  पातु - may He protect;  सर्वैः - with all;  स्व-रूपैः - His forms;  नः - us;  सदा - always;  सर्वत्र - everywhere;  सर्व-गः - who is all-pervasive.\\
\\
The Supreme Personality of Godhead, the living entities, the material energy, the spiritual energy and the entire creation are all individual substances. In the ultimate analysis, however, together they constitute the supreme one, the Personality of Godhead. Therefore those who are advanced in spiritual knowledge see unity in diversity. For such advanced persons, the Lord.as bodily decorations, His name, His fame, His attributes and forms and the weapons in His hand are manifestations of the strength of His potency. According to their elevated spiritual understanding, the omniscient Lord, who manifests various forms, is present everywhere. May He always protect us everywhere from all calamities.\\
\\
PURPORT - A person highly elevated in spiritual knowledge knows that nothing exists but the Supreme Personality of Godhead. This is also confirmed in Bhagavad-gItA (9.4) where Lord KRiShNa says, mayA tatam idaM sarvam, indicating that everything we see is an expansion of His energy. This is confirmed in the ViShNu PurANa (1.22.52):\\

\section{\sanskrit द्वादश स्तुति}
\subsection{\sanskrit प्रथमो अध्याय}
\sanskrit
वन्दे वन्द्यं सदानन्दं वासुदेवं निरञ्जनम् ।\\
इंदिरापतिमाद्यादिवरदेशवरप्रदम् ॥ 01 ॥\\
\\
नमामि निखिलाधीशकिरीटाघृष्ट पीठवत् ।\\
हृत्तमःशमनेऽर्क्काभं श्रीपतेः पादपङ्कजम् ॥ 02 ॥\\
\\
जाम्बूनदाम्बराधारं नितम्बं चिन्त्यमीशितुः ।\\
स्वर्ण्णमञ्जीरसंवीतमारूढं जगदम्भया ॥ 03 ॥\\
\\
उदरं चिन्त्यमीशस्य तनुत्वेऽप्यखिलम्भरम्।\\
वलित्रयाङ्कितं नित्यमुपगूढं श्रियैकया ॥ 04 ॥\\
\\
 स्मरणीयमुरो विष्णोरिन्दिरावासमुत्तमैः ।\\
अनन्तमन्तवदिव भुजयोरन्तरं गतम् ॥ 05 ॥\\
\\
 शङ्खचक्रगदापद्मधराश्चिन्त्या हरेर्भुजाः ।\\
पीनवृत्ता जगद्रक्षाकेवलोद्योगिनोऽनिशम् ॥ 06 ॥\\
\\
सन्ततं चिन्तयेत् कण्ठं भास्वत् कौस्तुभभासकम् ।\\
वैकुण्ठस्याखिला वेदा उद्गीर्यन्तेऽनिशं यतः ॥ 07 ॥\\
\\
स्मरेत यामिनीनाथसहस्रामितकान्तिमत् ।\\
भवतापापनोदीड्यं श्रीपतेर्मुखपङ्कजम् ॥ 08 ॥\\
\\
पूर्ण्णानन्य सुखोद्भासि मन्दस्मितमधीशितुः ।\\
गोविन्दस्य सदा चिन्त्यं नित्यानन्दपदप्रदम् ॥ 09 ॥\\
\\
 स्मरामि भवसन्ताप हानिदामृतसागरम् ।\\
पूर्ण्णानन्दस्य रामस्य सानुरागावलोकनम्  ॥ 10 ॥\\
\\
द्ध्यायेदजस्रमीशस्य पद्मजादिप्रतीक्षितम् ।\\
भ्रूभङ्गं पारमेष्ठ्यादि पददायि विमुक्तिदम्  ॥ 11 ॥\\
\\
सन्ततं चिन्तयेनं तमन्तकाले विशेषतः ।\\
नैवोदापुर्ग्गृणन्तोऽन्तं यद् गुणानामजादयः ॥ 12 ॥\\
\\
\subsection{\sanskrit द्वितियो अध्याय}
\sanskrit
स्वजनोदधिसंवृद्धिपूर्ण्णचन्द्रो गुणार्ण्णवः\\
अमन्दानन्दसान्द्रो नः सदाऽव्यादिन्दिरापतिः ॥ 01 ॥\\
\\
रमाचकोरीविधवे दुष्टसर्प्पोदवह्नये ।\\
सत् पान्थजनगेहाय नमो नारायणाय ते ॥ 02 ॥\\
\\
चिदचिद् भेदमखिलं विधायाऽधाय भुञ्जते ।\\
अव्याकृतगृहस्थाय रमाप्रणयिने नमः ॥ 03 ॥\\
\\
अमन्दगुणसारोऽपि मन्दहासेन वीक्षितः ।\\
नित्यमिन्दिरायाऽऽनन्दसान्द्रो यो नौमि तं हरिम् ॥ 04 ॥\\
\\
वशी वशो न कस्यापि योऽजितो विजिताखिलः ।\\
सर्वकर्त्ता न क्रियते तं नमामि रमापतिम् ॥ 05 ॥\\
\\
अगुणाय गुणोद्रेकस्वरूपायाऽदिकारिणे ।\\
विदारितारिसङ्घाय वासुदेवाय ते नमः ॥ 06 ॥\\
\\
आदिदेवाय देवानां पतये सादितारये ।\\
अनाद्यज्ञानपाराय नमः पारावराश्रय ॥ 07 ॥\\
\\
अजाय जनयित्रेऽस्य विजिताखिलदानव ।\\
अजादिपूज्यपादाय नमस्ते गरुडध्वज ॥ 08 ॥\\
\\
रमारमण एवैको रणजिच्छरणं सताम् ।\\
कारणं कारणस्यापि तरुणादित्यसप्रभः ॥ 09 ॥\\
\\
 इन्दिरामन्दसान्द्राग्र्यकटाक्षप्रेक्षितात्मने ।\\
अस्मदिष्टैककार्याय पूर्ण्णाय हरये नमः ॥ 10 ॥\\
\\
 एवंविधः परो विष्णुरव्याच्छ्रीपुरुषोत्तमः ।\\
तमहं सर्वदा वन्दे श्रीनिकेतं परं हरिम् ॥ 11 ॥\\
\\
\subsection{\sanskrit त्रुतियो अध्याय}
\sanskrit
शङ्खचक्रगदापद्मशार्ङ्गखड्गधरं सदा ।\\
नमामि पितरं नित्य वासुदेवं जगत् पतिम् ॥ 01 ॥\\
\\
सच्चिदानन्दरूपं तमनामयमनन्तरम् ।\\
एकमेकान्तमभयं विष्णुं विश्वहृदि स्थितम् ॥ 02 ॥\\
\\
नानास्वभावमत्यन्तस्वभावेन विचारितम् ।\\
अविशेषमनाद्यन्तं प्रणमामि सनातनम् ॥ 03 ॥\\
\\
कुन्देन्दुसन्निभं वन्दे सुधारसनिभं विभुम् ।\\
ज्ञानमुद्रापुस्तकारिशङ्खाक्षायुधधारिणम् ॥ 04 ॥\\
\\
आनन्दमजरं नित्यमात्मेशममितद्युतिम् ।\\
वन्दे विश्वस्य पितरं विष्णुं विश्वेश्वरं सदा ॥ 05 ॥\\
\\
अमन्दानन्दसन्दोहसन्तोषितजगत्रयम् ।\\
नारायणमणीयांसं वन्दे देवं सदातनम् ॥ 06 ॥\\
\\
सूर्यमण्डलमद्ध्यस्थं वराभयकरोद्यतम् ।\\
सूर्यामितद्युतिं वन्दे नारायणमनामयम् ॥ 07 ॥\\
\\
विश्वं विश्वाकरं वन्दे विश्वस्य प्रपितामहम् ।\\
नानारूपमजं नित्यं विश्वात्मानं प्रजापतिम् ॥ 08 ॥\\
\\
कालाकालविचारादिप्रकालितसदातनम् ।\\
पुरुषं प्रकृतिस्थं च वन्दे विष्णुमजोत्तमम् ॥ 09 ॥\\
\\
ब्रह्मणः पितरं वन्दे शङ्करस्य पितामहम् ।\\
श्रियः पतिमजं नित्यमिन्द्रादिप्रपितामहम् ॥ 10 ॥\\
\\
प्रतिप्रति स्थितं विष्णुं नित्यामृतमनामयम् ।\\
शङ्कचक्रधरं वन्दे वराभयकरोद्यतम् ॥ 11 ॥\\
\\
प्रतापप्रविशेषेण दूरीकृतविदूषणम् ।\\
विदारितारिमत्यन्तप्रसन्नं प्रणमाम्यहम् ॥ 12 ॥\\
\\
अप्रमेयमजं नित्यं विशालममृतं परम् ।\\
कृतनित्यालयं लोके नमस्यामि जगत् पतिम् ॥ 13 ॥\\
\\
सुनित्यसुखमक्षय्यं शुद्धं शान्तं निरञ्जनम् ।\\
लोकालोकविचाराढ्यं नमस्यामि श्रियःपतिम् ॥ 14 ॥\\
\\
अमन्दानन्दसन्दोहसान्द्रमिन्द्रानुजं परम् ।\\
नित्यावदातमेकान्तं प्रमाणातीतमक्षयम् ।\\
लोकालोकपतिं दिव्यं नमस्यामि रमापतिम् ॥ 15 ॥\\
\\
\subsection{\sanskrit चतुर्थो अध्याय}
\sanskrit
कुरु भुङ्क्ष्व च कर्म निजं नियतं हरिपादविनम्रधिया सततं ।\\
हरिरेव परो हरिरेव गुरुर्हरिरेव जगत् पितृमातृगतिः ॥ 01 ॥\\
\\
 न ततोऽस्त्यपरं जगदीड्यतमं परमात् परतः पुरुषोत्तमतः ।\\
तदलं बहुलोकविचिन्तनया प्रवणं कुरु मानसमीशपदे ॥ 02 ॥\\
\\
 यततोऽपि हरेः पदसंस्मरणे सकलं ह्यघमाशु लयं व्रजति ।\\
स्मरतस्तु विमुक्तिपदं परमं स्फुटमेष्यति तत् किमपाक्रियते ॥ 03 ॥\\
\\
 शृणुतामलसत्यवचः परमं शपथेरितमुच्छ्रितबाहुयुगम् ।\\
न हरेः परमो न हरेः सदृशः परमः स तु सर्वचिदात्मगणात्॥04॥\\
\\
यदि नाम परो न भवेत हरिः कथमस्य वशे जगदेतदभूत् ।\\
यदि नाम न तस्य वशे सकलं कथमेव तु नित्यसुखं न भवेत्॥ 05 ॥\\
\\
न च कर्म्मविमामलकालगुणप्रभृतीशमचित्तनु तद्धि यतः ।\\
चिदचित्तनु सर्वमसौ तु हरिर्यमयेदिति वैदिकमस्ति वचः ॥ 06 ॥\\
\\
व्यवहारभिदाऽपि गुरोर्ज्जगतां न तु चित्तगता स हि चोद्यपरम् ।\\
बहवः पुरुषाः पुरुषप्रवरो हरिरित्यवदत् स्वयमेव हरिः ॥ 07 ॥\\
\\
चतुराननपूर्वविमुक्तगणा हरिमेत्य तु पूर्ववदेव सदा ।\\
नियतोच्चविनीचतयैव निजां स्थितिमापुरिति स्म परं वचनम्॥ 08 ॥\\
\\
स्मरणे हि परेशितुरस्य विभोर्म्मलिनानि मनांसि कुतः करणम् ।\\
विमलं हि पदं परमं स्मरतं तरुणार्क्कसवर्ण्णमजस्य हरेः ॥ 09 ॥\\
\\
विमलैः श्रुतिशाणनिशाततमैः सुमनोऽसिभिराशु निहत्य दृढम् ।\\
बलिनं निजवैरिणमात्मतमोभिदमीशमनन्तमुपेत हरिम् ॥ 10 ॥\\
\\\
स हि विश्वसृजो विभुशम्भुपुरन्दरपूर्वमुखानपरानपरान् ।\\
सृजतीड्यतमोऽवति हन्ति निजं पदमापयति प्रणतान् स्वधिया ॥ 11 ॥\\
\\
परमोऽपि रमेशितुरस्य समो न हि कश्चिदभून्न भविष्यति च ।\\
क्वचिदद्यतनोऽपि न पूर्ण्णसदाऽगणितेड्यगुणानुभवैकतनोः ॥ 12 ॥\\
\\
 निजपूर्ण्णसुखामितबोधतनुः परशक्तिरनन्तगुणः परमः ।\\
अजरामरणः सकलार्त्तिहरः कमलापतिरीड्यतमोऽवतु नः ॥ 13 ॥\\
\\
 यदसुप्तिगतो वरिनः सुखवान् सुखरूपिणमाहुरतो निगमाः ।\\
स्वमतिप्रभवं जगदस्य यतः परबोधतनुं च ततः खपतिम् ॥ 14 ॥\\
\\
बहुचित्रजगद् बहुधाकरणात् परशक्तिरनन्तगुणः परमः ।\\
सुखरूपममुष्य पदं परमं स्मरतस्तु भविष्यति तत् सततम्॥ 15 ॥\\
\\
\subsection{\sanskrit पञ्चमो अध्याय}
\sanskrit
 ॐ ॥ पान्त्वस्मान् पुरुहूतवैरिबलवन्मातङ्गमाद्यद् घटा-
कुम्भोच्चाद्रिविपाटनाधिकपटुप्रत्येकवज्रायिताः ।\\
श्रीमत् कण्ठीरवास्यप्रततसुनखरा दारितारातिदूर-
प्रध्वस्तध्वान्तशान्तप्रविततमनसा भाविता भूरिभागैः ॥ * ॥\\
\[लक्ष्मीकान्त समन्ततोऽपि कलयन् नैवेशितुस्ते समं
पश्याम्युत्तमवस्तु दूरतरतोऽपास्तं रसो योऽष्टमः।\\
यद्रोषोत्करदक्षनेत्रकुटिलप्रान्तोत्थिताग्निस्फुरत्-
खद्योतोपमविष्फुलिङ्गभसिता ब्रह्मेशशक्रोत्कराः ॥ * ॥ \]\\
\\
\subsection{\sanskrit शश्टमो अध्याय}
\sanskrit
 वन्दिताशेषवन्द्योरुवृन्दारकं चन्दनाचर्चितोदारपीनांसकम् ।\\
इन्दिराचञ्चलापाङ्गनीराजितं मन्दरोद्धारिवृत्तोद्भुजाभोगिनम् ॥\\
प्रीणयामो वासुदेवं देवतामण्डलाखण्डमण्डानम् प्रीणयामो वासुदेवम् ॥ 1 ॥\\
\\
 सृष्टिसंहारलीलाविलासाततं पुष्टषाड्गुण्यसद्विग्रहोल्लासिनम् ।\\
दुष्टनिःशेषसंहारकर्म्मोद्यतं हृष्टपुष्टातिशिष्टप्रजासंश्रयम् ॥\\
प्रीणयामो वासुदेवं देवतामण्डलाखण्डमण्डानम् प्रीणयामो वासुदेवम्  ॥2॥\\
\\
उन्नतप्रार्थ्थिताशेषसंसाधकं सन्नतालौकिकानन्ददश्रीपदम् ।\\
भिन्नकर्म्माशयाप्राणिसम्प्रेरकम् तन्न किं नेति विद्वत्सु मीमांसितम् ।\\
प्रीणयामो वासुदेवं देवतामण्डलाखण्डमण्डानम् प्रीणयामो वासुदेवम् ॥3॥\\
\\
विप्रमुख्यैः सदा वेदवादोन्मुखैः सुप्रतापैः क्षितीशेश्वरैश्चार्च्चितम् ।\\
अप्रतर्क्क्योरुसंविद् गुणं निर्म्मलं सत्प्रकाशाजरानन्दरूपं परम् ॥\\
प्रीणयामो वासुदेवं देवतामण्डलाखण्डमण्डानम् प्रीणयामो वासुदेवम् ॥ 4 ॥\\
\\
 अत्ययो यस्य केनापि न क्वापि हि प्रत्ययो यद् गुणेषूत्तमानां परः ।\\
सत्यसङ्कल्प एको वरेण्यो वशी मत्यनूनैः सदा वेदवादोदितः ॥\\
प्रीणयामो वासुदेवं देवतामण्डलाखण्डमण्डानम् प्रीणयामो वासुदेवम् ॥ 5 ॥\\
\\
पश्यतां दुःखसन्ताननिर्मूलनं दृष्यतां दृश्यतामित्यजेशार्च्चिम् ।\\
नश्यतां दूरगं सर्वदाऽप्यात्मकं वश्यतां स्वेच्छया सज्जनेष्वागतम् ॥\\
प्रीणयामो वासुदेवं देवतामण्डलाखण्डमण्डानम्  प्रीणयामो वासुदेवम् ॥ 6 ॥\\
\\
अग्रजं यः ससर्ज्जाजमग्र्याकृतिं विग्रहो यस्य सर्वे गुणा एव हि ।\\
उग्र अद्योऽपि यस्याऽत्मजाग्र्यात्मजः सद्गृहीतः सदा यः परं दैवतम् ।\\
प्रीणयामो वासुदेवं देवतामण्डलाखण्डमण्डानम्  प्रीणयामो वासुदेवम् ॥ 7 ॥\\
\\
अच्युतो यो गुणैर्न्नित्यमेवाखिलैः प्रच्युतोऽशेषदोषैः सदा पूर्त्तितः ।\\
उच्यते सर्ववेदोरुवादैरजः स्वर्च्यते ब्रह्मरुद्रेन्द्र पूर्वैः सदा ॥\\
प्रीणयामो वासुदेवं देवतामण्डलाखण्डमण्डानम्  प्रीणयामो वासुदेवम् ॥ 8 ॥\\
\\
धार्यते येन विश्वं सदाऽजादिकं वार्यतेऽशेषदुःखं निजाद्ध्यायिनाम् ।\\
पार्यते सर्वमन्यैर्न्न यत् पार्यते कार्यते चाखिलं सर्वभूतैः सदा।\\
प्रीणयामो वासुदेवं देवतामण्डलाखण्डमण्डानम्  प्रीणयामो वासुदेवम् ॥ 9 ॥\\
\\
सर्वपापानि यत् संस्मृतेः सङ्क्षयं सर्वदा यान्ति भक्त्या विशुद्धात्मनाम्।\\
शर्वगुर्वादिनिर्बाणसंस्थानदः कुर्वते कर्म्म यत् प्रीतये सज्जनाः ॥\\
प्रीणयामो वासुदेवं देवतामण्डलाखण्डमण्डानम्  प्रीणयामो वासुदेवम् ॥10 ॥\\ 
\\
अक्षयं कर्म्म यस्मिन् परे स्वर्प्पितं प्रक्षयं यान्ति दुःखानि यन्नामतः ।\\
अक्षरो योऽजरः  सर्वदैवामृतः कुक्षिगं यस्य विश्वं सदाऽजादिकम् ॥\\
प्रीणयामो वासुदेवं देवतामण्डलाखण्डमण्डानम्  प्रीणयामो वासुदेवम्॥ 11 ॥\\
\\
नन्दितीर्थ्थोरुसन्नामिनो नन्दिना सन्दधानाः सदानन्ददेवे मतिम् ।\\
मन्दहासारुणापाङ्गदत्तोन्नतिं नन्दिताशेषदेवादिवृन्दं सदा ॥\\
प्रीणयामो वासुदेवं देवतामण्डलाखण्डमण्डनम्  प्रीणयामो वासुदेवम् ॥ 12 ॥\\
\\
\subsection{\sanskrit अश्टमो अध्याय}
\sanskrit
 देवकिनन्दन नन्दकुमारा वृन्दवनाञ्जन गोकुलचन्द्र ॥ 01 ॥\\
 कन्दफलाशन सुन्दररूपा नन्दितगोकुल वन्दितपाद ॥ 02 ॥\\
 इन्द्रसुतावक नन्दकहस्ता चन्दनचर्च्चित सुन्दरिनाथ ॥ 03 ॥\\
 इन्दिवरोदरदळनयनाऽऽमन्दरधारा गोविन्द वन्दे ॥ 04 ॥\\
 मत्स्यकरूप लयोदविहारिन् वेदविनेत्र चतुर्मुखवन्द्य ॥ 05 ॥\\
 कूर्मस्वरूपक मन्दरधारिन् लोकविधारक देववरेण्य ॥ 06 ॥\\
 सूकररूपक दानवशत्रो भूमिविधारक यङ्ञवराङ्ग ॥ 07 ॥\\
 देव नृसिंह हिरण्यकशत्रो सर्वभयान्तक दैवतबन्धो ॥ 08 ॥\\
 वामनवामन माणववेषा दैत्यकुलान्तक कारणभूत ॥ 09 ॥\\
 राम भृगूद्वह सूर्ज्जितदीप्ते क्षत्रकुलान्तक शम्भुवरेण्य ॥ 10 ॥\\
 राघवराघव राक्षसशत्रो मारुतिवल्लभ जानकिकान्त ॥ 11 ॥\\
 देवकिनन्दन सुन्दररूपा रुग्मिणिवल्लभ पाण्डवबन्धो ॥ 12 ॥\\
 दैत्यविमोहक नित्यसुखादे देवसुबोधक बुद्धस्वरूप ॥ 13 ॥\\
 दुष्टकुलान्तक कल्किस्वरूपा धर्म्मविवर्द्धन मूलयुगादे ॥ 14 ॥\\
 आनन्दतीर्त्थकृता हरिगाथा पापहरा शुभनित्यसुखार्त्था ॥ 15 ॥\\
\\
\subsection{\sanskrit नवमो अध्याय}
\sanskrit
 अतिमत तमॊगिरि समितिविभेदन पितमहविभूतिद गुणगणनिलय ।\\
शुभतमकथशय परम सदोदित जगदॆककारण रम रमरमणा॥ 01 ॥\\
\\
 विधिभवमुखसुरसततसुवन्दितरममनॊवल्लभ भव मम शरणम् ।\\
शुभतमकथशय परम सदोदित जगदॆककारण रम रमरमणा॥ 02 ॥\\
\\
अपरिमितगुणगणमयशरीर हॆ विगतगुणेतर भव मम शरणम् ।
शुभतमकथशय परम सदोदित जगदॆककारण रम रमरमणा॥ 03 ॥\\
\\
अगणितसुखनिधिविमलसुदेह हॆ विगतसुखेतर भव मम शरणम् ।\\
शुभतमकथशय परम सदोदित जगदॆककारण रम रमरमणा॥ 04 ॥\\
\\
 प्रचलितलयजलविहरण शाश्वत सुखमय मीन हॆ भव मम शरणम्।\\
शुभतमकथशय परम सदोदित जगदॆककारण रम रमरमणा॥ 05 ॥\\
\\
सुरदितिजसुबलविलुळितमन्दरधर परकूर्म्म हॆ भव मम शरणम् ।\\
शुभतमकथशय परम सदोदित जगदॆककारण रम रमरमणा ॥ 06 ॥\\
\\
सगिरिवरधरतळवह सुसूकर परमविबोध हॆ भव मम शरणम् ।\\
शुभतमकथशय परम सदोदित जगदॆककारण रम रमरमणा॥ 07 ॥\\
\\
अतिबलदितिसुतहृदयविभेदन जय नृहरेऽमल भव मम शरणम्।\\
शुभतमकथशय परम सदोदित जगदॆककारण रम रमरमणा॥ 08 ॥\\
\\
बलिमुखदितिसुतविजयविनाशन जगदवनाजित भव मम शरणम् ।\\
शुभतमकथशय परम सदोदित जगदॆककारण रम रमरमणा॥ 09 ॥\\
\\
अविजितकुनृपतिसमितिविखण्डन रमवर वीरप भव मम शरणम् ।\\
शुभतमकथशय परम सदोदित जगदॆककारण रम रमरमणा॥ 10 ॥\\
\\
खरतरनिशिचरदहन परामृत रघुवर मानद भव मम शरणम् ।\\
शुभतमकथशय परम सदोदित जगदॆककारण रम रमरमणा॥ 11 ॥\\
\\
सुलळिततनुवर वरद महाबल यदुवर पार्त्थप भव मम शरणम्।\\
शुभतमकथशय परम सदोदित जगदॆककारण रम रमरमणा॥ 12 ॥\\
\\
दितिसुतविमॊहन विमलविबोधन परगुणबुद्ध हॆ भव मम शरणम्।\\
शुभतमकथशय परम सदोदित जगदॆककारण रम रमरमणा॥ 13 ॥\\
\\
कलिमलहुतवह सुभग महोत्सव शरणद कल्किश भव मम शरणम् ।\\
शुभतमकथशय परम सदोदित जगदॆककारण रम रमरमणा॥ 14 ॥\\
\\
इति तव नुतिवरसततरतेर्भव शरणमुरुसुखतीर्र्थमुनेर्भगवन्।\\
शुभतमकथशय परम सदोदित जगदॆककारण रम रमरमणा॥ 15 ॥\\
\\
\subsection{\sanskrit दशमो अध्याय}
\sanskrit
 केशवकेशव शासक वन्दे पाशधरार्च्चित शूर परेश ।\\
नारायणामरतारण वन्दे कारणकारण पूर्ण्ण वरेण्य ॥ 01 ॥\\
\\
माधवमाधव शोधक वन्दे बाधक बोधक शुद्धसमाधे ।\\
गोविन्दगोविन्द पुरन्दर वन्दे स्कन्दसनन्दनवन्दितपाद ॥ 02 ॥\\
\\
विष्णु सृजिष्णु ग्रसिष्णु वि वन्दे कृष्ण सदुष्णवधिष्ण स्वधृष्णो ।\\
मधुसूदन दानवसादन वन्दे दैवतमोदन वेदितपाद ॥ 03 ॥\\
\\
त्रिविक्रम निष्क्रम विक्रम वन्दे सुक्रम सङ्क्रमहुङ्कृतवक्त्र ।\\
वामनवामन भामन वन्दे सामन सीमन शामन सानो ॥ 04 ॥\\
\\
श्रीधरश्रीधर शन्धर वन्दे भूर्द्धर वार्द्धर कन्धरधारिन् ।\\
हृषीकेश सुकेश परेश वि वन्दे शरणेश कलेश बलेश सुखेश ॥ 05 ॥\\
\\
पद्मनाभ शुभोद्भव वन्दे सम्भृतलोकभराभर भूरे ।\\
दामोदर दूरतरान्तर वन्दे दारितपारक पार परस्मात् ॥ 06 ॥\\
\\
आनन्दसुतीर्त्थमुनीन्द्र कृता हरिगीतिरियं परमादरतः ।\\
परलोकविलोकनसूर्यनिभा हरिभक्तिविवर्द्धनशौण्डतमा ॥ 07 ॥\\
\\
\subsection{\sanskrit एकदशो अध्याय}
\sanskrit
अवन श्रीपतिरप्रतिरधिकेशादिभवादे ।\\
करुणापूर्ण्ण वरप्रद चरितं ङ्ञापय मे ते ॥ 01 ॥\\
\\
सुरवन्द्याधिप सद्वर भरिताशेषगुणालम् ।\\
करुणापूर्ण्ण वरप्रद चरितं ङ्ञापय मे ते ॥ 02 ॥\\
\\
सकलध्वान्तविनाशन परमानन्दसुधाहो ।\\
करुणापूर्ण्ण वरप्रद चरितं ङ्ञापय मे ते ॥ 03 ॥\\
\\
त्रिजगत् पोत सदार्च्चितचरणाशापतिधातो ।\\
करुणापूर्ण्ण वरप्रद चरितं ङ्ञापय मे ते ॥ 04 ॥\\
\\
त्रिगुणातीत विधारक परितो देहि सुभक्तिम् ।\\
करुणापूर्ण्ण वरप्रद चरितं ङ्ञापय मे ते ॥ 05 ॥\\
\\
मरणप्राणद पालक जगदीशाव सुभक्तिम् ।\\
करुणापूर्ण्ण वरप्रद चरितं ङ्ञापय मे ते ॥ 06 ॥\\
\\
शरणं कारण भावन भव मे तात सदाऽलम् ।\\
करुणापूर्ण्ण वरप्रद चरितं ङ्ञापय मे ते ॥ 07 ॥\\
\\
तरुणादित्यसवर्ण्णक चरणाब्जामलकीर्त्ते ।\\
करुणापूर्ण्ण वरप्रद चरितं ङ्ञापय मे ते ॥ 08 ॥\\
\\
सलिलप्रोत्थसरागकमणिवर्ण्णोच्चनखादे ।\\
करुणापूर्ण्ण वरप्रद चरितं ङ्ञापय मे ते ॥ 09 ॥\\
\\
कजतूणीनिभपावनवरजङ्घामितशक्ते ।\\
करुणापूर्ण्ण वरप्रद चरितं ङ्ञापय मे ते ॥ 10 ॥\\
\\
असनोत्फुल्लसुपुष्पकसमवर्ण्णावरणांन्ते ।\\
करुणापूर्ण्ण वरप्रद चरितं ङ्ञापय मे ते ॥ 11 ॥\\
\\
शतमोदोद्भवसुन्दरवरपद्मोत्थितनाभे ।\\
करुणापूर्ण्ण वरप्रद चरितं ङ्ञापय मे ते ॥ 12 ॥\\
\\
इभहस्तप्रभशोभनपरमोरुस्थरमाळे ।\\
करुणापूर्ण्ण वरप्रद चरितं ङ्ञापय मे ते ॥ 13 ॥\\
\\
जगदागूहकपल्लवसमकुक्षे शरणादे ।\\
करुणापूर्ण्ण वरप्रद चरितं ङ्ञापय मे ते ॥ 14 ॥\\
\\
जगदम्बामलसुन्दरिगृहवक्षोवरयोगिन् ।\\
करुणापूर्ण्ण वरप्रद चरितं ङ्ञापय मे ते ॥ 15 ॥\\
\\
दितिजान्तप्रदचक्रधरगदायुग्वरबाहो ।\\
करुणापूर्ण्ण वरप्रद चरितं ङ्ञापय मे ते ॥ 16 ॥\\
\\
परमज्ञानमहानिधिवदनश्रीरमणेन्दो ।\\
करुणापूर्ण्ण वरप्रद चरितं ङ्ञापय मे ते ॥ 17 ॥\\
\\
निखिलाघौघविनाशकपरसौख्यप्रददृष्टे ।\\
करुणापूर्ण्ण वरप्रद चरितं ङ्ञापय मे ते ॥ 18 ॥\\
\\
परमानन्दसुतीर्त्थसुमुनिराजो हरिगाथाम्।\\
कृतवान् नित्यसुपूर्ण्णकपरमानन्दपदैषिन् ॥ 19 ॥\\
\\
\subsection{\sanskrit द्वादशो अध्याय}
\sanskrit
उदीर्ण्णमजरं दिव्यममृतस्यन्द्यधीशितुः ।\\
आनन्दस्य पदं वन्दे ब्रह्मेन्द्राद्यभिवन्दितम् ॥ 01 ॥\\
\\
सर्ववेदपदोद्गीतमिन्दिराधारमुत्तमम् ।\\
आनन्दस्य पदं वन्दे ब्रह्मेन्द्राद्यभिवन्दितम् ॥ 02 ॥\\
\\
सर्वदेवादिदेवस्य विदारितमहत्तमः ।\\
आनन्दस्य पदं वन्दे ब्रह्मेन्द्राद्यभिवन्दितम् ॥ 03 ॥\\
\\
उदारमादरान्नित्यमनिन्द्यं सुन्दरीपतेः ।\\
आनन्दस्य पदं वन्दे ब्रह्मेन्द्राद्यभिवन्दितम् ॥ 04 ॥\\
\\
इन्दीवरोदरनिभं सुपूर्ण्णं वादिमोहनम् ।\\
आनन्दस्य पदं वन्दे ब्रह्मेन्द्राद्यभिवन्दितम् ॥ 05 ॥\\
\\
दातृ सर्वामरैश्वर्यविमुक्त्यादेरहो वरम् ।\\
आनन्दस्य पदं वन्दे ब्रह्मेन्द्राद्यभिवन्दितम् ॥ 06 ॥\\
\\
दूराद् दूरतरं यत्तु तदेवान्तिकमन्तिकात् ।\\
आनन्दस्य पदं वन्दे ब्रह्मेन्द्राद्यभिवन्दितम् ॥ 07 ॥\\
\\
पूर्ण्णसर्वगुणैकार्ण्णमनाद्यन्तं सुरेशितुः ।\\
आनन्दस्य पदं वन्दे ब्रह्मेन्द्राद्यभिवन्दितम् ॥ 08 ॥\\
\\
आनन्दतीर्थमुनिना हरेरानन्दरूपिणः\\
कृतं स्तोत्रमिदं पुण्यं पठन्नानन्दमाप्नुयात् ॥ 09 ॥\\
 -13-
आनन्द मुकुन्द अरविन्दनयन । आनन्दतीर्त्थपरानन्दवरद ॥ 01 ॥
सुन्दरिमन्दिर गोविन्द वन्दे । आनन्दतीर्त्थपरानन्दवरद ॥ 02 ॥
चन्द्रकमन्दिरनन्दक वन्दे । आनन्दतीर्त्थपरानन्दवरद ॥ 03 ॥
चन्द्रसुरेन्द्र सुवन्दित वन्दे । आनन्दतीर्त्थपरानन्दवरद ॥ 04 ॥
वृन्दारवृन्दसुवन्दित वन्दे । आनन्दतीर्त्थपरानन्दवरद ॥ 05 ॥
इन्दिरस्यन्दनस्यन्दक वन्दे । आनन्दतीर्त्थपरानन्दवरद ॥ 06 ॥
इन्दिरानन्दक सुन्दर वन्दे । आनन्दतीर्त्थपरानन्दवरद ॥ 07 ॥
मन्दारस्यन्दितमन्दिर वन्दे । आनन्दतीर्त्थपरानन्दवरद ॥ 08 ॥
आनन्दचन्द्रिकास्यन्दक वन्दे । आनन्दतीर्त्थपरानन्दवरद ॥ 09 ॥
-14-
रामोऽखिलानन्दतनुः स एव भीमोऽतिपापेषु दुरन्तवीर्यः ।
कामः स एवाजितकामिनीषु सोमः स एवाऽत्मपदाश्रितेषु ॥01॥
 -15-
अम्बरगङ्गाचुम्बितपादः पदतलविदलितगुरुतरशकटः ।
कालियनागक्ष्वेळनिहन्ता सरसिजनवदळविकसितनयनः ॥01॥
 कालघनालीकर्बुरकायः शरशतशकलितरिपुशतनिवहः ।
सन्ततमस्मान् पातु मुरारिः सततगसमजवखगपतिनिरतः ॥02॥
 ॥ मुख्यप्राण वशे सर्वं स विष्णोर्वशगः सदा ॥


\chapter{\color{orange}\sanskrit दशावतारस्तुती }
प्रोष्ठीशविग्रह सुनिष्ठीवनोद्धत विशिष्टाम्बुचारिजलधे  ।\\
कोष्ठान्तराहितविचेष्टागमौघ परमेष्ठीडित त्वमवमाम्  ।\\
प्रेष्ठार्कसूनुमनुचेष्ठार्थमात्मविदतीष्टो युगान्तसमये  ।\\
स्थेष्ठात्मश‍ृङ्गधृतकाष्ठाम्बुवाहन वराष्टापदप्रभतनो  ॥ १॥\\
\\
खण्डीभवद्बहुळडिण्डीरजृम्भण सुचण्डी कृतो दधि महा  ।\\
काण्डाति चित्र गति शौण्डाद्य हैमरद भाण्डा प्रमेय चरित  ।\\
चण्डाश्वकण्ठमद शुण्डाल दुर्हृदय गण्डा भिखण्डाकर दो-\\
श्चण्डा मरेशहय तुण्डाकृते दृशमखण्डामलं प्रदिश मे  ॥ २॥\\
\\
कूर्माकृते त्ववतु नर्मात्म पृष्ठधृत भर्मात्म मन्दर गिरे  ।\\
धर्मावलम्बन सुधर्मासदाकलितशर्मा सुधावितरणात्  ।\\
दुर्मान राहुमुख दुर्मायि दानवसुमर्माभिभेदन पटो  ।\\
धर्मार्क कान्ति वर वर्मा भवान् भुवन निर्माण धूत विकृतिः  ॥ ३॥\\
\\
धन्वन्तरेऽङ्गरुचि धन्वन्तरेऽरितरु धन्वन्स्तरीभवसुधा-
भान्वन्तरावसथ मन्वन्तराधिकृत तन्वन्तरौषधनिधे  ।\\
दन्वन्तरङ्गशुगुदन्वन्तमाजिषु वितन्वन्ममाब्धि तनया  ।\\
सून्वन्तकात्महृदतन्वरावयव तन्वन्तरार्तिजलधौ  ॥ ४॥\\
\\
या क्षीरवार्धिमथनाक्षीणदर्पदितिजाक्षोभितामरगणा-
पेक्षाप्तयेऽजनि वलक्षांषुबिम्बजिदतीक्ष्णालकावृतमुखी  ।\\
सूक्ष्मावलग्नवसनाक्षेपकृत्कुच कटाक्षाक्षमीकृतमनो-\\
दीक्षासुराहृतसुधाक्षाणिनोऽवतुसु रूक्षेक्षणाद्धरितनुः  ॥ ५॥\\
\\
शिक्षादियुङ्निगम दीक्षासुलक्षण परिक्षाक्षमाविधिसती  ।\\
दाक्षायणी क्षमति साक्षाद्रमापिनय दाक्षेपवीक्षणविधौ  ।\\
प्रेक्षाक्षिलोभकरलाक्षार सोक्षित पदाक्षेपलक्षितधरा  ।\\
साऽक्षारितात्मतनु भूक्षारकारिनिटिलाक्षाक्षमानवतु नः  ॥ ६॥\\
\\
नीलाम्बुदाभशुभ शीलाद्रिदेहधर खेलाघृतोधधिधुनी-\\
शैलादियुक्त निखिलेला कटाद्यसुर तूलाटवीदहन ते  ।\\
कोलाकृते जलधि कालाचलावयव नीलाब्जदंष्ट्र धरणी-\\
लीलास्पदोरुतर मूलाशियोगिवर जालाभिवन्दित नमः  ॥ ७॥\\
\\
दम्भोलितीक्ष्णनख सम्भेदितेन्द्ररिपु कुम्भीन्द्र पाहि कृपया  ।\\
स्तम्भार्भ कासहनडिम्भाय दत्तवर गम्भीरनाद नृहरे  ।\\
अंभोधिजानुसरणांभोजभूपवनकुम्भीनसेशखगराट्  ।\\
कुम्भीन्द्रकृत्तिधर जम्भारिषण्मुखमुखांभोरुहाभिनुत माम्  ॥ ८॥\\
\\
पिङ्गाक्ष विक्रम तुरङ्गादि सैन्य चतुरङ्गा वलिप्त दनुजा-\\
साङ्गाध्वरस्थ बलि साङ्गावपात हृषिताङ्गा मरालिनुत ते  ।\\
श‍ृङ्गारपादनख तुङ्गाग्रभिन्न कन काङ्गाण्डपत्तितटिनी-\\
तुङ्गाति मङ्गल तरङ्गाभिभूत भज काङ्गाघ वामन नमः  ॥ ९॥\\
\\
ध्यानार्ह वामनतनोनाथ पाहि यजमाना सुरेशवसुधा-\\
दानाय याचनिक लीनार्थवाग्वशितनानासदस्यदनुज  ।\\
मीनाङ्कनिर्मलनिशानाथकोटिलसमानात्म मौञ्जिगुण कौ-\\
पीनाच्छसूत्रपदयानातपत्रकरकानम्यदण्डवरभृत्  ॥ १०॥\\
\\
धैर्याम्बुधे परशुचर्याधिकृत्तखलवर्यावनीश्वर महा-\\
शौर्याभिभूत कृतवीर्यात्मजातभुजवीर्यावलेपनिकर  ।\\
भार्यापराधकुपितार्याज्ञयागलितनार्यात्मसूगलतरो  ।\\
कार्यापराधमविचार्यार्यमौघजयिवीर्यामिता मयि दया  ॥ ११॥\\
\\
श्रीरामलक्ष्मणशुकाराम भूरवतुगौरामलामितमहो-\\
हारामरस्तुत यशोरामकान्तिसुत नोरामनोरथहर  ।\\
स्वारामवर्यरिपु वीरामयार्धिकर चीरामलावृतकटे ।\\
स्वाराम दर्शनजमारामयागतसुघोरामनोरमलब्धकलह  ॥ १२॥\\
\\
श्रीकेशवप्रदिशनाकेश जातकपिलोकेश भग्नरविभू-\\
तोकेतरार्तिहरणाकेवलार्तसुखधीकेकिकालजलद  ।\\
साकेतनाथवरपाकेरमुख्यसुत कोकेन भक्तिमतुलाम्  ।\\
राकेन्दु बिम्बमुख काकेक्षणापह हृशीकेश तेऽङ्घ्रिकमले  ॥ १३॥\\
\\
रामे नृणां हृदभिरामेनराशिकुलभीमे मनोऽद्यरमताम्  ।\\
गोमेदिनीजयितपोऽमेयगाधिसुतकामेनिविष्ट मनसि  ।\\
श्यामे सदा त्वयि जितामेयतापसजरामे गताधिकसमे  ।\\
भीमेशचापदलनामेयशौर्यजितवामेक्षणे विजयिनि  ॥ १४॥\\
\\
कान्तारगेहखलकान्तारटद्वदन कान्तालकान्तकशरम्  ।\\
कान्तारयाम्बुजनिकान्तान्ववायविधुकान्ताश्मभादिपहरे  ।\\
कान्तालिलोलदलकान्ताभिशोभितिलकान्ताभवन्तमनुसा  ।\\
कान्तानुयानजित  कान्तारदुर्गकटकान्ता रमात्ववतु माम्  ॥ १५॥\\
\\
दान्तं दशाननसुतान्तं धरामधिवसन्तं प्रचण्डतपसा  ।\\
क्लान्तं समेत्य विपिनान्तं त्ववाप यमनन्तं तपस्विपटलम्   ।\\
यान्तं भवारतिभयान्तं ममाशु भगवन्तं भरेण भजतात्  ।\\
स्वान्तं सवारिदनुजान्तं धराधरनिशान्तं स तापसवरम्  ॥ १६॥\\
\\
शम्पाभचापलवकंपास्तशत्रुबलसम्पादितामितयशाः  ।\\
शं पादतामरससम्पातिनोऽलमनुकम्पारसेन दिश मे  ।\\
सम्पातिपक्षिसहजं पापिरावणहतं पावनं यदकृथाः  ।\\
त्वं पापकूपपतितं पाहि मां तदपि पम्पासरस्तटचर  ॥ १७॥\\
\\
लोलाक्ष्यपेक्षितसुलीलाकुरङ्गवधखेलाकुतूहलगते  ।\\
स्वालापभूमिजनिबालापहार्यनुजपालाद्य भो जयजय  ।\\
बालाग्निदग्धपुरशालानिलात्मजनिफालात्तपत्तलरजो  ।\\
नीलाङ्गदादिकपिमालाकृतालिपथमूलाभ्यतीतजलधे  ॥ १८॥\\
\\
तूणीरकार्मुक कृपाणीकिणाङ्कभुजपाणीरविप्रतिमभाः  ।\\
क्षोणिधरालिनिभघोणीमुखादिघनवेणीसुरक्षणकरः  ।\\
शोणिभवन्नयन कोणीजिताम्बुनिधिपाणीरितार्हणमणि-\\
श्रेणीवृताङ्घ्रिरिह वाणीशसूनुवरवाणीस्तुतो विजयते  ॥ १९॥\\
\\
हुङ्कारपूर्वमथ टङ्कारनादमतिपङ्कावधार्यचलिता  ।\\
लङ्काशिलोच्चयविशङ्कापतद्भिदुर शङ्काऽऽस यस्य धनुषः  ।\\
लङ्काधिपोऽमनुत यं कालरात्रिमिव शङ्काशताकुलधिया  ।\\
तं कालदण्डशतसङ्काशकार्मुकशराङ्कान्वितं भज हरिम्  ॥ २०॥\\
\\
धीमानमेयतनुधामार्तमङ्गळदनामा रमाकमलभू-\\
कामारिपन्नगपकामाहिवैरिगुरुसोमादिवन्द्यमहिमा  ।\\
स्थेमादिनापगतसीमावतात्सखलसामाजरावणरिपू  ।\\
रामाभिदो हरिरभौमाकृतिः प्रतनसामादिवेदविषयः  ॥ २१॥\\
\\
दोषात्मभूवशतुराषाडतिक्रमजदोषात्मभर्तृवचसा  ।\\
पाषाणभूतमुनियोषावरात्मतनुवेषादिदायिचरणः  ।\\
नैषादयोषिदशुभेषाकृदण्डजनिदोषाचरादिशुभदो  ।\\
दोषाग्रजन्ममृतिशोषापहोऽवतु सुदोषाङ्घ्रिजातहननात्  ॥ २२॥\\
\\
वृन्दावनस्थपशुवृन्दावनं विनुतवृन्दारकैकशरणम्  ।\\
नन्दात्मजं निहतनिन्दाकृदासुरजनं दामबद्धजठरम्  ।\\
वन्दामहे वयममन्दावदातरुचिमान्दाक्षकारिवदनम्  ।\\
कुन्दालिदन्तमुत कन्दासितप्रभतनुं दावराक्षसहरम्  ॥ २३॥\\
\\
गोपालकोत्सवकृतापारभक्ष्यरससूपान्नलोपकुपिता  ।\\
शापालयापितलयापाम्बुदालिसलिलापायधारितगिरे  ।\\
स्वापाङ्गदर्शनज तापाङ्गरागयुतगोपाङ्गनांशुकहृति-\\
व्यापारशौण्ड विविधापायतस्त्वमव गोपारिजातहरण  ॥ २४॥\\
\\
कंसादिकासदवतंसावनीपतिविहिंसाकृतात्मजनुषम्  ।\\
संसारभूतमिह संसारबद्धमनसं सारचित्सुखतनुम्  ।\\
संसाधयन्तमनिशं सात्त्विकव्रजमहं सादरं बत भजे  ।\\
हंसादितापसरिरंसास्पदं परमहंसादिवन्द्यचरणम्  ॥ २५॥\\
\\
राजीवनेत्र विदुराजीव मामवतु राजीवकेतनवशम्  ।\\
वाजीभपत्तिनृपराजीरथान्वितजराजीवगर्वशमन  ।\\
वाजीशवाह सितवाजीश दैत्यतनुवाजीशभेदकरदोः  ।\\
जाजीकदम्बनवराजीवमुख्यसुमराजीसुवासितशिरः  ॥ २६॥\\
\\
कालीहृदावसथकालीयकुण्डलिपकालीस्थपादनखरा  ।\\
व्यालीनवांशुकरवालीगणारुणितकालीरुचे जय जय  ।\\
केलीलवापहृतकालीशदत्तवरनालीकदृप्तदितिभू-\\
चूलीकगोपमहिलालीतनूघुसृणधूलीकणाङ्कहृदय  ॥ २७॥\\
\\
कृष्णादिपाण्डुसुतकृष्णामनःप्रचुरतृष्णासुतृप्तिक रवाक्  ।\\
कृष्णाङ्कपालिरत कृष्णाभिधाघहर कृष्णादिषण्महिळ भोः  ।\\
पुष्णातु मामजित निष्णातवार्धिमुदनुष्णांशुमण्डल हरे  ।\\
जिष्णो गिरीन्द्रधर विष्णो वृषावरज धृष्णो भवान्करुणया  ॥ २८॥\\
\\
रामाशिरोमणिधरामासमेत बलरामानुजाभिध रतिम्  ।\\
व्योमासुरान्तकर ते मारतात दिश मे माधवाङ्घ्रिकमले  ।\\
कामार्तभौमपुररामावलीप्रणयवामाक्षिपीततनुभा  ।\\
भीमाहिनाथमुखवैमानिकाभिनुत भीमाभिवन्द्यचरण  ॥ २९॥\\
\\
सक्ष्वेळभक्ष्यभयदाक्षिश्रवोगणजलाक्षेपपाशयमनम्  ।\\
लाक्षागृहज्वलनरक्षोहिडिम्बबकभैक्षान्नपूर्वविपदः  ।\\
अक्षानुबन्धभवरूक्षाक्षरश्रवणसाक्षान्महिष्यवमती  ।\\
कक्षानुयानमधमक्ष्मापसेवनमभीक्ष्णापहासमसताम्  ॥ ३०॥\\
\\
चक्षाण एव निजपक्षाग्रभूदशशताक्षात्मजादिसुहृदा-\\
माक्षेपकारिकुनृपाक्षौहिणीशतबलाक्षोभदीक्षितमनाः  ।\\
तार्क्ष्यासिचापशरतीक्ष्णारिपूर्वनिजलक्ष्माणि चाप्यगणयन्  ।\\
वृक्षालयध्वजरिरक्षाकरो जयति लक्ष्मीपतिर्यदुपतिः  ॥ ३१॥\\
\\
बुद्धावतार कविबद्धानुकम्प कुरु बद्धाञ्जलौ मयि दयाम्  ।\\
शौद्धोदनिप्रमुखसैद्धान्तिकासुगमबौद्धागमप्रणयन  ।\\
क्रुद्धाहितासुहृतिसिद्धासिखेटधर शुद्धाश्वयान कमला  ।\\
शुद्धान्त मां रुचिपिनद्धाखिलाङ्ग निजमद्धाव कल्क्यभिध भोः  ॥ ३२॥\\
\\
सारङ्गकृत्तिधरसारङ्गवारिधर सारङ्गराजवरदा-\\
सारं गदारितरसारं गतात्ममदसारं गतौषधबलम्  ।\\
सारङ्गवत्कुसुमसारं गतं च तव सारङ्गमाङ्घ्रियुगलम्  ।\\
सारङ्गवर्णमपसारं गताब्जमदसारं गदिंस्त्वमव माम्  ॥ ३३॥\\
\\
ग्रीवास्यवाहतनुदेवाण्डजादिदशभावाभिरामचरितम्  ।\\
भावातिभव्यशुभधीवादिराजयतिभूवाग्विलासनिलयम्  ।\\
श्रीवागधीशमुखदेवाभिनम्यहरिसेवार्चनेषु पठता-\\
मावास एव भवितावाग्भवेतरसुरावासलोकनिकरे  ॥ ३४॥\\
\\
इति श्रीमद्वादिराजपूज्यचरण विरचितं\\
श्रीदशावतारस्तुतिः सम्पूर्णम्\\
 भारतीरमणमुख्यप्राणान्तर्गत श्रीकृष्णार्पणमस्तु\\

\chapter{\color{orange}\sanskrit वायुस्तुतिः अथवा खिलवायुस्तुतिः  }
॥ श्रीहरिवायुस्तुतिः ॥ \\
॥ अथ श्रीनखस्तुतिः ॥  \\
\\
पान्त्वस्मान् पुरुहूतवैरि बलवन्मातङ्ग माद्यद्घटा\\
     कुम्भोच्चाद्रि विपाटनाधिकपटु प्रत्येक वज्रायिताः ।\\
श्रीमत्कण्ठीरवास्य प्रतत सुनखरा दारितारातिदूर\\
     प्रद्ध्वस्तध्वान्त शान्त प्रवितत मनसा भावितानाकिवृन्दैः ॥ १॥ भाविता भूरिभागैः\\
\\
लक्ष्मीकान्त समन्ततोऽपिकलयन् नैवेशितुस्ते समं\\
     पश्याम्युत्तम वस्तु दूरतरतोपास्तं रसोयोऽष्टमः ।\\
यद्रोशोत्कर दक्ष नेत्र कुटिल प्रान्तोत्थिताग्नि स्फुरत्\\
     खद्योतोपम विस्फुलिङ्गभसिता ब्रह्मेशशक्रोत्कराः ॥ २॥\\
\\
          इति श्रीमदानन्दतीर्थभगवत्पादाचार्यविरचिता\\
          श्रीनृसिंहनखस्तुतिः सम्पुर्णा ।\\
 ॥ अथ श्रीहरिवायुस्तुतिः ॥\\
\\
श्रीमद्विष्ण्वङ्घ्रि निष्ठा अतिगुणगुरुतम श्रीमदानन्दतीर्थ\\
     त्रैलोक्याचार्य पादोज्ज्वल जलजलसत् पांसवोऽस्मान्पुनन्तु ।\\
वाचांयत्रप्रणेत्रीत्रिभुवनमहिता शारदा शारदेन्दुः\\
     ज्योत्स्नाभद्रस्मित श्रीधवळितककुभाप्रेमभारम्बभार ॥ १॥\\
\\
उत्कण्ठाकुण्ठकोलाहलजवविदिताजस्रसेवानुवृद्ध\\
     प्राज्ञात्मज्ञान धूतान्धतमससुमनो मौलिरत्नावळीनाम् ।\\
भक्त्युद्रेकावगाढ प्रघटनसघटात्कार सङ्घृष्यमाण\\
     प्रान्तप्राग्र्याङ्घ्रि पीठोत्थित कनकरजः पिञ्जरारञ्जिताशाः ॥ २॥\\
\\
जन्माधिव्याध्युपाधिप्रतिहतिविरहप्रापकाणां गुणानाम्\\
     अग्र्याणां अर्पकाणां चिरमुदितचिदानन्द सन्दोहदानाम् ।\\
एतेषामेशदोष प्रमुषितमनसां द्वेषिणां दूषकाणाम्\\
     दैत्यानामार्थिमन्धे तमसि विदधतां संस्तवेनास्मि शक्तः ॥ ३॥\\
\\
अस्याविष्कर्तुकामं कलिमलकलुषेऽस्मिन्जनेज्ञानमार्गम्\\
     वन्द्यं चन्द्रेन्द्ररुद्र द्युमणिफणिवयोः नायकद्यैरिहाद्य ।\\
मध्वाख्यं मन्त्रसिद्धं किमुतकृतवतो मारुतस्यावतारम्\\
     पातारं पारमेष्ट्यं पदमपविपदः प्राप्तुरापन्न पुंसाम् ॥ ४॥\\
\\
उद्यद्विद्युत्प्रचण्डां निजरुचि निकरव्याप्त लोकावकाशो\\
     बिभ्रद्भीमो भुजेयोऽभ्युदित दिनकराभाङ्गदाढ्य प्रकाण्डे ।\\
वीर्योद्धार्यां गदाग्र्यामयमिह सुमतिंवायुदेवोविदध्यात्\\
     अध्यात्मज्ञाननेता यतिवरमहितो भूमिभूषामर्णिमे ॥ ५॥\\
\\
संसारोत्तापनित्योपशमद सदय स्नेहहासाम्बुपूर\\
     प्रोद्यद्विद्यावनद्य द्युतिमणिकिरण श्रेणिसम्पूरिताशः ।\\
श्रीवत्साङ्काधि वासोचित तरसरलश्रीमदानन्दतीर्थ\\
     क्षीराम्भोधिर्विभिन्द्याद्भवदनभिमतम्भूरिमेभूति हेतुः ॥ ६॥\\
\\
मूर्धन्येषोऽन्जलिर्मे दृढतरमिहते बध्यते बन्धपाश\\
     क्षेत्रेधात्रे सुखानां भजति भुवि भविष्यद्विधात्रे द्युभर्त्रे ।\\
अत्यन्तं सन्ततं त्वं प्रदिश पदयुगे हन्त सन्ताप भाजाम्\\
     अस्माकं भक्तिमेकां भगवत उतते माधवस्याथ वायोः ॥ ७॥\\
\\
साभ्रोष्णाभीशु शुभ्रप्रभमभयनभो भूरिभूभृद्विभूतिः\\
     भ्राजिष्णुर्भूरृभूणां भवनमपि विभोऽभेदिबभ्रेबभूवे ।\\
येनभ्रोविभ्रमस्ते भ्रमयतुसुभृशं बभ्रुवद्दुर्भृताशान्\\
     भ्रान्तिर्भेदाव भासस्त्वितिभयमभि भोर्भूक्ष्यतोमायिभिक्षून् ॥ ८॥\\
\\
येऽमुम्भावम्भजन्ते सुरमुखसुजनाराधितं ते तृतीयम्\\
     भासन्ते भासुरैस्ते सहचरचलितैश्चामरैश्चारुवेशाः ।\\
वैकुण्ठे कण्ठलग्न स्थिरशुचि विलसत्कान्ति तारुण्यलीला\\
     लावण्या पूर्णकान्ता कुचभरसुलभाश्लेषसम्मोदसान्द्राः ॥ ९॥\\
\\
आनन्दान्मन्दमन्दा ददति हि मरुतः कुन्दमन्दारनन्द्यावर्ता\\
     ऽमोदान् दधानां मृदुपद मुदितोद्गीतकैः सुन्दरीणाम् ।\\
वृन्दैरावन्द्य मुक्तेन्द्वहिमगु मदनाहीन्द्र देवेन्द्रसेव्ये\\
     मौकुन्दे मन्दरेऽस्मिन्नविरतमुदयन्मोदिनां देव देव ॥ १०॥\\
\\
उत्तप्तात्युत्कटत्विट् प्रकटकटकट ध्वानसङ्घट्टनोद्यद्\\
     विद्युद्व्यूढस्फुलिङ्ग प्रकर विकिरणोत्क्वाथिते बाधिताङ्गान् ।\\
उद्गाढम्पात्यमाना तमसि तत इतः किङ्करैः पङ्किलेते\\
     पङ्क्तिर्ग्राव्णां गरिम्णां ग्लपयति हि भवद्वेषिणो विद्वदाद्य ॥ ११॥\\
\\
अस्मिन्नस्मद्गुरूणां हरिचरण चिरध्यान सन्मङ्गलानाम्\\
     युष्माकं  पार्ष्वभूमिं धृतरणरणिकः स्वर्गिसेव्यांप्रपन्नः ।\\
यस्तूदास्ते स आस्तेऽधिभवमसुलभ क्लेश निर्मूकमस्त\\
     प्रायानन्दं कथं चिन्नवसति सततं पञ्चकष्टेऽतिकष्टे ॥ १२॥\\
\\
क्षुत् क्षामान् रूक्षरक्षो रदखरनखर क्षुण्णविक्षोभिताक्षान्\\
     आमग्नानान्धकूपे क्षुरमुखमुखरैः पक्षिभिर्विक्षताङ्गान् ।\\
पूयासृन्मूत्र विष्ठा क्रिमिकुलकलिलेतत्क्षणक्षिप्त शक्त्याद्यस्त्र\\
     व्रातार्दितान् स्त्वद्विष उपजिहते वज्रकल्पा जलूकाः ॥ १३॥\\
\\
मातर्मेमातरिश्वन् पितरतुलगुरो भ्रातरिष्टाप्तबन्धो\\
     स्वामिन्सर्वान्तरात्मन्नजरजरयितः जन्ममृत्यामयानाम् ।\\
गोविन्दे देहिभक्तिं भवतिच भगवन्नूर्जितां निर्निमित्ताम्\\
     निर्व्याजां निश्चलां सद्गुणगण बृहतीं शाश्वतीमाशुदेव ॥ १४॥\\
\\
विष्णोरत्त्युत्तमत्वादखिलगुणगणैस्तत्र भक्तिङ्गरिष्ठाम्\\
     संश्लिष्टे श्रीधराभ्याममुमथ परिवारात्मना सेवकेषु ।\\
यः सन्धत्ते विरिञ्चि श्वसन विहगपानन्त रुद्रेन्द्र पूर्वे\\
     ष्वाध्यायंस्तारतम्यं स्फुटमवति सदा वायुरस्मद्गुरुस्तम् ॥ १५॥\\
\\
तत्त्वज्ञान् मुक्तिभाजः सुखयिसि हि गुरो योग्यतातारतम्यात्\\
     आधत्से मिश्रबुद्धिं स्त्रिदिवनिरयभूगोचरान्नित्यबद्धान् ।\\
तामिस्रान्धादिकाख्ये  तमसिसुबहुलं दुःखयस्यन्यथाज्ञान्\\
     विष्णोराज्ञाभिरित्थं श‍ृति शतमितिहासादि चाकर्णयामः ॥ १६॥\\
\\
वन्देऽहं तं हनूमानिति महितमहापौरुषो बाहुशालि\\
     ख्यातस्तेऽग्र्योऽवतारः सहित इह बहुब्रह्मचर्यादि धर्मैः ।\\
सस्नेहानां सहस्वानहरहरहितं निर्दहन् देहभाजाम्\\
     अंहोमोहापहो यः स्पृहयति महतीं भक्तिमद्यापि रामे ॥ १७॥\\
\\
प्राक्पञ्चाशत्सहस्रैर्व्यवहितमहितं योजनैः पर्वतं त्वम्\\
     यावत्सञ्जीवनाद्यौषध निधिमधिकप्राणलङ्कामनैषिः ।\\
अद्राक्षीदुत्पतन्तं तत उत गिरिमुत्पाटयन्तं गृहीत्वा\\
     यान्तं खे राघवाङ्घ्रौ प्रणतमपि तदैकक्षणे त्वांहिलोकः ॥ १८॥\\
\\
क्षिप्तः पश्चात्सत्सलीलं शतमतुलमते योजनानां स\\
     उच्चस्तावद्विस्तार वंश्च्यापि उपललवैव व्यग्रबुद्ध्या त्वयातः ।\\
स्वस्वस्थानस्थिताति स्थिरशकल शिलाजाल संश्लेष नष्ट\\
     छेदाङ्कः प्रागिवाभूत् कपिवरवपुषस्ते नमः कौशलाय ॥ १९॥\\
\\
दृष्ट्वा दृष्टाधिपोरः स्फुटितकनक सद्वर्म घृष्टास्थिकूटम्\\
     निष्पिष्टं हाटकाद्रि प्रकट तट तटाकाति शङ्को जनोऽभूत् ।\\
येनाजौ रावणारिप्रियनटनपटुर्मुष्टिरिष्टं प्रदेष्टुम्\\
     किंनेष्टे मे स तेऽष्टापदकट कतटित्कोटि भामृष्ट काष्ठः ॥ २०॥\\
\\
देव्यादेश प्रणीति दृहिण हरवरावद्य रक्षो विघाता\\
     ऽद्यासेवोद्यद्दयार्द्रः सहभुजमकरोद्रामनामा मुकुन्दः ।\\
दुष्प्रापे पारमेष्ठ्ये करतलमतुलं मूर्धिविन्यस्य धन्यम्\\
     तन्वन्भूयः प्रभूत प्रणय विकसिताब्जेक्षणस्त्वेक्षमाणः ॥ २१॥\\
\\
जघ्नेनिघ्नेनविघ्नो  बहुलबलबकध्वंस नाद्येनशोचत्\\
     विप्रानुक्रोश पाशैरसु विधृति सुखस्यैकचक्राजनानाम् ।\\
तस्मैतेदेव कुर्मः कुरुकुलपतये कर्मणाचप्रणामान्\\
     किर्मीरं दुर्मतीनां प्रथमं अथ च यो नर्मणा निर्ममाथ ॥ २२॥\\
\\
निर्मृद्नन्नत्य यत्नं विजरवर जरासन्ध कायास्थिसन्धीन्\\
     युद्धे त्वं स्वध्वरे वापशुमिवदमयन् विष्णु पक्षद्विडीशम् ।\\
यावत्प्रत्यक्ष भूतं निखिलमखभुजं तर्पयामासिथासौ\\
     तावत्यायोजि तृप्त्याकिमुवद भघवन् राजसूयाश्वमेधे ॥ २३॥\\
\\
क्ष्वेलाक्षीणाट्टहासहं तवरणमरिहन्नुद्गदोद्दामबाहोः\\
     बह्वक्षौहिण्य नीकक्षपण सुनिपुणं यस्य सर्वोत्तमस्य ।\\
शुष्रूशार्थं चकर्थ स्वयमयमथ संवक्तुमानन्दतीर्थ\\
     श्रीमन्नामन्समर्थस्त्वमपि हि युवयोः पादपद्मं प्रपद्ये ॥ २४॥\\
\\
दृह्यन्तींहृदृहं मां दृतमनिल बलाद्रावयन्तीमविद्या\\
     निद्रांविद्राव्य सद्यो रचनपटुमथापाद्यविद्यासमुद्र ।\\
वाग्देवी सा सुविद्या द्रविणद विदिता द्रौपदी रुद्रपत्न्यात्\\
     उद्रिक्ताद्रागभद्रा द्रहयतु दयिता पूर्वभीमाज्ञयाते ॥ २५॥\\
\\
याभ्यां शुश्रूषुरासीः  कुरुकुल जनने क्षत्रविप्रोदिताभ्याम्\\
     ब्रह्मभ्यां बृंहिताभ्यां चितसुख वपुषा कृष्णनामास्पदाभ्याम् ।\\
निर्भेदाभ्यां विशेषाद्विवचन विशयाभ्यामुभाभ्याममूभ्याम्\\
     तुभ्यं च क्षेमदेभ्यः सरिसिजविलसल्लोचनेभ्यो नमोऽस्तु ॥ २६॥\\
\\
गच्छन् सौगन्धिकार्थं पथि स हनुमतः पुच्छमच्छस्य\\
     भीमः प्रोद्धर्तुं नाशकत्स त्वमुमुरुवपुषा भीषयामास चेति ।\\
पूर्णज्ञानौजसोस्ते गुरुतमवपुषोः श्रीमदानन्दतीर्थ\\
     क्रीडामात्रं तदेतत् प्रमदद सुधियां मोहक द्वेषभाजाम् ॥ २७॥\\
\\
बह्वीः कोटीरटीकः कुटलकटुमतीनुत्कटाटोप कोपान्\\
     द्राक्चत्वं सत्वरत्वाच्चरणद गदया पोथयामासिथारीन् ।\\
उन्मथ्या तत्थ्य मिथ्यात्व वचन वचनान् उत्पथस्थांस्तथाऽयान्\\
     प्रायच्छः स्वप्रियायै प्रियतम कुसुमं प्राण तस्मै नमस्ते ॥ २८॥\\
\\
देहादुत्क्रामितानामधिपति रसतामक्रमाद्वक्रबुद्धिः\\
     क्रुद्धः क्रोधैकवश्यः क्रिमिरिव मणिमान् दुष्कृती निष्क्रियार्थम् ।\\
चक्रे भूचक्रमेत्य क्रकचमिव सतां चेतसः कष्टशास्त्रं\\
     दुस्तर्कं चक्रपाणेर्गुणगण विरहं जीवतां चाधिकृत्य ॥ २९॥\\
\\
तद्दुत्प्रेक्षानुसारात्कतिपय कुनरैरादृतोऽन्यैर्विसृष्टो\\
     ब्रह्माहं निर्गुणोऽहं वितथमिदमिति ह्येषपाशण्डवादः ।\\
तद्युक्त्याभास जाल प्रसर विषतरूद्दाहदक्षप्रमाण\\
     ज्वालामालाधरोऽग्निः पवन विजयते तेऽवतारस्तृतीयः ॥ ३०॥\\
\\
आक्रोशन्तोनिराशा भयभर विवशस्वाशयाच्छिन्नदर्पा\\
     वाशन्तो देशनाशस्विति बत कुधियां नाशमाशादशाऽशु ।\\
धावन्तोऽश्लीलशीला वितथ शपथ शापा शिवाः शान्त शौर्याः\\
     त्वद्व्याख्या सिंहनादे सपदि ददृशिरे मायि गोमायवस्ते ॥ ३१॥\\
\\
त्रिष्वप्येवावतारेष्वरिभिरपघृणं हिंसितोनिर्विकारः\\
     सर्वज्ञः सर्वशक्तिः सकलगुणगणापूर्ण रूपप्रगल्भः ।\\
स्वच्छः स्वच्छन्द मृत्युः सुखयसि सुजनं देवकिं चित्रमत्र\\
     त्राता यस्य त्रिधामा जगदुतवशगं किङ्कराः शङ्कराद्याः ॥ ३२॥\\
\\
उद्यन्मन्दस्मित श्रीर्मृदु मधुमधुरालाप पीयूषधारा\\
     पूरासेकोपशान्ता सुखसुजन मनोलोचना पीयमानं ।\\
सन्द्रक्ष्येसुन्दरं सन्दुहदिह महदानन्दं आनन्दतीर्थ\\
     श्रीमद्वक्तेन्द्रु बिम्बं दुरतनुदुदितं नित्यदाहं कदानु ॥ ३३॥\\
\\
प्राचीनाचीर्ण पुण्योच्चय चतुरतराचारतश्चारुचित्तान्\\
     अत्युच्चां रोचयन्तीं श‍ृतिचित वचनांश्राव कांश्चोद्यचुञ्चून् ।\\
व्याख्यामुत्खात दुःखां चिरमुचित महाचार्य चिन्तारतांस्ते\\
     चित्रां सच्छास्त्रकर्ताश्चरण परिचरां छ्रावयास्मांश्चकिञ्चित् ॥ ३४॥\\
\\
पीठेरत्नोकपक्लृप्ते रुचिररुचिमणि ज्योतिषा सन्निषण्णम्\\
     ब्रह्माणं भाविनं त्वां ज्वलति निजपदे वैदिकाद्या हि विद्याः ।\\
सेवन्ते मूर्तिमत्यः सुचरितचरितं भाति गन्धर्व गीतं\\
     प्रत्येकं देवसंसत्स्वपि तव भघवन्नर्तितद्द्योवधूषु ॥ ३५॥\\
\\
सानुक्रोषैरजस्रं जनिमृति निरयाद्यूर्मिमालाविलेऽस्मिन्\\
     संसाराब्धौनिमग्नांशरणमशरणानिच्छतो वीक्ष्यजन्तून् ।\\
युष्माभिः प्र्राथितः सन् जलनिधिशयनः सत्यवत्यां महर्षेः\\
     व्यक्तश्चिन्मात्र मूर्तिनखलु भगवतः प्राकृतो जातु देहः ॥ ३६॥\\
\\
अस्तव्यस्तं समस्तश‍ृति गतमधमैः रत्नपूगं यथान्धैः\\
     अर्थं लोकोपकृत्यैः गुणगणनिलयः सूत्रयामास कृत्स्नम् ।\\
योऽसौ व्यासाभिधानस्तमहमहरहः भक्तितस्त्वत्प्रसादात्\\
     सद्यो विद्योपलब्ध्यै गुरुतममगुरुं देवदेवं नमामि ॥ ३७॥\\
\\
आज्ञामन्यैरधार्यां शिरसि परिसरद्रश्मि कोटीरकोटौ\\
     कृष्णस्याक्लिष्ट कर्मादधदनु सराणादर्थितो देवसङ्घैः ।\\
भूमावागत्य भूमन्नसुकरमकरोर्ब्रह्मसूत्रस्य भाष्यम्\\
     दुर्भाष्यं व्यास्यदस्योर्मणिमत उदितं वेदसद्युक्तिभिस्त्वम् ॥ ३८॥\\
\\
भूत्वाक्षेत्रे विशुद्धे द्विजगणनिलये रौप्यपीठाभिधाने\\
     तत्रापि ब्रह्मजातिस्त्रिभुवन  विशदे मध्यगेहाख्य गेहे ।\\
पारिव्राज्याधि राजः पुनरपि बदरीं प्राप्य कृष्णं च नत्वा\\
     कृत्वा भाष्याणि सम्यक् व्यतनुत च भवान् भरतार्थप्रकाशम् ॥ ३९॥\\
\\
वन्दे तं त्वां सुपूर्ण प्रमतिमनुदिना सेवितं देववृन्दैः\\
     वन्दे वन्दारुमीशे श्रिय उत नियतं श्रीमदानन्दतीर्थम् ।\\
वन्दे मन्दाकिनी सत्सरिदमल जलासेक साधिक्य सङ्गम्\\
     वन्देऽहं देव भक्त्या भव भय दहनं सज्जनान्मोदयन्तम् ॥ ४०॥\\
\\
सुब्रह्मण्याख्य सूरेः सुत इति सुभृशं केशवानन्दतीर्थ\\
     श्रीमत्पादाब्ज भक्तः स्तुतिमकृत हरेर्वायुदेवस्य चास्य ।\\
त्वत्पादार्चादरेण ग्रथित पदल सन्मालया त्वेतयाये\\
     संराध्यामूनमन्ति प्रततमतिगुणा मुक्तिमेते व्रजन्ति ॥ ४१॥\\
\\
          इति श्रीत्रिविक्रमपण्डिताचार्य विरचितं\\
          श्रीहरिवायुस्तुतिः सम्पूर्णम् ।\\
 ॥ अथ श्री नखस्तुतिः ॥\\
\\
पान्त्वस्मान् पुरुहूतवैरि बलवन्मातङ्ग माद्यद्घटा\\
     कुम्भोच्चाद्रि विपाटनाधिकपटु प्रत्येक वज्रायिताः ।\\
श्रीमत्कण्ठीरवास्य प्रतत सुनखरा दारितारातिदूर\\
     प्रद्ध्वस्तध्वान्त शान्त प्रवितत मनसा भावितानाकिवृन्दैः ॥ १॥\\
\\
लक्ष्मीकान्त समन्ततोऽपिकलयन् नैवेशितुस्ते समम्\\
     पश्याम्युत्तम वस्तु दूरतरतोपास्तं रसोयोऽष्टमः ।\\
यद्रोशोत्कर दक्ष नेत्र कुटिलः प्रान्तोत्थिताग्नि स्फुरत्\\
     खद्योतोपम विस्फुलिङ्गभसिता ब्रह्मेशशक्रोत्कराः ॥ २॥\\
\\
इति श्रीमदानन्दतीर्थभगवत्पादाचार्यविरचितं \\
श्रीनृसिंहनखस्तुतिः सम्पुर्णम् । \\
\section{\color{blue}\sanskrit भजगोविन्दं}
\sanskrit
भजगोविन्दं भजगोविन्दं गोविन्दं भज मूढमते ।\\
संप्राप्ते सन्निहिते काले नहि नहि रक्षति डुकृञ्करणे ॥ १ ॥\\
\\
मूढ जहीहि धनागमतृष्णां कुरु सद्बुद्धिं मनसि वितृष्णाम् ।\\
यल्लभसे निजकर्मोपात्तं वित्तं तेन विनोदय चित्तम् ॥ २ ॥\\
\\
नारीस्तनभर नाभीदेशं दृष्ट्वा मागामोहावेशम् ।\\
एतन्मांसावसादि विकारं मनसि विचिन्तय वारं वारम् ॥ ३ ॥\\
\\
नलिनीदलगत जलमतितरलं तद्वज्जीवितमतिशयचपलं ।\\
विद्धि व्याध्यभिमानग्रस्तं लोकं शोकहतं च समस्तम् ॥ ४ ॥\\
\\
यावद्वित्तोपार्जन सक्तः तावन्निज परिवारो रक्तः ।\\
पश्चाज्जीवति जर्जर देहे वार्तां कोऽपि न पृच्छति गेहे ॥ ५ ॥\\
\\
यावत्पवनो निवसति देहे तावत्पृच्छति कुशलं गेहे ।\\
गतवति वायौ देहापाये भार्या बिभ्यति तस्मिन्काये ॥ ६ ॥\\
\\
बालस्तावत्क्रीडासक्तः तरुणस्तावत्तरुणीसक्तः ।\\
वृद्धस्तावत्चिन्तासक्तः परे ब्रह्मणि कोऽपि न सक्तः ॥ ७ ॥\\
\\
काते कान्ता कस्ते पुत्रः संसारोऽयमतीव विचित्रः ।\\
कस्य त्वं कः कुत आयातः तत्त्वं चिन्तय तदिह भ्रातः ॥ ८ ॥\\
\\
सत्सङ्गत्वे निस्सङ्गत्वं निस्सङ्गत्वे निर्मोहत्वम् ।\\
निर्मोहत्वे निश्चलतत्त्वं निश्चलतत्त्वे जीवन्मुक्तिः ॥ ९ ॥\\
\\
वयसिगते कः कामविकारः शुष्के नीरे कः कासारः ।\\
क्षीणेवित्ते कः परिवारः ज्ञाते तत्त्वे कः संसारः ॥ १० ॥\\
\\
मा कुरु धन जन यौवन गर्वं हरति निमेषात्कालः सर्वम् ।\\
मायामयमिदमखिलं हित्वा ब्रह्मपदं त्वं प्रविश विदित्वा ॥ ११ ॥\\
\\
दिनयामिन्यौ सायं प्रातः शिशिरवसन्तौ पुनरायातः ।\\
कालः क्रीडति गच्छत्यायुः तदपि न मुञ्चत्याशावायुः ॥ १२ ॥\\
\\
द्वादशमञ्जरिकाभिरशेषः कथितो वैयाकरणस्यैषः ।\\
उपदेशोऽभूद्विद्यानिपुणैः श्रीमच्छङ्करभगवच्छरणैः ॥ १२अ ॥\\
\\
काते कान्ता धन गतचिन्ता वातुल किं तव नास्ति नियन्ता ।\\
त्रिजगति सज्जन सङ्गतिरेका भवति भवार्णवतरणे नौका ॥ १३ ॥\\
\\
जटिलो मुण्डी लुञ्छितकेशः काषायाम्बरबहुकृतवेषः ।\\
पश्यन्नपि चन पश्यति मूढः उदरनिमित्तं बहुकृतवेषः ॥ १४ ॥\\
\\
अङ्गं गलितं पलितं मुण्डं दशनविहीनं जतं तुण्डम् ।\\
वृद्धो याति गृहीत्वा दण्डं तदपि न मुञ्चत्याशापिण्डम् ॥ १५ ॥\\
\\
अग्रे वह्निः पृष्ठेभानुः रात्रौ चुबुकसमर्पितजानुः ।\\
करतलभिक्षस्तरुतलवासः तदपि न मुञ्चत्याशापाशः ॥ १६ ॥\\
\\
कुरुते गङ्गासागरगमनं व्रतपरिपालनमथवा दानम् ।\\
ज्ञानविहिनः सर्वमतेन मुक्तिं न भजति जन्मशतेन ॥ १७ ॥\\
\\
सुर मन्दिर तरु मूल निवासः शय्या भूतलमजिनं वासः ।\\
सर्व परिग्रह भोग त्यागः कस्य सुखं न करोति विरागः ॥ १८ ॥\\
\\
योगरतो वा भोगरतोवा सङ्गरतो वा सङ्गविहीनः ।\\
यस्य ब्रह्मणि रमते चित्तं नन्दति नन्दति नन्दत्येव ॥ १९ ॥\\
\\
भगवद् गीता किञ्चिदधीता गङ्गा जललव कणिकापीता ।\\
सकृदपि येन मुरारि समर्चा क्रियते तस्य यमेन न चर्चा ॥ २० ॥\\
\\
पुनरपि जननं पुनरपि मरणं पुनरपि जननी जठरे शयनम् ।\\
इह संसारे बहुदुस्तारे कृपयाऽपारे पाहि मुरारे ॥ २१ ॥\\
\\
रथ्या चर्पट विरचित कन्थः पुण्यापुण्य विवर्जित पन्थः ।\\
योगी योगनियोजित चित्तो रमते बालोन्मत्तवदेव ॥ २२ ॥\\
\\
कस्त्वं कोऽहं कुत आयातः का मे जननी को मे तातः ।\\
इति परिभावय सर्वमसारं विश्वं त्यक्त्वा स्वप्न विचारम् ॥ २३ ॥\\
\\
त्वयि मयि चान्यत्रैको विष्णुः व्यर्थं कुप्यसि मय्यसहिष्णुः ।\\
भव समचित्तः सर्वत्र त्वं वाञ्छस्यचिराद्यदि विष्णुत्वम् ॥ २४ ॥\\
\\
शत्रौ मित्रे पुत्रे बन्धौ मा कुरु यत्नं विग्रहसन्धौ ।\\
सर्वस्मिन्नपि पश्यात्मानं सर्वत्रोत्सृज भेदाज्ञानम् ॥ २५ ॥\\
\\
कामं क्रोधं लोभं मोहं त्यक्त्वाऽत्मानं भावय कोऽहम् ।\\
आत्मज्ञान विहीना मूढाः ते पच्यन्ते नरकनिगूढाः ॥ २६ ॥\\
\\
गेयं गीता नाम सहस्रं ध्येयं श्रीपति रूपमजस्रम् ।\\
नेयं सज्जन सङ्गे चित्तं देयं दीनजनाय च वित्तम् ॥ २७ ॥\\
\\
सुखतः क्रियते रामाभोगः पश्चाद्धन्त शरीरे रोगः ।\\
यद्यपि लोके मरणं शरणं तदपि न मुञ्चति पापाचरणम् ॥ २८ ॥\\
\\
अर्थमनर्थं भावय नित्यं नास्तिततः सुखलेशः सत्यम् ।\\
पुत्रादपि धन भाजां भीतिः सर्वत्रैषा विहिता रीतिः ॥ २९ ॥\\
\\
प्राणायामं प्रत्याहारं नित्यानित्य विवेकविचारम् ।\\
जाप्यसमेत समाधिविधानं कुर्ववधानं महदवधानम् ॥ ३० ॥\\
\\
गुरुचरणाम्बुज निर्भर भक्तः संसारादचिराद्भव मुक्तः ।\\
सेन्द्रियमानस नियमादेवं द्रक्ष्यसि निज हृदयस्थं देवम् ॥ ३१ ॥\\
\\
मूढः कश्चन वैयाकरणो डुकृञ्करणाध्ययन धुरिणः ।\\
श्रीमच्छङ्कर भगवच्छिष्यै बोधित आसिच्छोधितकरणः ॥ ३२ ॥\\
\\
भजगोविन्दं भजगोविन्दं गोविन्दं भजमूढमते ।\\
नामस्मरणादन्यमुपायं नहि पश्यामो भवतरणे ॥ ३३ ॥\\


\end{document}